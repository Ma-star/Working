% !TeX spellcheck = en_US
%% 字体:方正静蕾简体
%%		 方正粗宋
\documentclass[a4paper,left=1.5cm,right=1.5cm,11pt]{article}

\usepackage[utf8]{inputenc}
\usepackage{fontspec}
\usepackage{cite}
\usepackage{xeCJK}
\usepackage{indentfirst}
\usepackage{titlesec}
\usepackage{etoolbox}%
\makeatletter
\patchcmd{\ttlh@hang}{\parindent\z@}{\parindent\z@\leavevmode}{}{}%
\patchcmd{\ttlh@hang}{\noindent}{}{}{}%
\makeatother

\usepackage{longtable}
\usepackage{empheq}
\usepackage{graphicx}
\usepackage{float}
\usepackage{rotating}
\usepackage{subfigure}
\usepackage{tabu}
\usepackage{amsmath}
\usepackage{setspace}
\usepackage{amsfonts}
\usepackage{appendix}
\usepackage{listings}
\usepackage{xcolor}
\usepackage{geometry}
\setcounter{secnumdepth}{4}
%\titleformat*{\section}{\LARGE}
%\renewcommand\refname{参考文献}
%\titleformat{\chapter}{\centering\bfseries\huge}{}{0.7em}{}{}
\titleformat{\section}{\LARGE\bf}{\thesection}{1em}{}{}
\titleformat{\subsection}{\Large\bfseries}{\thesubsection}{1em}{}{}
\titleformat{\subsubsection}{\large\bfseries}{\thesubsubsection}{1em}{}{}
\renewcommand{\contentsname}{{ \centerline{目{  } 录}}}
\setCJKfamilyfont{cjkhwxk}{STXINGKA.TTF}
%\setCJKfamilyfont{cjkhwxk}{华文行楷}
%\setCJKfamilyfont{cjkfzcs}{方正粗宋简体}
%\newcommand*{\cjkfzcs}{\CJKfamily{cjkfzcs}}
\newcommand*{\cjkhwxk}{\CJKfamily{cjkhwxk}}
%\newfontfamily\wryh{Microsoft YaHei}
%\newfontfamily\hwzs{华文中宋}
%\newfontfamily\hwst{华文宋体}
%\newfontfamily\hwfs{华文仿宋}
%\newfontfamily\jljt{方正静蕾简体}
%\newfontfamily\hwxk{华文行楷}
\newcommand{\verylarge}{\fontsize{60pt}{\baselineskip}\selectfont}  
\newcommand{\chuhao}{\fontsize{44.9pt}{\baselineskip}\selectfont}  
\newcommand{\xiaochu}{\fontsize{38.5pt}{\baselineskip}\selectfont}  
\newcommand{\yihao}{\fontsize{27.8pt}{\baselineskip}\selectfont}  
\newcommand{\xiaoyi}{\fontsize{25.7pt}{\baselineskip}\selectfont}  
\newcommand{\erhao}{\fontsize{23.5pt}{\baselineskip}\selectfont}  
\newcommand{\xiaoerhao}{\fontsize{19.3pt}{\baselineskip}\selectfont} 
\newcommand{\sihao}{\fontsize{14pt}{\baselineskip}\selectfont}      % 字号设置  
\newcommand{\xiaosihao}{\fontsize{12pt}{\baselineskip}\selectfont}  % 字号设置  
\newcommand{\wuhao}{\fontsize{10.5pt}{\baselineskip}\selectfont}    % 字号设置  
\newcommand{\xiaowuhao}{\fontsize{9pt}{\baselineskip}\selectfont}   % 字号设置  
\newcommand{\liuhao}{\fontsize{7.875pt}{\baselineskip}\selectfont}  % 字号设置  
\newcommand{\qihao}{\fontsize{5.25pt}{\baselineskip}\selectfont}    % 字号设置 

\usepackage{diagbox}
\usepackage{multirow}
\boldmath
\XeTeXlinebreaklocale "zh"
\XeTeXlinebreakskip = 0pt plus 1pt minus 0.1pt
\definecolor{cred}{rgb}{0.8,0.8,0.8}
\definecolor{cgreen}{rgb}{0,0.3,0}
\definecolor{cpurple}{rgb}{0.5,0,0.35}
\definecolor{cdocblue}{rgb}{0,0,0.3}
\definecolor{cdark}{rgb}{0.95,1.0,1.0}
\lstset{
	language=bash,
	numbers=left,
	numberstyle=\tiny\color{black},
	showspaces=false,
	showstringspaces=false,
	basicstyle=\scriptsize,
	keywordstyle=\color{purple},
	commentstyle=\itshape\color{cgreen},
	stringstyle=\color{blue},
	frame=lines,
	% escapeinside=``,
	extendedchars=true, 
	xleftmargin=1em,
	xrightmargin=1em, 
	backgroundcolor=\color{cred},
	aboveskip=1em,
	breaklines=true,
	tabsize=4
} 

%\newfontfamily{\consolas}{Consolas}
%\newfontfamily{\monaco}{Monaco}
%\setmonofont[Mapping={}]{Consolas}	%英文引号之类的正常显示,相当于设置英文字体
%\setsansfont{Consolas} %设置英文字体 Monaco, Consolas,  Fantasque Sans Mono
%\setmainfont{Times New Roman}
%\setCJKmainfont{STZHONGS.TTF}
%\setmonofont{Consolas}
% \newfontfamily{\consolas}{YaHeiConsolas.ttf}
\newfontfamily{\monaco}{MONACO.TTF}
\setCJKmainfont{STZHONGS.TTF}
%\setmainfont{MONACO.TTF}
%\setsansfont{MONACO.TTF}

\newcommand{\fic}[1]{\begin{figure}[H]
		\center
		\includegraphics[width=0.8\textwidth]{#1}
	\end{figure}}
	
\newcommand{\sizedfic}[2]{\begin{figure}[H]
		\center
		\includegraphics[width=#1\textwidth]{#2}
	\end{figure}}

\newcommand{\codefile}[1]{\lstinputlisting{#1}}

\newcommand{\interval}{\vspace{0.5em}}

\newcommand{\tablestart}{
	\interval
	\begin{longtable}{p{2cm}p{10cm}}
	\hline}
\newcommand{\tableend}{
	\hline
	\end{longtable}
	\interval}

% 改变段间隔
\setlength{\parskip}{0.2em}
\linespread{1.1}

\usepackage{lastpage}
\usepackage{fancyhdr}
\pagestyle{fancy}
\lhead{\space \qquad \space}
\chead{恒天云3.8.1环境安装和配置\qquad}
\rhead{\qquad\thepage/\pageref{LastPage}}

\begin{document}

\tableofcontents

\clearpage
\section{控制节点hty-controller}
\subsection{配置网络}
\begin{lstlisting}
# This file describes the network interfaces available on your system
# and how to activate them. For more information, see interfaces(5).

# The loopback network interface
auto lo
iface lo inet loopback

# The primary network interface
auto eth0
iface eth0 inet static
address 外网ip
netmask 外网掩码
gateway 外网网关
dns-nameservers 域名服务器ip

auto eth1
iface eth1 inet static
address 10.10.10.10
netmask 255.255.255.0
\end{lstlisting}
\subsection{配置/etc/hosts}
\begin{lstlisting}
127.0.0.1       localhost
10.10.10.11     hty-compute1
10.10.10.12     hty-compute2

10.10.10.10 hty-controller hty-keystone hty-ntp hty-glance hty-nova hty-mysql hty-mq hty-ceilometer hty-cinder
\end{lstlisting}
\subsection{配置/etc/nova/nova.conf}
\begin{lstlisting}
my_ip = 10.10.10.10
\end{lstlisting}
\subsection{配置/etc/cinder/cinder.conf}
\begin{lstlisting}
my_ip = 10.10.10.10
\end{lstlisting}
\subsection{配置监控/usr/local/nagios/libexec/hty_monitor_client/config.py}
\begin{lstlisting}
#控制节点两块网卡,注释掉eth2
monitor_interfaces = {
                       'public': "eth0",
                       'manage': "eth1",
                      # 'data': "eth2",
                      # 'store': "eth2",
                      }
\end{lstlisting}
\subseciton{配置/usr/share/openstack_dashboard/openstack_dashboard/local/htyun_settings.py}
\begin{lstlisting}
#flatDHCP模式注释掉eth2
NETWORK = {
    'type': 'vlan',
    'vlan_start': 100,
    'vlan_end': 200,
    'bridge_interface': 'eth2',
    'ip_address': '10.0.10.0',
}
SSH = {
    'username': 'root',
    'password': 'hengtian',
    'timeout': 1,
}
ADMIN_EMAIL_ADDRESS = 'xiaojunguan@hengtiansoft.com'
NAGIOS_SERVER = {'regionOne': '172.16.19.248'}
\end{lstlisting}
\subsection{配置/usr/share/openstack_dashboard/openstack_dashboard/local/local_settings.py}
\begin{lstlisting}
MAIL_HOST = 'mail.hengtiansoft.com'
MAIL_PORT = 25
MAIL_SENDER = '172.16.19.230@hengtiansoft.com'
#MAIL_PASSWORD = '<PASSWORD>'
\end{lstlisting}
\subsection{配置/etc/hengtianyun/hengtianyun.conf}
\begin{lstlisting}
region_monitor_mapping = regionOne:172.16.19.248
[mail]
host = mail.hengtiansoft.com
port = 25
sender = 172.16.19.230@hengtiansoft.com
password = sender_password
[mail]
host = mail.hengtiansoft.com
port = 25
sender = 172.16.19.230@hengtiansoft.com
password = sender_password
\end{lstlisting}
\subsection{重启服务}
\begin{lstlisting}
bash hty-controller_service.sh
\end{lstlisting}
\subsection{计算节点hty-compute1}
\subsection{配置网络}
flatDHCP模式配置两张网卡:\\
\begin{lstlisting}
# This file describes the network interfaces available on your system
# and how to activate them. For more information, see interfaces(5).

# The loopback network interface
auto lo
iface lo inet loopback

# The primary network interface
auto eth0
iface eth0 inet static
address 外网ip
netmask 外网掩码
gateway 外网网关
dns-nameservers 域名服务器ip

auto eth1
iface eth1 inet static
address 10.10.10.11
netmask 255.255.255.0
\end{lstlisting}
vlan模式要额外多配置一张trunk模式的网卡:\\
\begin{lstlisting}
#This file describes the network interfaces available on your system
# and how to activate them. For more information, see interfaces(5).

# The loopback network interface
auto lo
iface lo inet loopback

# The primary network interface
auto eth0
iface eth0 inet static
address 外网ip
netmask 外网掩码
gateway 外网网关
dns-nameservers 域名服务器ip

auto eth1
iface eth1 inet static
address 10.10.10.11
netmask 255.255.255.0

auto eth2
iface eth2 inet static
address 0.0.0.0
netmask 255.255.255.0
\end{lstlisting}
\subsection{配置/etc/hosts}
\begin{lstlisting}
127.0.0.1       localhost
10.10.10.11     hty-compute1
10.10.10.12     hty-compute2

10.10.10.10 hty-controller hty-keystone hty-ntp hty-glance hty-nova hty-mysql hty-mq hty-ceilometer hty-cinder
\end{lstlisting}
\subsection{配置/etc/nova/nova.conf}
\begin{lstlisting}
my_ip = 10.10.10.11
vnc_enabled = True
vncserver_listen = 0.0.0.0
vncserver_proxyclient_address = 10.10.10.11
novncproxy_base_url = http://控制节点外网ip:6080/vnc_auto.html

#network_manager = nova.network.manager.FlatDHCPManager
network_manager = nova.network.manager.VlanManager

flat_interface = eth2
public_interface = eth0
#最后一行
virtio_image_id=<VIRTIO_IMAGE_ID>
\end{lstlisting}
\subsection{配置/etc/nova/cinder.conf}
\begin{lstlisting}
my_ip = 10.10.10.11
\end{lstlisting}
\subsection{配置监控/usr/local/nagios/libexec/hty_monitor_client/config.py}
\begin{lstlisting}
#vlan模式启用eth2
monitor_interfaces = {
                       'public': "eth0",
                       'manage': "eth1",
                       'data': "eth2",
                      # 'store': "eth2",
                      }
service nrpe restart
\end{lstlisting}
\subsection{配置断电恢复/root/watch_services/check.conf}
\begin{lstlisting}
compute_nodes=(hty-compute1 hty-compute2)
mail_sender=hengyun@hengtiansoft.com

mail_host=mail.hengtiansoft.com
bash /root/watch_services/deploy.sh
\end{lstlisting}
\subsection{重启所有服务}
\begin{lstlisting}
bash hty_compute_service.sh
\end{lstlisting}
\end{document}