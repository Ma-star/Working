% !TeX spellcheck = en_US
%% 字体:方正静蕾简体
%%		 方正粗宋
\documentclass[a4paper,left=1.5cm,right=1.5cm,11pt]{article}

\usepackage[utf8]{inputenc}
\usepackage{fontspec}
\usepackage{cite}
\usepackage{xeCJK}
\usepackage{indentfirst}
\usepackage{titlesec}
\usepackage{etoolbox}%
\makeatletter
\patchcmd{\ttlh@hang}{\parindent\z@}{\parindent\z@\leavevmode}{}{}%
\patchcmd{\ttlh@hang}{\noindent}{}{}{}%
\makeatother

\usepackage{longtable}
\usepackage{empheq}
\usepackage{graphicx}
\usepackage{float}
\usepackage{rotating}
\usepackage{subfigure}
\usepackage{tabu}
\usepackage{amsmath}
\usepackage{setspace}
\usepackage{amsfonts}
\usepackage{appendix}
\usepackage{listings}
\usepackage{xcolor}
\usepackage{geometry}
\setcounter{secnumdepth}{4}
%\titleformat*{\section}{\LARGE}
%\renewcommand\refname{参考文献}
%\titleformat{\chapter}{\centering\bfseries\huge}{}{0.7em}{}{}
\titleformat{\section}{\LARGE\bf}{\thesection}{1em}{}{}
\titleformat{\subsection}{\Large\bfseries}{\thesubsection}{1em}{}{}
\titleformat{\subsubsection}{\large\bfseries}{\thesubsubsection}{1em}{}{}
\renewcommand{\contentsname}{{ \centerline{目{  } 录}}}
\setCJKfamilyfont{cjkhwxk}{STXINGKA.TTF}
%\setCJKfamilyfont{cjkhwxk}{华文行楷}
%\setCJKfamilyfont{cjkfzcs}{方正粗宋简体}
%\newcommand*{\cjkfzcs}{\CJKfamily{cjkfzcs}}
\newcommand*{\cjkhwxk}{\CJKfamily{cjkhwxk}}
%\newfontfamily\wryh{Microsoft YaHei}
%\newfontfamily\hwzs{华文中宋}
%\newfontfamily\hwst{华文宋体}
%\newfontfamily\hwfs{华文仿宋}
%\newfontfamily\jljt{方正静蕾简体}
%\newfontfamily\hwxk{华文行楷}
\newcommand{\verylarge}{\fontsize{60pt}{\baselineskip}\selectfont}  
\newcommand{\chuhao}{\fontsize{44.9pt}{\baselineskip}\selectfont}  
\newcommand{\xiaochu}{\fontsize{38.5pt}{\baselineskip}\selectfont}  
\newcommand{\yihao}{\fontsize{27.8pt}{\baselineskip}\selectfont}  
\newcommand{\xiaoyi}{\fontsize{25.7pt}{\baselineskip}\selectfont}  
\newcommand{\erhao}{\fontsize{23.5pt}{\baselineskip}\selectfont}  
\newcommand{\xiaoerhao}{\fontsize{19.3pt}{\baselineskip}\selectfont} 
\newcommand{\sihao}{\fontsize{14pt}{\baselineskip}\selectfont}      % 字号设置  
\newcommand{\xiaosihao}{\fontsize{12pt}{\baselineskip}\selectfont}  % 字号设置  
\newcommand{\wuhao}{\fontsize{10.5pt}{\baselineskip}\selectfont}    % 字号设置  
\newcommand{\xiaowuhao}{\fontsize{9pt}{\baselineskip}\selectfont}   % 字号设置  
\newcommand{\liuhao}{\fontsize{7.875pt}{\baselineskip}\selectfont}  % 字号设置  
\newcommand{\qihao}{\fontsize{5.25pt}{\baselineskip}\selectfont}    % 字号设置 

\usepackage{diagbox}
\usepackage{multirow}
\boldmath
\XeTeXlinebreaklocale "zh"
\XeTeXlinebreakskip = 0pt plus 1pt minus 0.1pt
\definecolor{cred}{rgb}{0.8,0.8,0.8}
\definecolor{cgreen}{rgb}{0,0.3,0}
\definecolor{cpurple}{rgb}{0.5,0,0.35}
\definecolor{cdocblue}{rgb}{0,0,0.3}
\definecolor{cdark}{rgb}{0.95,1.0,1.0}
\lstset{
	language=bash,
	numbers=left,
	numberstyle=\tiny\color{black},
	showspaces=false,
	showstringspaces=false,
	basicstyle=\scriptsize,
	keywordstyle=\color{purple},
	commentstyle=\itshape\color{cgreen},
	stringstyle=\color{blue},
	frame=lines,
	% escapeinside=``,
	extendedchars=true, 
	xleftmargin=1em,
	xrightmargin=1em, 
	backgroundcolor=\color{cred},
	aboveskip=1em,
	breaklines=true,
	tabsize=4
} 

%\newfontfamily{\consolas}{Consolas}
%\newfontfamily{\monaco}{Monaco}
%\setmonofont[Mapping={}]{Consolas}	%英文引号之类的正常显示,相当于设置英文字体
%\setsansfont{Consolas} %设置英文字体 Monaco, Consolas,  Fantasque Sans Mono
%\setmainfont{Times New Roman}
%\setCJKmainfont{STZHONGS.TTF}
%\setmonofont{Consolas}
% \newfontfamily{\consolas}{YaHeiConsolas.ttf}
\newfontfamily{\monaco}{MONACO.TTF}
\setCJKmainfont{STZHONGS.TTF}
%\setmainfont{MONACO.TTF}
%\setsansfont{MONACO.TTF}

\newcommand{\fic}[1]{\begin{figure}[H]
		\center
		\includegraphics[width=0.8\textwidth]{#1}
	\end{figure}}
	
\newcommand{\sizedfic}[2]{\begin{figure}[H]
		\center
		\includegraphics[width=#1\textwidth]{#2}
	\end{figure}}

\newcommand{\codefile}[1]{\lstinputlisting{#1}}

\newcommand{\interval}{\vspace{0.5em}}

\newcommand{\tablestart}{
	\interval
	\begin{longtable}{p{2cm}p{10cm}}
	\hline}
\newcommand{\tableend}{
	\hline
	\end{longtable}
	\interval}

% 改变段间隔
\setlength{\parskip}{0.2em}
\linespread{1.1}

\usepackage{lastpage}
\usepackage{fancyhdr}
\pagestyle{fancy}
\lhead{\space \qquad \space}
\chead{ipv6相关知识\qquad}
\rhead{\qquad\thepage/\pageref{LastPage}}

\begin{document}

\tableofcontents

\clearpage

\subsection{ipv6基础知识}
    \begin{itemize}
        \item[1.]一共有128位,每16位一个地址段,用:隔开
        \item[2.]分级结构:可聚合全局单点广播地址
        \item[3.]自动获取地址:全状态自动配置和无状态自动配置
        \item[4.]ipv6安全协议:IPSec,提供认证和加密
    \end{itemize}
\subsection{ipv4到ipv6}
    \begin{itemize}
        \item[1.]段时间内将ipv4升级到ipv6是不可能的,ipv6和ipv4长期共存是不可避免的。
        \item[2.]ipv4向ipv6过渡的方案,包括3个机制:兼容ipv4的ipv6地址,双ip协议栈,基于ipv4隧道的ipv6
		\item[3.]兼容ipv4的ipv6地址。\par
			兼容IPv4的IPv6地址是一种特殊的IPv6单点广播地址,一个IPv6节点与一个IPv4节点可以使用这种地址在IPv4网络中通信。\par
			这种地址是由96个0位加上32位IPv4地址组成的,例如,假设某节点的IPv4地址是192.56.1.1,\par
			那么兼容IPv4的IPv6地址就是0:0:0:0:0:0:C038:101。
		\item[4.]双ip协议栈。 \par
			双IP协议栈是在一个系统(如一个主机或一个路由器)中同时使用IPv4和IPv6两个协议栈。\par
			这类系统既拥有 IPv4地址,也拥有IPv6地址,因而可以收发IPv4和IPv6两种IP数据报。
		\item[5.]基于ipv4隧道的ipv6。\par
			它是将整个IPv6数据报封装在IPv4数据报中,由此实现在当前的IPv4网络(如Internet)中IPv6节点与IPv4节点之间的IP通信。\par
			分为三个步骤:封装,解封,隧道管理 \par
			封装:指由隧道起始点创建一个IPv4包头,将IPv6数据报装入一个新的IPv4数据报中\par
			解封:由隧道终结点移去IPv4包头,还原原始的IPv6数据报\par
			隧道管理:隧道起始点维护隧道的配置信息\par
			IPv4隧道有四种方案:路由器对路由器、主机对路由器、主机对主机、路由器对主机
    \end{itemize}
\subsection{ipv6地址类型}
	\begin{itemize}
		\item[1.]单点传送:单个接口的地址。发送一个单点传送地址的信息包只会送到地址为这个的接口
		\item[2.]任意点传送:一组接口的地址。发送到这组地址中的一个(根据路由距离的远近来选择)(ipv4中所没有的机制)
		\item[3.]多点传送:一组接口的地址。,发送到这组地址中的所有(类似广播地址)
	\end{itemize}
\subsection{ipv6地址表示}
	\begin{itemize}
		\item[1.]把IPv6地址的128位(16个字节)写成8个16位的无符号整数,每个整数用四个十六进制位表示,这些数之间用冒号(:)分开,\par
				例如:3ffe:3201:1401:1:280:c8ff:fe4d:db39
		\item[2.]为了进一步简化IPv6的地址表示,可以用0来表示0000,用1来表示0001,用20来表示0020, 用300来表示0300,只要保证数值不便,就可以将前面的0省略\par
				比如:\par
					1080:0000:0000:0000:0008:0800:200C:417A\par
					0000:0000:0000:0000:0000:0000:0A00:0001\par
				可以简写成:\par
					1080:0:0:0:8:800:200C:417A \par
					0:0:0:0:0:0:A00:1 \par
				另外,还规定可以用符号::表示一系列的0。那么上面的地址又可以简化为:\par
					1080::0:8:800:200C:417A \par
					::A00:1 \par
		\item[3.]IPv6地址的前缀(FP, Format Prefix)的表示和IPv4地址前缀在CIDR中的表示方法类似。比如 0020:0250:f002::/48表示一个前缀为48位的网络地址空间。\par
	\end{itemize}
\end{document}