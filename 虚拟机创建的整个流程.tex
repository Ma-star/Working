% !TeX spellcheck = en_US
%% 字体:方正静蕾简体
%%		 方正粗宋
\documentclass[a4paper,left=1.5cm,right=1.5cm,11pt]{article}

\usepackage[utf8]{inputenc}
\usepackage{fontspec}
\usepackage{cite}
\usepackage{xeCJK}
\usepackage{indentfirst}
\usepackage{titlesec}
\usepackage{etoolbox}%
\makeatletter
\patchcmd{\ttlh@hang}{\parindent\z@}{\parindent\z@\leavevmode}{}{}%
\patchcmd{\ttlh@hang}{\noindent}{}{}{}%
\makeatother

\usepackage{longtable}
\usepackage{empheq}
\usepackage{graphicx}
\usepackage{float}
\usepackage{rotating}
\usepackage{subfigure}
\usepackage{tabu}
\usepackage{amsmath}
\usepackage{setspace}
\usepackage{amsfonts}
\usepackage{appendix}
\usepackage{listings}
\usepackage{xcolor}
\usepackage{geometry}
\setcounter{secnumdepth}{4}
%\titleformat*{\section}{\LARGE}
%\renewcommand\refname{参考文献}
%\titleformat{\chapter}{\centering\bfseries\huge}{}{0.7em}{}{}
\titleformat{\section}{\LARGE\bf}{\thesection}{1em}{}{}
\titleformat{\subsection}{\Large\bfseries}{\thesubsection}{1em}{}{}
\titleformat{\subsubsection}{\large\bfseries}{\thesubsubsection}{1em}{}{}
\renewcommand{\contentsname}{{ \centerline{目{  } 录}}}
\setCJKfamilyfont{cjkhwxk}{STXINGKA.TTF}
%\setCJKfamilyfont{cjkhwxk}{华文行楷}
%\setCJKfamilyfont{cjkfzcs}{方正粗宋简体}
%\newcommand*{\cjkfzcs}{\CJKfamily{cjkfzcs}}
\newcommand*{\cjkhwxk}{\CJKfamily{cjkhwxk}}
%\newfontfamily\wryh{Microsoft YaHei}
%\newfontfamily\hwzs{华文中宋}
%\newfontfamily\hwst{华文宋体}
%\newfontfamily\hwfs{华文仿宋}
%\newfontfamily\jljt{方正静蕾简体}
%\newfontfamily\hwxk{华文行楷}
\newcommand{\verylarge}{\fontsize{60pt}{\baselineskip}\selectfont}  
\newcommand{\chuhao}{\fontsize{44.9pt}{\baselineskip}\selectfont}  
\newcommand{\xiaochu}{\fontsize{38.5pt}{\baselineskip}\selectfont}  
\newcommand{\yihao}{\fontsize{27.8pt}{\baselineskip}\selectfont}  
\newcommand{\xiaoyi}{\fontsize{25.7pt}{\baselineskip}\selectfont}  
\newcommand{\erhao}{\fontsize{23.5pt}{\baselineskip}\selectfont}  
\newcommand{\xiaoerhao}{\fontsize{19.3pt}{\baselineskip}\selectfont} 
\newcommand{\sihao}{\fontsize{14pt}{\baselineskip}\selectfont}      % 字号设置  
\newcommand{\xiaosihao}{\fontsize{12pt}{\baselineskip}\selectfont}  % 字号设置  
\newcommand{\wuhao}{\fontsize{10.5pt}{\baselineskip}\selectfont}    % 字号设置  
\newcommand{\xiaowuhao}{\fontsize{9pt}{\baselineskip}\selectfont}   % 字号设置  
\newcommand{\liuhao}{\fontsize{7.875pt}{\baselineskip}\selectfont}  % 字号设置  
\newcommand{\qihao}{\fontsize{5.25pt}{\baselineskip}\selectfont}    % 字号设置 

\usepackage{diagbox}
\usepackage{multirow}
\boldmath
\XeTeXlinebreaklocale "zh"
\XeTeXlinebreakskip = 0pt plus 1pt minus 0.1pt
\definecolor{cred}{rgb}{0.8,0.8,0.8}
\definecolor{cgreen}{rgb}{0,0.3,0}
\definecolor{cpurple}{rgb}{0.5,0,0.35}
\definecolor{cdocblue}{rgb}{0,0,0.3}
\definecolor{cdark}{rgb}{0.95,1.0,1.0}
\lstset{
	language=bash,
	numbers=left,
	numberstyle=\tiny\color{black},
	showspaces=false,
	showstringspaces=false,
	basicstyle=\scriptsize,
	keywordstyle=\color{purple},
	commentstyle=\itshape\color{cgreen},
	stringstyle=\color{blue},
	frame=lines,
	% escapeinside=``,
	extendedchars=true, 
	xleftmargin=1em,
	xrightmargin=1em, 
	backgroundcolor=\color{cred},
	aboveskip=1em,
	breaklines=true,
	tabsize=4
} 

%\newfontfamily{\consolas}{Consolas}
%\newfontfamily{\monaco}{Monaco}
%\setmonofont[Mapping={}]{Consolas}	%英文引号之类的正常显示,相当于设置英文字体
%\setsansfont{Consolas} %设置英文字体 Monaco, Consolas,  Fantasque Sans Mono
%\setmainfont{Times New Roman}
%\setCJKmainfont{STZHONGS.TTF}
%\setmonofont{Consolas}
% \newfontfamily{\consolas}{YaHeiConsolas.ttf}
\newfontfamily{\monaco}{MONACO.TTF}
\setCJKmainfont{STZHONGS.TTF}
%\setmainfont{MONACO.TTF}
%\setsansfont{MONACO.TTF}

\newcommand{\fic}[1]{\begin{figure}[H]
		\center
		\includegraphics[width=0.8\textwidth]{#1}
	\end{figure}}
	
\newcommand{\sizedfic}[2]{\begin{figure}[H]
		\center
		\includegraphics[width=#1\textwidth]{#2}
	\end{figure}}

\newcommand{\codefile}[1]{\lstinputlisting{#1}}

\newcommand{\interval}{\vspace{0.5em}}

\newcommand{\tablestart}{
	\interval
	\begin{longtable}{p{2cm}p{10cm}}
	\hline}
\newcommand{\tableend}{
	\hline
	\end{longtable}
	\interval}

% 改变段间隔
\setlength{\parskip}{0.2em}
\linespread{1.1}

\usepackage{lastpage}
\usepackage{fancyhdr}
\pagestyle{fancy}
\lhead{\space \qquad \space}
\chead{虚拟机创建的整个流程\qquad}
\rhead{\qquad\thepage/\pageref{LastPage}}

\begin{document}

\tableofcontents

\clearpage

\subsection{}
   \begin{nova的体系结构}
        \item[1.]目前的nova主要由API,compute,conductor,scheduler四个核心组成,它们之间通过AMQP消息队列进行通信。
        \item[2.]API是进入Nova的HTTP接口
        \item[3.]compute和VMM交互来运行虚拟机并管理虚拟机的生命周期。
        \item[4.]scheduler从可用池中选择最合适的计算节点来创建新的虚拟机实例
        \item[5.]conductor为数据库的访问提供一层安全保障
		\item[6.]scheduler只是读取数据库的内容,api则有Policy的保护,因此它们都可以直接操作数据库。不用经过conductor,但openstack还是希望
		涉及数据库的操作都通过conductor
		\item[7.]client是openstack为了简化对restful api的使用所提供的api封装novaclient,负责将用户的请求转换标准的http请求。
    \end{itemize}
\subsection{创建虚拟机的一个例子}
    \begin{itemize}
        \item[1.]用户执行novaclient提供的创建虚拟机的命令,api服务监听到novaclient发送的http请求并且将它转换成AMQP消息,通过消息队列调用conductor服务。
        \item[2.]conductor服务通过消息队列接受到任务之后,做一些准备工作,例如汇总虚拟机参数,再通过消息队列告诉scheduler去选择一个满足虚拟机创建要求的主机。
        \item[3.]conductor拿到scheduler提供的目标主机之后,会去要求compute服务创建虚拟机。
    \end{itemize}
\subsection{nova-api执行过程}
	nova api的执行过程主要分为3个阶段:
	 \begin{itemize}
        \item[1.]novaclient将用户命令转换为标准http请求的阶段
        \item[2.]paste deploy将请求路由到具体的wsgi application的阶段
        \item[3.]routes将请求路由到具体函数并执行的阶段
    \end{itemize}
\subsection{nova-compute创建虚拟机的过程}
	create instance 创建虚拟机
    \begin{itemize}
        \item[1.]compute.api.create_db_entry_for_new_instance:为新的实例创建数据库条目
        \item[2.]compute.manager._start_building:开始创建
        \item[3.]compute.manager._allocate_network:分配网络
		\item[4.]compute.manager._prep_block_device:准备块设备
		\item[5.]compute.manager._spawn:调用virt driver创建虚拟机
		\item[6.]
    \end{itemize}
\subsection{instance创建的具体流程}
\begin{itemize}
        \item[1.]keystone认证
		\item[2.]生成Token
		\item[3.]nova-api接受请求
		\item[4.]验证token
		\item[5.]check policies检查访问权限
		\item[6.]check quota
		\item[7.]create instance db entry 创建实例的数据库条目
		\item[8.]创建filter_properties,用于scheduler
		\item[9.]发送rpc给nova-conductor
		\item[10.]nova-conductor创建request_spec,用于scheduler
		\item[11.]nova-conductor发送rpc给nova-scheduler
		\item[12.]nova-scheduler选择物理机
		\item[13.]对Host进行Filtering
		\item[14.]对合适的Host进行Weighting并排序
		\item[15.]发送RPC给选择出的nova-compute
		\item[16.]nova-compute接受请求
		\item[17.]resource_tracker claims resources for the new instance and update resources usages
		\item[18.]调用neutron api 配置network,instance处于networking状态
		\item[19.]生成MAC Address
		\item[20.]获取DHCP Server配置
		\item[21.]获取network信息
		\item[22.]获取security group 信息
		\item[23.]根据上面搜集的信息创建port,配置了fixed ip
		\item[24.]nova-compute调用libvirt driver spawn instance
		\item[25.]libvirt driver 创建image
		\item[26.]调用glance下载image,成为base,格式为raw
		\item[27.]基于base image,用qemu-img命令创建一个qcow2的image
		\item[28.]按照flavor的大小resize image
		\item[29.]配置configuration drive
		\item[30.]配置injection file
		\item[31.]Libvirt Driver定义XML
		\item[32.]Libvirt Driver将instance连接到网络设备上
		\item[33.]创建qbr网桥
		\item[34.]创建veth pair:qvb和qvo
		\item[35.]将qvb添加到qbr上
		\item[36.]将qvo添加到br-int上
		\item[37.]Libvirt启动VM
		\item[38.]VM从DHCP Server获取IP
		\item[39.]cloud-init连接metadata server,注入key
		\item[40.]通过VNC可以看到启动过程
		\item[41.]attach一个floating ip,然后ssh进去
		\item[42.]vm内部可以访问外网
		\item[43.]添加一个cinder volume
		\item[44.]cinder api接受到请求创建一个volume
		\item[45.]cinder api调用cinder scheduler选择一个cinder volume
		\item[46.]cinder volume创建一个iscsi target
		\item[47.]cinder volume创建一个logic volume,加入iscis target
		\item[48.]compute node连接那个iscis target,disk 出现在host上
		\item[49.]将这个disk attach到虚拟机上
		\item[50.]
\end{itemize}	
\end{document}