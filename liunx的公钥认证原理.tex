% !TeX spellcheck = en_US
%% 字体:方正静蕾简体
%%		 方正粗宋
\documentclass[a4paper,left=1.5cm,right=1.5cm,11pt]{article}

\usepackage[utf8]{inputenc}
\usepackage{fontspec}
\usepackage{cite}
\usepackage{xeCJK}
\usepackage{indentfirst}
\usepackage{titlesec}
\usepackage{etoolbox}%
\makeatletter
\patchcmd{\ttlh@hang}{\parindent\z@}{\parindent\z@\leavevmode}{}{}%
\patchcmd{\ttlh@hang}{\noindent}{}{}{}%
\makeatother

\usepackage{longtable}
\usepackage{empheq}
\usepackage{graphicx}
\usepackage{float}
\usepackage{rotating}
\usepackage{subfigure}
\usepackage{tabu}
\usepackage{amsmath}
\usepackage{setspace}
\usepackage{amsfonts}
\usepackage{appendix}
\usepackage{listings}
\usepackage{xcolor}
\usepackage{geometry}
\setcounter{secnumdepth}{4}
%\titleformat*{\section}{\LARGE}
%\renewcommand\refname{参考文献}
%\titleformat{\chapter}{\centering\bfseries\huge}{}{0.7em}{}{}
\titleformat{\section}{\LARGE\bf}{\thesection}{1em}{}{}
\titleformat{\subsection}{\Large\bfseries}{\thesubsection}{1em}{}{}
\titleformat{\subsubsection}{\large\bfseries}{\thesubsubsection}{1em}{}{}
\renewcommand{\contentsname}{{ \centerline{目{  } 录}}}
\setCJKfamilyfont{cjkhwxk}{STXINGKA.TTF}
%\setCJKfamilyfont{cjkhwxk}{华文行楷}
%\setCJKfamilyfont{cjkfzcs}{方正粗宋简体}
%\newcommand*{\cjkfzcs}{\CJKfamily{cjkfzcs}}
\newcommand*{\cjkhwxk}{\CJKfamily{cjkhwxk}}
%\newfontfamily\wryh{Microsoft YaHei}
%\newfontfamily\hwzs{华文中宋}
%\newfontfamily\hwst{华文宋体}
%\newfontfamily\hwfs{华文仿宋}
%\newfontfamily\jljt{方正静蕾简体}
%\newfontfamily\hwxk{华文行楷}
\newcommand{\verylarge}{\fontsize{60pt}{\baselineskip}\selectfont}  
\newcommand{\chuhao}{\fontsize{44.9pt}{\baselineskip}\selectfont}  
\newcommand{\xiaochu}{\fontsize{38.5pt}{\baselineskip}\selectfont}  
\newcommand{\yihao}{\fontsize{27.8pt}{\baselineskip}\selectfont}  
\newcommand{\xiaoyi}{\fontsize{25.7pt}{\baselineskip}\selectfont}  
\newcommand{\erhao}{\fontsize{23.5pt}{\baselineskip}\selectfont}  
\newcommand{\xiaoerhao}{\fontsize{19.3pt}{\baselineskip}\selectfont} 
\newcommand{\sihao}{\fontsize{14pt}{\baselineskip}\selectfont}      % 字号设置  
\newcommand{\xiaosihao}{\fontsize{12pt}{\baselineskip}\selectfont}  % 字号设置  
\newcommand{\wuhao}{\fontsize{10.5pt}{\baselineskip}\selectfont}    % 字号设置  
\newcommand{\xiaowuhao}{\fontsize{9pt}{\baselineskip}\selectfont}   % 字号设置  
\newcommand{\liuhao}{\fontsize{7.875pt}{\baselineskip}\selectfont}  % 字号设置  
\newcommand{\qihao}{\fontsize{5.25pt}{\baselineskip}\selectfont}    % 字号设置 

\usepackage{diagbox}
\usepackage{multirow}
\boldmath
\XeTeXlinebreaklocale "zh"
\XeTeXlinebreakskip = 0pt plus 1pt minus 0.1pt
\definecolor{cred}{rgb}{0.8,0.8,0.8}
\definecolor{cgreen}{rgb}{0,0.3,0}
\definecolor{cpurple}{rgb}{0.5,0,0.35}
\definecolor{cdocblue}{rgb}{0,0,0.3}
\definecolor{cdark}{rgb}{0.95,1.0,1.0}
\lstset{
	language=bash,
	numbers=left,
	numberstyle=\tiny\color{black},
	showspaces=false,
	showstringspaces=false,
	basicstyle=\scriptsize,
	keywordstyle=\color{purple},
	commentstyle=\itshape\color{cgreen},
	stringstyle=\color{blue},
	frame=lines,
	% escapeinside=``,
	extendedchars=true, 
	xleftmargin=1em,
	xrightmargin=1em, 
	backgroundcolor=\color{cred},
	aboveskip=1em,
	breaklines=true,
	tabsize=4
} 

%\newfontfamily{\consolas}{Consolas}
%\newfontfamily{\monaco}{Monaco}
%\setmonofont[Mapping={}]{Consolas}	%英文引号之类的正常显示,相当于设置英文字体
%\setsansfont{Consolas} %设置英文字体 Monaco, Consolas,  Fantasque Sans Mono
%\setmainfont{Times New Roman}
%\setCJKmainfont{STZHONGS.TTF}
%\setmonofont{Consolas}
% \newfontfamily{\consolas}{YaHeiConsolas.ttf}
\newfontfamily{\monaco}{MONACO.TTF}
\setCJKmainfont{STZHONGS.TTF}
%\setmainfont{MONACO.TTF}
%\setsansfont{MONACO.TTF}

\newcommand{\fic}[1]{\begin{figure}[H]
		\center
		\includegraphics[width=0.8\textwidth]{#1}
	\end{figure}}
	
\newcommand{\sizedfic}[2]{\begin{figure}[H]
		\center
		\includegraphics[width=#1\textwidth]{#2}
	\end{figure}}

\newcommand{\codefile}[1]{\lstinputlisting{#1}}

\newcommand{\interval}{\vspace{0.5em}}

\newcommand{\tablestart}{
	\interval
	\begin{longtable}{p{2cm}p{10cm}}
	\hline}
\newcommand{\tableend}{
	\hline
	\end{longtable}
	\interval}

% 改变段间隔
\setlength{\parskip}{0.2em}
\linespread{1.1}

\usepackage{lastpage}
\usepackage{fancyhdr}
\pagestyle{fancy}
\lhead{\space \qquad \space}
\chead{liunx的公钥认证原理\qquad}
\rhead{\qquad\thepage/\pageref{LastPage}}

\begin{document}

\tableofcontents

\clearpage
\section{简述}
   ssh有密码登录和证书登录,初学者都喜欢用密码登录,甚至是root账户登录,密码是123456。但是在实际工作中,尤其是互联网公司,基本都是证书登录的。
内网的机器有可能是通过密码登录的,但在外网的机器,如果是密码登录,很容易受到攻击,真正的生产环境中,ssh登录都是证书登录。
\subsection{证书登录的步骤}
   1.客户端生成证书:私钥和公钥,然后私钥放在客户端,妥当保存,一般为了安全,访问有黑客拷贝客户端的私钥,客户端在生成私钥时,会设置一个密码,
以后每次登录ssh服务器时,客户端都要输入密码解开私钥(如果工作中,你使用了一个没有密码的私钥,有一天服务器被黑了,你是跳到黄河都洗不清)。

   2.服务器添加信用公钥:把客户端生成的公钥,上传到ssh服务器,添加到指定的文件中,这样,就完成ssh证书登录的配置了。
假设客户端想通过私钥要登录其他ssh服务器,同理,可以把公钥上传到其他ssh服务器。

    真实的工作中:员工生成好私钥和公钥(千万要记得设置私钥密码),然后把公钥发给运维人员,运维人员会登记你的公钥,为你开通一台或者多台服务器的权限,
然后员工就可以通过一个私钥,登录他有权限的服务器做系统维护等工作,所以,员工是有责任保护他的私钥的,如果被别人恶意拷贝,你又没有设置私钥密码,
那么,服务器就全完了,员工也可以放长假了。
\subsection{客户端建立私钥和公钥}
    在客户端终端运行命令
	\begin{lstlisting}
		ssh-keygen -t rsa
	\end{lstlisting}
	rsa是一种密码算法,还有一种是dsa,证书登录常用的是rsa。
	假设用户是blue,执行 ssh-keygen 时,才会在我的home目录底下的 .ssh/ 这个目录里面产生所需要的两把 Keys ,分别是私钥 (id_rsa) 与公钥 (id_rsa.pub)。
另外就是私钥的密码了,如果不是测试,不是要求无密码ssh,那么对于passphrase,不能输入空(直接回车),要妥当想一个有特殊字符的密码。
\subsection{客户端建立私钥和公钥}
	ssh服务器配置如下:
	\begin{lstlisting}
		vim /etc/ssh/sshd_config
		#禁用root账户登录,非必要,但为了安全性,请配置
		PermitRootLogin no

		# 是否让 sshd 去检查用户家目录或相关档案的权限数据,
		# 这是为了担心使用者将某些重要档案的权限设错,可能会导致一些问题所致。
		# 例如使用者的 ~.ssh/ 权限设错时,某些特殊情况下会不许用户登入
		StrictModes no

		# 是否允许用户自行使用成对的密钥系统进行登入行为,仅针对 version 2。
		# 至于自制的公钥数据就放置于用户家目录下的 .ssh/authorized_keys 内
		RSAAuthentication yes
		PubkeyAuthentication yes
		AuthorizedKeysFile      %h/.ssh/authorized_keys

		#有了证书登录了,就禁用密码登录吧,安全要紧
		PasswordAuthentication no
	\end{lstlisting}
	配置好ssh服务器的配置了,那么我们就要把客户端的公钥上传到服务器端,然后把客户端的公钥添加到authorized_keys
	在客户端执行命令
	\begin{lstlisting}
		scp ~/.ssh/id_rsa.pub blue@<ssh_server_ip>:~
	\end{lstlisting}
	在服务端执行命令
	\begin{lstlisting}
		cat  id_rsa.pub >> ~/.ssh/authorized_keys
	\end{lstlisting}
	如果有修改配置/etc/ssh/sshd_config,需要重启ssh服务器
	\begin{lstlisting}
		/etc/init.d/ssh restart
	\end{lstlisting}
\subsection{客户端通过私钥登录ssh服务器}
	ssh命令
	\begin{lstlisting}
		cat  id_rsa.pub >> ~/.ssh/authorized_keys
	\end{lstlisting}
	scp命令,基于ssh的
	\begin{lstlisting}
		scp -i /blue/.ssh/id_rsa filename blue@<ssh_server_ip>:/blue
	\end{lstlisting}
	每次敲命令,都要指定私钥,是一个很繁琐的事情,所以我们可以把私钥的路径加入ssh客户端的默认配置里
	修改/etc/ssh/ssh_config
	\begin{lstlisting}
		#其实默认id_rsa就已经加入私钥的路径了,这里只是示例而已
		IdentityFile ~/.ssh/id_rsa
		#如果有其他的私钥,还要再加入其他私钥的路径
		IdentityFile ~/.ssh/blue_rsa
	\end{lstlisting}
\end{document}