% !TeX spellcheck = en_US
%% 字体:方正静蕾简体
%%		 方正粗宋
\documentclass[a4paper,left=1.5cm,right=1.5cm,11pt]{article}

\usepackage[utf8]{inputenc}
\usepackage{fontspec}
\usepackage{cite}
\usepackage{xeCJK}
\usepackage{indentfirst}
\usepackage{titlesec}
\usepackage{etoolbox}%
\makeatletter
\patchcmd{\ttlh@hang}{\parindent\z@}{\parindent\z@\leavevmode}{}{}%
\patchcmd{\ttlh@hang}{\noindent}{}{}{}%
\makeatother

\usepackage{longtable}
\usepackage{empheq}
\usepackage{graphicx}
\usepackage{float}
\usepackage{rotating}
\usepackage{subfigure}
\usepackage{tabu}
\usepackage{amsmath}
\usepackage{setspace}
\usepackage{amsfonts}
\usepackage{appendix}
\usepackage{listings}
\usepackage{xcolor}
\usepackage{geometry}
\setcounter{secnumdepth}{4}
%\titleformat*{\section}{\LARGE}
%\renewcommand\refname{参考文献}
%\titleformat{\chapter}{\centering\bfseries\huge}{}{0.7em}{}{}
\titleformat{\section}{\LARGE\bf}{\thesection}{1em}{}{}
\titleformat{\subsection}{\Large\bfseries}{\thesubsection}{1em}{}{}
\titleformat{\subsubsection}{\large\bfseries}{\thesubsubsection}{1em}{}{}
\renewcommand{\contentsname}{{ \centerline{目{  } 录}}}
\setCJKfamilyfont{cjkhwxk}{STXINGKA.TTF}
%\setCJKfamilyfont{cjkhwxk}{华文行楷}
%\setCJKfamilyfont{cjkfzcs}{方正粗宋简体}
%\newcommand*{\cjkfzcs}{\CJKfamily{cjkfzcs}}
\newcommand*{\cjkhwxk}{\CJKfamily{cjkhwxk}}
%\newfontfamily\wryh{Microsoft YaHei}
%\newfontfamily\hwzs{华文中宋}
%\newfontfamily\hwst{华文宋体}
%\newfontfamily\hwfs{华文仿宋}
%\newfontfamily\jljt{方正静蕾简体}
%\newfontfamily\hwxk{华文行楷}
\newcommand{\verylarge}{\fontsize{60pt}{\baselineskip}\selectfont}  
\newcommand{\chuhao}{\fontsize{44.9pt}{\baselineskip}\selectfont}  
\newcommand{\xiaochu}{\fontsize{38.5pt}{\baselineskip}\selectfont}  
\newcommand{\yihao}{\fontsize{27.8pt}{\baselineskip}\selectfont}  
\newcommand{\xiaoyi}{\fontsize{25.7pt}{\baselineskip}\selectfont}  
\newcommand{\erhao}{\fontsize{23.5pt}{\baselineskip}\selectfont}  
\newcommand{\xiaoerhao}{\fontsize{19.3pt}{\baselineskip}\selectfont} 
\newcommand{\sihao}{\fontsize{14pt}{\baselineskip}\selectfont}      % 字号设置  
\newcommand{\xiaosihao}{\fontsize{12pt}{\baselineskip}\selectfont}  % 字号设置  
\newcommand{\wuhao}{\fontsize{10.5pt}{\baselineskip}\selectfont}    % 字号设置  
\newcommand{\xiaowuhao}{\fontsize{9pt}{\baselineskip}\selectfont}   % 字号设置  
\newcommand{\liuhao}{\fontsize{7.875pt}{\baselineskip}\selectfont}  % 字号设置  
\newcommand{\qihao}{\fontsize{5.25pt}{\baselineskip}\selectfont}    % 字号设置 

\usepackage{diagbox}
\usepackage{multirow}
\boldmath
\XeTeXlinebreaklocale "zh"
\XeTeXlinebreakskip = 0pt plus 1pt minus 0.1pt
\definecolor{cred}{rgb}{0.8,0.8,0.8}
\definecolor{cgreen}{rgb}{0,0.3,0}
\definecolor{cpurple}{rgb}{0.5,0,0.35}
\definecolor{cdocblue}{rgb}{0,0,0.3}
\definecolor{cdark}{rgb}{0.95,1.0,1.0}
\lstset{
	language=bash,
	numbers=left,
	numberstyle=\tiny\color{black},
	showspaces=false,
	showstringspaces=false,
	basicstyle=\scriptsize,
	keywordstyle=\color{purple},
	commentstyle=\itshape\color{cgreen},
	stringstyle=\color{blue},
	frame=lines,
	% escapeinside=``,
	extendedchars=true, 
	xleftmargin=1em,
	xrightmargin=1em, 
	backgroundcolor=\color{cred},
	aboveskip=1em,
	breaklines=true,
	tabsize=4
} 

%\newfontfamily{\consolas}{Consolas}
%\newfontfamily{\monaco}{Monaco}
%\setmonofont[Mapping={}]{Consolas}	%英文引号之类的正常显示,相当于设置英文字体
%\setsansfont{Consolas} %设置英文字体 Monaco, Consolas,  Fantasque Sans Mono
%\setmainfont{Times New Roman}
%\setCJKmainfont{STZHONGS.TTF}
%\setmonofont{Consolas}
% \newfontfamily{\consolas}{YaHeiConsolas.ttf}
\newfontfamily{\monaco}{MONACO.TTF}
\setCJKmainfont{STZHONGS.TTF}
%\setmainfont{MONACO.TTF}
%\setsansfont{MONACO.TTF}

\newcommand{\fic}[1]{\begin{figure}[H]
		\center
		\includegraphics[width=0.8\textwidth]{#1}
	\end{figure}}
	
\newcommand{\sizedfic}[2]{\begin{figure}[H]
		\center
		\includegraphics[width=#1\textwidth]{#2}
	\end{figure}}

\newcommand{\codefile}[1]{\lstinputlisting{#1}}

\newcommand{\interval}{\vspace{0.5em}}

\newcommand{\tablestart}{
	\interval
	\begin{longtable}{p{2cm}p{10cm}}
	\hline}
\newcommand{\tableend}{
	\hline
	\end{longtable}
	\interval}

% 改变段间隔
\setlength{\parskip}{0.2em}
\linespread{1.1}

\usepackage{lastpage}
\usepackage{fancyhdr}
\pagestyle{fancy}
\lhead{\space \qquad \space}
\chead{screen窗口的使用\qquad}
\rhead{\qquad\thepage/\pageref{LastPage}}

\begin{document}

\tableofcontents

\clearpage

\subsection{问题汇总}
1.--------------------------------------------------------------------------------------------------------------------------------------------------------------------------
	来源:/var/log/neutron/neutron-server.log
	操作:neutron.plugins.ml2.plugin----> Attempt 10 to bind port ***
	错误提示:neutron.plugins.ml2.managers ---->Failed to bind port *** on host controller for vnic_type normal using segments
	[{'segmentation_id': None, 'physical_network': u'external', 'id': u'dbba80f8-b1e9-4116-a498-866a325aea81', 'network_type': u'flat'}]
	警告:neutron.plugins.ml2.rpc----> Device *** requested by agent ovs-agent-controller on network 89f4a974-c745-4263-87e5-7f268253c80e not bound, vif_type: binding_failed

	解决方法:在启动neutron-server服务的时候,配置文件加上ml2_conf.ini
2.--------------------------------------------------------------------------------------------------------------------------------------------------------------------------
	问题二:keyerror:fips
	解决办法:vi /usr/lib/python2.7/dist-packages/nova/api/openstack/compute/servers.py
	修改floating_ips的语句如下:
	floating_ips = body['server'].get('fips', None)

3.--------------------------------------------------------------------------------------------------------------------------------------------------------------------------
	问题三:l3_agent服务启动之后,日志报错,需要l3_agent.ini的[DEFAULT]verbose的参数,但配置文件中并没有这个选项,手动添加也不行
	解决办法:vi venv/lib/python2.7/site-packages/neutron/agent/common/config.py  
	修改verbose的条件语句如下:
	if getattr(conf, "verbose", False):
        cmd_args.append('--verbose')
4.--------------------------------------------------------------------------------------------------------------------------------------------------------------------------
	问题四:l3_agent服务启动之后,日志报错,unauthrized command ip netns exec qrouter/qdhcp/snat arping/neutron-ns-matadata-proxy
	解决办法:apt-get install arping ,ln -s /root/venv/bin/neutron-ns-metadata-proxy /usr/bin/
	遇到同样问题的额外安装包:额外的安装:conntrack,dnsmasq-utils,ipset
5.--------------------------------------------------------------------------------------------------------------------------------------------------------------------------
	修改:控制节点和计算节点,调整nova.conf和neutron.conf的keystone部分。
	将l3_agent.ini的interface_driver的值修改为openvswitch
6.--------------------------------------------------------------------------------------------------------------------------------------------------------------------------
	问题五:虚拟机获取不到ip地址
	原因:无意中发现在宿主机ping内网网关192.168.0.1,能够ping通。使用traceroute进行跟踪,先到外网的网关172.16.19.254,再到外外网的网关,之后找到这个ip。怀疑
	是内网网段特殊影响。
	解决办法:将内网换成10.0.0.0/24网段,即可。该问题有待追查
7.--------------------------------------------------------------------------------------------------------------------------------------------------------------------------
	问题六:添加dvr功能之后,虚拟机获取不到ip
	原因:暂时还不能给出很好的解释
	解决办法:将ml2_conf.ini的mechanism_drivers的l2population选项删除。将openvswitch_agent.ini的l2_population = True改为False。
8.--------------------------------------------------------------------------------------------------------------------------------------------------------------------------
	问题七:通过撤销dvr的配置和删除路由等操作仍旧无法重新恢复之前的功能。
	原因:openvswitch版本为2.0.2,版本过低,需要升级。否则无法使用
	
	功能
	解决办法:手动下载源码安装至少2.1版本以上的。
	
\end{document}

	
	
