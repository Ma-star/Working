% !TeX spellcheck = en_US
%% 字体:方正静蕾简体
%%		 方正粗宋
\documentclass[a4paper,left=1.5cm,right=1.5cm,11pt]{article}

\usepackage[utf8]{inputenc}
\usepackage{fontspec}
\usepackage{cite}
\usepackage{xeCJK}
\usepackage{indentfirst}
\usepackage{titlesec}
\usepackage{etoolbox}%
\makeatletter
\patchcmd{\ttlh@hang}{\parindent\z@}{\parindent\z@\leavevmode}{}{}%
\patchcmd{\ttlh@hang}{\noindent}{}{}{}%
\makeatother

\usepackage{longtable}
\usepackage{empheq}
\usepackage{graphicx}
\usepackage{float}
\usepackage{rotating}
\usepackage{subfigure}
\usepackage{tabu}
\usepackage{amsmath}
\usepackage{setspace}
\usepackage{amsfonts}
\usepackage{appendix}
\usepackage{listings}
\usepackage{xcolor}
\usepackage{geometry}
\setcounter{secnumdepth}{4}
%\titleformat*{\section}{\LARGE}
%\renewcommand\refname{参考文献}
%\titleformat{\chapter}{\centering\bfseries\huge}{}{0.7em}{}{}
\titleformat{\section}{\LARGE\bf}{\thesection}{1em}{}{}
\titleformat{\subsection}{\Large\bfseries}{\thesubsection}{1em}{}{}
\titleformat{\subsubsection}{\large\bfseries}{\thesubsubsection}{1em}{}{}
\renewcommand{\contentsname}{{ \centerline{目{  } 录}}}
\setCJKfamilyfont{cjkhwxk}{STXINGKA.TTF}
%\setCJKfamilyfont{cjkhwxk}{华文行楷}
%\setCJKfamilyfont{cjkfzcs}{方正粗宋简体}
%\newcommand*{\cjkfzcs}{\CJKfamily{cjkfzcs}}
\newcommand*{\cjkhwxk}{\CJKfamily{cjkhwxk}}
%\newfontfamily\wryh{Microsoft YaHei}
%\newfontfamily\hwzs{华文中宋}
%\newfontfamily\hwst{华文宋体}
%\newfontfamily\hwfs{华文仿宋}
%\newfontfamily\jljt{方正静蕾简体}
%\newfontfamily\hwxk{华文行楷}
\newcommand{\verylarge}{\fontsize{60pt}{\baselineskip}\selectfont}  
\newcommand{\chuhao}{\fontsize{44.9pt}{\baselineskip}\selectfont}  
\newcommand{\xiaochu}{\fontsize{38.5pt}{\baselineskip}\selectfont}  
\newcommand{\yihao}{\fontsize{27.8pt}{\baselineskip}\selectfont}  
\newcommand{\xiaoyi}{\fontsize{25.7pt}{\baselineskip}\selectfont}  
\newcommand{\erhao}{\fontsize{23.5pt}{\baselineskip}\selectfont}  
\newcommand{\xiaoerhao}{\fontsize{19.3pt}{\baselineskip}\selectfont} 
\newcommand{\sihao}{\fontsize{14pt}{\baselineskip}\selectfont}      % 字号设置  
\newcommand{\xiaosihao}{\fontsize{12pt}{\baselineskip}\selectfont}  % 字号设置  
\newcommand{\wuhao}{\fontsize{10.5pt}{\baselineskip}\selectfont}    % 字号设置  
\newcommand{\xiaowuhao}{\fontsize{9pt}{\baselineskip}\selectfont}   % 字号设置  
\newcommand{\liuhao}{\fontsize{7.875pt}{\baselineskip}\selectfont}  % 字号设置  
\newcommand{\qihao}{\fontsize{5.25pt}{\baselineskip}\selectfont}    % 字号设置 

\usepackage{diagbox}
\usepackage{multirow}
\boldmath
\XeTeXlinebreaklocale "zh"
\XeTeXlinebreakskip = 0pt plus 1pt minus 0.1pt
\definecolor{cred}{rgb}{0.8,0.8,0.8}
\definecolor{cgreen}{rgb}{0,0.3,0}
\definecolor{cpurple}{rgb}{0.5,0,0.35}
\definecolor{cdocblue}{rgb}{0,0,0.3}
\definecolor{cdark}{rgb}{0.95,1.0,1.0}
\lstset{
	language=bash,
	numbers=left,
	numberstyle=\tiny\color{black},
	showspaces=false,
	showstringspaces=false,
	basicstyle=\scriptsize,
	keywordstyle=\color{purple},
	commentstyle=\itshape\color{cgreen},
	stringstyle=\color{blue},
	frame=lines,
	% escapeinside=``,
	extendedchars=true, 
	xleftmargin=1em,
	xrightmargin=1em, 
	backgroundcolor=\color{cred},
	aboveskip=1em,
	breaklines=true,
	tabsize=4
} 

%\newfontfamily{\consolas}{Consolas}
%\newfontfamily{\monaco}{Monaco}
%\setmonofont[Mapping={}]{Consolas}	%英文引号之类的正常显示,相当于设置英文字体
%\setsansfont{Consolas} %设置英文字体 Monaco, Consolas,  Fantasque Sans Mono
%\setmainfont{Times New Roman}
%\setCJKmainfont{STZHONGS.TTF}
%\setmonofont{Consolas}
% \newfontfamily{\consolas}{YaHeiConsolas.ttf}
\newfontfamily{\monaco}{MONACO.TTF}
\setCJKmainfont{STZHONGS.TTF}
%\setmainfont{MONACO.TTF}
%\setsansfont{MONACO.TTF}

\newcommand{\fic}[1]{\begin{figure}[H]
		\center
		\includegraphics[width=0.8\textwidth]{#1}
	\end{figure}}
	
\newcommand{\sizedfic}[2]{\begin{figure}[H]
		\center
		\includegraphics[width=#1\textwidth]{#2}
	\end{figure}}

\newcommand{\codefile}[1]{\lstinputlisting{#1}}

\newcommand{\interval}{\vspace{0.5em}}

\newcommand{\tablestart}{
	\interval
	\begin{longtable}{p{2cm}p{10cm}}
	\hline}
\newcommand{\tableend}{
	\hline
	\end{longtable}
	\interval}

% 改变段间隔
\setlength{\parskip}{0.2em}
\linespread{1.1}

\usepackage{lastpage}
\usepackage{fancyhdr}
\pagestyle{fancy}
\lhead{\space \qquad \space}
\chead{openvswitch2.4.0源码安装\qquad}
\rhead{\qquad\thepage/\pageref{LastPage}}

\begin{document}

\tableofcontents

\clearpage

\subsection{}
	\begin{itemize}
        \item[1.]下载openvswitch源码 http://openvswitch.org/download/ ,选择的是openvswitch2.4.0
		\item[2.]解压安装包:tar -xzf openvswitch-2.4.0.tar.gz
		\item[3.]构建基于Linux内核的交换机,uname -r用来得到自己linux内核版本号,用反引号(键盘上数字1左边)括起来,其中aptitude和apt-get一样,是Debian及其衍生系统中功能极其强大的包管理工具,但是aptitede在处理依赖问题上更佳一些,在删除一个包时,会删除本身所依赖的包,使系统更为干净。
                 ./configure部分可以用--prefix=参数,可以让OVS完全安装在该目录底下。
				 命令如下:
				 \begin{lstlisting}
					cd openvswitch-2.4.0
					aptitude install dh-autoreconf libssl-dev openssl                    #预先安装一些库
					./configure --with-linux=/lib/modules/对应linux内核(可以用uname -r命令获得)/build
				 \end{lstlisting}
		\item[4.]编译并安装OVS2.4.0
		         命令如下:
				 \begin{lstlisting}
					make
					make install
				 \end{lstlisting}
		\item[5.]安装并加载构建的内核模块
				 命令如下:
				 \begin{lstlisting}
					modprobe gre
					insmod datapath/linux/openvswitch.ko(可能会出现错误,insmod: ERROR: could not insert module openvswitch.ko: Unknown symbol in module)
					如果有错误,找到网上的一个解决方法如下
					-----------------------------------------
					rmmod openvswitch
					modprobe libcrc32c
					modprobe nf_conntrack_ipv6
					modprobe nf_nat_ipv6
					modprobe gre
					insmod ./datapath/linux/openvswitch.ko
					insmod ./datapath/linux/vport-geneve.ko
					----------------------------------------
					make modules_install
					modprobe openvswitch
				 \end{lstlisting}
				 此时可以通过lsmod |grep openvswitch来查看已载入系统的模块,发现有OpenvSwitch
		\item[6.]使用ovsdb工具初始化配置数据库
				命令如下:
				 \begin{lstlisting}
					ovsdb-tool create /usr/local/etc/openvswitch/conf.db /usr/local/share/openvswitch/vswitch.ovsschema
				 \end{lstlisting}
		\item[7.]开启ovsdb-server配置数据库
				命令如下:
				 \begin{lstlisting}
					ovsdb-server --remote=punix:/usr/local/var/run/openvswitch/db.sock \
					--remote=db:Open_vSwitch,Open_vSwitch,manager_options \
					--private-key=db:Open_vSwitch,SSL,private_key \
					--certificate=db:Open_vSwitch,SSL,certificate \
					--bootstrap-ca-cert=db:Open_vSwitch,SSL,ca_cert \
					--pidfile --detach --log-file
				 \end{lstlisting}
		\item[8.]开启ovs-vsctl
				命令如下:
				 \begin{lstlisting}
					ovs-vsctl --no-wait init
				 \end{lstlisting}
		\item[9.]开启ovs-switchd功能,即主进程
				命令如下:
				 \begin{lstlisting}
					ovs-vswitchd --pidfile --detach
				 \end{lstlisting}
		\item[10.]开机自动加载内核模块
				命令如下:
				 \begin{lstlisting}
					echo "openvswitch " >> /etc/modules
					echo "gre" >> /etc/modules
					echo "libcrc32c" >> /etc/modules
				 \end{lstlisting}
		\item[11.]开机自动启动
				首先新建一个文件
				命令如下:
				 \begin{lstlisting}
					vim /etc/init.d/openvswitch
				 \end{lstlisting}
				文件内容如下:
				#!/bin/sh
				start-stop-daemon -q -S -x /usr/local/sbin/ovsdb-server -- --remote=punix:/usr/local/var/run/openvswitch/db.sock --remote=db:Open_vSwitch,Open_vSwitch,manager_options --pidfile --detach --log-file
				sleep 3 # waiting ovsdb-server 
				start-stop-daemon -q -S -x /usr/local/bin/ovs-vsctl -- --no-wait init
				start-stop-daemon -q -S -x /usr/local/sbin/ovs-vswitchd -- --pidfile --detach --log-file
				接着改变文件权限,执行开机启动命令
				chmod +x /etc/init.d/openvswitch
				updata-rc.d -f openvswitch defaults
	\end{itemize}
\end{document}