% !TeX spellcheck = en_US
%% 字体:方正静蕾简体
%%		 方正粗宋
\documentclass[a4paper,left=1.5cm,right=1.5cm,11pt]{article} 
\title{Openstack之neutron服务}
\author{周威光整理}
\date{2017-06-22} 
\usepackage[utf8]{inputenc}
\usepackage{fontspec}
\usepackage{cite}
\usepackage{xeCJK}
\usepackage{indentfirst}
\usepackage{titlesec}
\usepackage{etoolbox}%
\makeatletter
\patchcmd{\ttlh@hang}{\parindent\z@}{\parindent\z@\leavevmode}{}{}%
\patchcmd{\ttlh@hang}{\noindent}{}{}{}%
\makeatother

\usepackage{longtable}
\usepackage{empheq}
\usepackage{graphicx}
\usepackage{float}
\usepackage{rotating}
\usepackage{subfigure}
\usepackage{tabu}
\usepackage{amsmath}
\usepackage{setspace}
\usepackage{amsfonts}
\usepackage{appendix}
\usepackage{listings}
\usepackage{xcolor}
\usepackage{geometry}
\setcounter{secnumdepth}{4}
%\titleformat*{\section}{\LARGE}
%\renewcommand\refname{参考文献}
%\titleformat{\chapter}{\centering\bfseries\huge}{}{0.7em}{}{}
\titleformat{\section}{\LARGE\bf}{\thesection}{1em}{}{}
\titleformat{\subsection}{\Large\bfseries}{\thesubsection}{1em}{}{}
\titleformat{\subsubsection}{\large\bfseries}{\thesubsubsection}{1em}{}{}
\renewcommand{\contentsname}{{ \centerline{目{  } 录}}}
\setCJKfamilyfont{cjkhwxk}{STXINGKA.TTF}
%\setCJKfamilyfont{cjkhwxk}{华文行楷}
%\setCJKfamilyfont{cjkfzcs}{方正粗宋简体}
%\newcommand*{\cjkfzcs}{\CJKfamily{cjkfzcs}}
\newcommand*{\cjkhwxk}{\CJKfamily{cjkhwxk}}
%\newfontfamily\wryh{Microsoft YaHei}
%\newfontfamily\hwzs{华文中宋}
%\newfontfamily\hwst{华文宋体}
%\newfontfamily\hwfs{华文仿宋}
%\newfontfamily\jljt{方正静蕾简体}
%\newfontfamily\hwxk{华文行楷}
\newcommand{\verylarge}{\fontsize{60pt}{\baselineskip}\selectfont}  
\newcommand{\chuhao}{\fontsize{44.9pt}{\baselineskip}\selectfont}  
\newcommand{\xiaochu}{\fontsize{38.5pt}{\baselineskip}\selectfont}  
\newcommand{\yihao}{\fontsize{27.8pt}{\baselineskip}\selectfont}  
\newcommand{\xiaoyi}{\fontsize{25.7pt}{\baselineskip}\selectfont}  
\newcommand{\erhao}{\fontsize{23.5pt}{\baselineskip}\selectfont}  
\newcommand{\xiaoerhao}{\fontsize{19.3pt}{\baselineskip}\selectfont} 
\newcommand{\sihao}{\fontsize{14pt}{\baselineskip}\selectfont}      % 字号设置  
\newcommand{\xiaosihao}{\fontsize{12pt}{\baselineskip}\selectfont}  % 字号设置  
\newcommand{\wuhao}{\fontsize{10.5pt}{\baselineskip}\selectfont}    % 字号设置  
\newcommand{\xiaowuhao}{\fontsize{9pt}{\baselineskip}\selectfont}   % 字号设置  
\newcommand{\liuhao}{\fontsize{7.875pt}{\baselineskip}\selectfont}  % 字号设置  
\newcommand{\qihao}{\fontsize{5.25pt}{\baselineskip}\selectfont}    % 字号设置 

\usepackage{diagbox}
\usepackage{multirow}
\boldmath
\XeTeXlinebreaklocale "zh"
\XeTeXlinebreakskip = 0pt plus 1pt minus 0.1pt
\definecolor{cred}{rgb}{0.8,0.8,0.8}
\definecolor{cgreen}{rgb}{0,0.3,0}
\definecolor{cpurple}{rgb}{0.5,0,0.35}
\definecolor{cdocblue}{rgb}{0,0,0.3}
\definecolor{cdark}{rgb}{0.95,1.0,1.0}
\lstset{
	language=bash,
	numbers=left,
	numberstyle=\tiny\color{black},
	showspaces=false,
	showstringspaces=false,
	basicstyle=\scriptsize,
	keywordstyle=\color{purple},
	commentstyle=\itshape\color{cgreen},
	stringstyle=\color{blue},
	frame=lines,
	% escapeinside=``,
	extendedchars=true, 
	xleftmargin=1em,
	xrightmargin=1em, 
	backgroundcolor=\color{cred},
	aboveskip=1em,
	breaklines=true,
	tabsize=4
} 

%\newfontfamily{\consolas}{Consolas}
%\newfontfamily{\monaco}{Monaco}
%\setmonofont[Mapping={}]{Consolas}	%英文引号之类的正常显示,相当于设置英文字体
%\setsansfont{Consolas} %设置英文字体 Monaco, Consolas,  Fantasque Sans Mono
%\setmainfont{Times New Roman}
%\setCJKmainfont{STZHONGS.TTF}
%\setmonofont{Consolas}
% \newfontfamily{\consolas}{YaHeiConsolas.ttf}
\newfontfamily{\monaco}{MONACO.TTF}
\setCJKmainfont{STZHONGS.TTF}
%\setmainfont{MONACO.TTF}
%\setsansfont{MONACO.TTF}

\newcommand{\fic}[1]{\begin{figure}[H]
		\center
		\includegraphics[width=0.8\textwidth]{#1}
	\end{figure}}
	
\newcommand{\sizedfic}[2]{\begin{figure}[H]
		\center
		\includegraphics[width=#1\textwidth]{#2}
	\end{figure}}

\newcommand{\codefile}[1]{\lstinputlisting{#1}}

\newcommand{\interval}{\vspace{0.5em}}

\newcommand{\tablestart}{
	\interval
	\begin{longtable}{p{2cm}p{10cm}}
	\hline}
\newcommand{\tableend}{
	\hline
	\end{longtable}
	\interval}

% 改变段间隔
\setlength{\parskip}{0.2em}
\linespread{1.1}

\usepackage{lastpage}
\usepackage{fancyhdr}
\pagestyle{fancy}
\lhead{\space \qquad \space}
\chead{Openstack之neutron服务 \qquad}
\rhead{\qquad\thepage/\pageref{LastPage}}

\begin{document}

\tableofcontents

\clearpage

\maketitle
\section{neutron简介}
neutron是虚拟化网络的一种实现方式,为什么要网络虚拟化,主要有两个方面的需求,一是互联网行业数据中心的基本特征就是服务器的规模偏大。进入云计算时代后,
其业务特征变得更加复杂,包括:虚拟化支持、多业务承载、资源灵活调度等。与此同时,互联网云计算的规模不但没有缩减,
反而更加庞大。这就给云计算的网络带来了巨大的压力。二是数据中心(Data Center)中的物理网络是固定的、需要手工配置的、单一的、没有多租户隔离的网络。
而云架构往往是多租户架构,这意味着多个客户会共享单一的物理网络。因此,除了提供基本的网络连接能力以外,云还需要提供网络在租户之间的隔离能力;
同时云是自服务的,这意味着租户可以通过云提供的 API 来使用虚拟出的网络组建来设计,构建和部署各种他们需要的网络。OpenStack云也不例外,其通过Neutron项目在物理网络环境之上提供满足多租户要求的虚拟网络和服务。
\section{neutron功能介绍}
Neutron作为虚拟化网络的一种实现方式,提供的网络虚拟化能力主要包括如下:
\begin{itemize}
	\item[1.]二层到七层网络的虚拟化
	\item[2.]租户隔离性
	\item[3.]网络安全性
	\item[4.]网络高可用性和扩展性
	\item[5.]更高级的服务,包括 LBaaS,FWaaS,VPNaaS 等
\end{itemize}
\subsection{二层到七层网络的虚拟化}
\subsubsection{二层虚拟化}
二层提供network,subnet,port资源,默认采用开源的Open vSwitch作为其虚机交换机,同时还支持使用Linux bridge
\begin{itemize}
	\item[1.]网络(network)是一个隔离的二层网段,类似于物理网络世界中的虚拟 LAN (VLAN)。更具体来讲,它是为创建它的租户而保留的一个广播域,或者被显式配置为共享网段。端口和子网始终被分配给某个特定的网络。
	\\\\
	根据创建网络的用户的权限,Neutron network 可以分为:
	\begin{itemize}
		\item[(1).]Provider network:管理员创建的和物理网络有直接映射关系的虚拟网络;
		\item[(2).]Tenant network:租户普通用户创建的网络,物理网络对创建者透明,其配置由 Neutron根据管理员在系统中的配置决定;
	\end{itemize}
	根据网络的类型,Neutron network 可以分为:
	\begin{itemize}
		\item[(1).]local network(本地网络):一个只允许在本服务器内通信的虚拟网络,不知道跨服务器的通信。主要用于单节点上测试。
		\item[(2).]Flat network:基于不使用 VLAN 的物理网络实现的虚拟网络。每个物理网络最多只能实现一个虚拟网络。
		\item[(3).]VLAN network(虚拟局域网) :基于物理 VLAN 网络实现的虚拟网络。共享同一个物理网络的多个 VLAN 网络是相互隔离的,甚至可以使用重叠的 IP 地址空间。
		每个支持 VLAN network 的物理网络可以被视为一个分离的 VLAN trunk,它使用一组独占的 VLAN ID。有效的 VLAN ID 范围是 1 到 4094。
		\item[(4).]GRE network (通用路由封装网络):一个使用 GRE 封装网络包的虚拟网络。GRE 封装的数据包基于 IP 路由表来进行路由,因此 GRE network 不和具体的物理网络绑定。
		\item[(5).]VXLAN network(虚拟可扩展网络):基于 VXLAN 实现的虚拟网络。同 GRE network 一样, VXLAN network 中 IP 包的路由也基于 IP 路由表,也不和具体的物理网络绑定。
	\end{itemize}
	\item[2.]子网(subnet)是一组 IPv4 或 IPv6 地址以及与其有关联的配置。它是一个地址池,OpenStack 可从中向虚拟机 (VM) 分配 IP 地址。每个子网指定为一个无类别域间路由 (Classless Inter-Domain Routing) 范围,必须与一个网络相关联。
	除了子网之外,租户还可以指定一个网关、一个域名系统 (DNS) 名称服务器列表,以及一组主机路由。这个子网上的 VM 实例随后会自动继承该配置。
	\item[3.]端口(Port)代表虚拟网络交换机(logical network switch)上的虚机交换端口(virtual switch port)。虚机的网卡(VIF - Virtual Interface)会被连接到 port 上。
	当虚机的 VIF 连接到 Port 后,这个 vNIC 就会拥有 MAC 地址和 IP 地址。Port 的 IP 地址是从 subnet 中分配的。
\end{itemize}
\subsubsection{三层虚拟化}
三层虚拟化通过一个Virtual router提供不同网段之间的IP包路由功能,由Nuetron L3 agent负责管理
\subsubsection{四层到七层虚拟化:}
neutron还提供提供负载均衡,VPN,防火墙等四层到七层的虚拟化
\begin{itemize}
	\item[1.]Neutron LBaas(load-balancer-as-a-service)扩展(extension)提供向在多个Nova虚机中运行的应用提供负载均衡的方法。
	它还提供 API 来快速方便地部署负载均衡器。Neutron默认以HAProxy为负载均衡的driver,同时也支持 A10 network、netscaler、radware等作为 driver。
	\item[2.]Neutron项目的VPNaaS是一种site-to-site类型的VPN解决方案,通过向用户提供RESETS API,CLI和Horizon GUI去操作IPSec来实现。
	\item[3.]Neutron提供一种基于 Neutron L3 Agent的一种网络四层防火墙虚拟化参考实现 Firewall-as-a-service,简称FWaas。FWaaS在租户网络边缘实现的虚拟路由器上通过创建防火墙规则实实现。
\end{itemize}
\subsection{租户隔离性}
Neutron 实现了不同层次的租户网络隔离性
\begin{itemize}
	\item[1.]租户之间的网络是三层隔离的,连通过 VR 做路由都不行,实在要连通的话,需要走物理网络
	\item[2.]一个租户内的不同网络之间二层隔离的,需要通过 VR 做三层连通
	\item[3.]一个网络内的不同子网也是二层隔离的,需要通过 VR 做三层连通
\end{itemize}
\subsection{网络安全性}
\begin{itemize}
	\item[1.]Neutron 还提供数据网络与外部网络的隔离性。默认情况下,所有虚机通往外网的流量全部走网络节点上的 L3 agent。
	在这里,内部的固定 IP 被转化为外部的浮动 IP 地址。这种做法一方面保证了网络包能够回来,另一方面也隐藏了内部的 IP 地址。
	\item[2.]Neutron 还是用 Linux iptables 特性,实现其 Security Group 特性,从而保证访问虚机的安全性。
	\item[3.]Neutron利用网络控制节点上的 network namespace 中的 iptables,实现了进出租户网络的网络包防火墙,从而保证了进出租户网络的安全性。
\end{itemize}
\subsection{网络高可用性和扩展性}
OpenStack 云中可能用于成千上万台虚机,成千上万个租户,因此,Neutron 的数据网络的可用性和扩展性非常重要。Neutron 中,这些特性包括几个层次:
\begin{itemize}
	\item[1.]软件架构上,Neutron 实现OpenStack 标准的去中心化架构和插件机制,有效地保证了其扩展性。如下图所示
	\sizedfic{0.8}{neutron软件架构.jpg}
	\item[2.]支持分布式Virtual Router(DVR),默认情况下,L3 agent部署在网络节点上,这在大规模的云环境中可能会存在性能问题。
	通过使用 DVR,L3 转发和 NAT 会被分布在计算节点上,这使得计算节点变成了网络节点,这样集中式的网络节点的负载就被分担了。
	\item[3.]支持Virtual Router Redundancy Protocol(VRRP)机制,借助实现VRRP协议的软件,来保证 Neutron L3 Agent 的高可用性
	\item[4.]L2 Population 和 ARP Responder:这两个功能大大减少了网络的复杂性,提交了网络效率,从而促进了扩展性。
\end{itemize}
\subsection{提供高级服务}
在实际的网络中,除了网络的核心功能以外,还有一些普遍应用的网络服务,比如 VPN, Load Balancing 和 Firewall 
\section{neutron基本架构}
neutron架构包括两个部分:neutron各服务在机器节点上的分布概况,neutron各组件及其连接概况
\begin{itemize}
	\item[1.]neutron各服务在机器节点上的分布概况,如下图所示
	\sizedfic{0.8}{neutron_1.png}
	\item[2.]neutron各组件及其连接概况,如下图所示
	\sizedfic{0.8}{neutron网络拓扑.png}
\end{itemize}

\end{document}