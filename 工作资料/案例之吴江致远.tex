% !TeX spellcheck = en_US
%% 字体:方正静蕾简体
%%		 方正粗宋
\documentclass[a4paper,left=1.5cm,right=1.5cm,11pt]{article}
\title{案例之吴江致远}
\author{周威光整理\footnote{简介:恒天云FTE}}
\date{2017-07-12} 
% 引用宏包
\usepackage[utf8]{inputenc}
\usepackage{fontspec}
\usepackage{cite}
\usepackage{xeCJK}
\usepackage{indentfirst}
\usepackage{titlesec}
\usepackage{etoolbox}%
\makeatletter
\patchcmd{\ttlh@hang}{\parindent\z@}{\parindent\z@\leavevmode}{}{}%
\patchcmd{\ttlh@hang}{\noindent}{}{}{}%
\makeatother

\usepackage{longtable}
\usepackage{empheq}
\usepackage{graphicx}
\usepackage{float}
\usepackage{rotating}
\usepackage{subfigure}
\usepackage{tabu}
\usepackage{amsmath}
\usepackage{setspace}
\usepackage{amsfonts}
\usepackage{appendix}
\usepackage{listings}
\usepackage{xcolor}
\usepackage{geometry}
\setcounter{secnumdepth}{4}
%\titleformat*{\section}{\LARGE}
%\renewcommand\refname{参考文献}
%\titleformat{\chapter}{\centering\bfseries\huge}{}{0.7em}{}{}
\titleformat{\section}{\LARGE\bf}{\thesection}{1em}{}{}
\titleformat{\subsection}{\Large\bfseries}{\thesubsection}{1em}{}{}
\titleformat{\subsubsection}{\large\bfseries}{\thesubsubsection}{1em}{}{}
\renewcommand{\contentsname}{{ \centerline{目{  } 录}}}
\setCJKfamilyfont{cjkhwxk}{STXINGKA.TTF}
%\setCJKfamilyfont{cjkhwxk}{华文行楷}
%\setCJKfamilyfont{cjkfzcs}{方正粗宋简体}
%\newcommand*{\cjkfzcs}{\CJKfamily{cjkfzcs}}
\newcommand*{\cjkhwxk}{\CJKfamily{cjkhwxk}}
%\newfontfamily\wryh{Microsoft YaHei}
%\newfontfamily\hwzs{华文中宋}
%\newfontfamily\hwst{华文宋体}
%\newfontfamily\hwfs{华文仿宋}
%\newfontfamily\jljt{方正静蕾简体}
%\newfontfamily\hwxk{华文行楷}
\newcommand{\verylarge}{\fontsize{60pt}{\baselineskip}\selectfont}  
\newcommand{\chuhao}{\fontsize{44.9pt}{\baselineskip}\selectfont}  
\newcommand{\xiaochu}{\fontsize{38.5pt}{\baselineskip}\selectfont}  
\newcommand{\yihao}{\fontsize{27.8pt}{\baselineskip}\selectfont}  
\newcommand{\xiaoyi}{\fontsize{25.7pt}{\baselineskip}\selectfont}  
\newcommand{\erhao}{\fontsize{23.5pt}{\baselineskip}\selectfont}  
\newcommand{\xiaoerhao}{\fontsize{19.3pt}{\baselineskip}\selectfont} 
\newcommand{\sihao}{\fontsize{14pt}{\baselineskip}\selectfont}      % 字号设置  
\newcommand{\xiaosihao}{\fontsize{12pt}{\baselineskip}\selectfont}  % 字号设置  
\newcommand{\wuhao}{\fontsize{10.5pt}{\baselineskip}\selectfont}    % 字号设置  
\newcommand{\xiaowuhao}{\fontsize{9pt}{\baselineskip}\selectfont}   % 字号设置  
\newcommand{\liuhao}{\fontsize{7.875pt}{\baselineskip}\selectfont}  % 字号设置  
\newcommand{\qihao}{\fontsize{5.25pt}{\baselineskip}\selectfont}    % 字号设置 

\usepackage{diagbox}
\usepackage{multirow}
\boldmath
\XeTeXlinebreaklocale "zh"
\XeTeXlinebreakskip = 0pt plus 1pt minus 0.1pt
\definecolor{cred}{rgb}{0.8,0.8,0.8}
\definecolor{cgreen}{rgb}{0,0.3,0}
\definecolor{cpurple}{rgb}{0.5,0,0.35}
\definecolor{cdocblue}{rgb}{0,0,0.3}
\definecolor{cdark}{rgb}{0.95,1.0,1.0}
\lstset{
	language=bash,
	numbers=left,
	numberstyle=\tiny\color{black},
	showspaces=false,
	showstringspaces=false,
	basicstyle=\scriptsize,
	keywordstyle=\color{purple},
	commentstyle=\itshape\color{cgreen},
	stringstyle=\color{blue},
	frame=lines,
	% escapeinside=``,
	extendedchars=true, 
	xleftmargin=1em,
	xrightmargin=1em, 
	backgroundcolor=\color{cred},
	aboveskip=1em,
	breaklines=true,
	tabsize=4
} 

%\newfontfamily{\consolas}{Consolas}
%\newfontfamily{\monaco}{Monaco}
%\setmonofont[Mapping={}]{Consolas}	%英文引号之类的正常显示,相当于设置英文字体
%\setsansfont{Consolas} %设置英文字体 Monaco, Consolas,  Fantasque Sans Mono
%\setmainfont{Times New Roman}
%\setCJKmainfont{STZHONGS.TTF}
%\setmonofont{Consolas}
% \newfontfamily{\consolas}{YaHeiConsolas.ttf}
\newfontfamily{\monaco}{MONACO.TTF}
\setCJKmainfont{STZHONGS.TTF}
%\setmainfont{MONACO.TTF}
%\setsansfont{MONACO.TTF}
% 自定义添加图片命令
\newcommand{\fic}[1]{\begin{figure}[H]
		\center
		\includegraphics[width=0.8\textwidth]{#1}
	\end{figure}}
	
\newcommand{\sizedfic}[2]{\begin{figure}[H]
		\center
		\includegraphics[width=#1\textwidth]{#2}
	\end{figure}}
% 
\newcommand{\codefile}[1]{\lstinputlisting{#1}}

\newcommand{\interval}{\vspace{0.5em}}

\newcommand{\tablestart}{
	\interval
	\begin{longtable}{p{2cm}p{10cm}}
	\hline}
\newcommand{\tableend}{
	\hline
	\end{longtable}
	\interval}

% 改变段间隔
\setlength{\parskip}{0.2em}
\linespread{1.1}

\usepackage{lastpage}
% 设置页眉页脚
\usepackage{fancyhdr}
\pagestyle{fancy}
\lhead{\space \qquad \space}
\chead{案例之吴江致远\qquad}
\rhead{\qquad\thepage/\pageref{LastPage}}

% 参考文献
\def\hang{\hangindent\parindent}
\def\textindent#1{\indent\llap{#1\enspace}\ignorespaces}
\def\re{\par\hang\textindent}
%超链接
\usepackage[CJKbookmarks=true,
            bookmarksnumbered=true,
            bookmarksopen=true,
            colorlinks, %注释掉此项则交叉引用为彩色边框(将colorlinks和pdfborder同时注释掉)
            pdfborder=001,   %注释掉此项则交叉引用为彩色边框
            linkcolor=green,
            anchorcolor=green,
            citecolor=green
            ]{hyperref}  

\begin{document}
\maketitle
\clearpage
\tableofcontents
\clearpage

\section{部署恒天云服务器配置}
5台服务器,两台IBM服务器,三台惠普
\begin{center}
\begin{tabular}{|l|c|c|c|c|r|}
 \hline
节点名& 型号& CPU& 内存& 磁盘 &备注\\
 \hline
hty-controller&IBM X3650 M2 & XEON E5530*8核& 12G &300G*2 & 已使用,未做RAID\\
 \hline
hty-compute1&HP proliant DL165 G7 & AMD 6218*2块*8核& 16G &1T*2 & 已使用,未做RAID\\
 \hline
hty-compute2&IBM X3650 M2 & XEON E5530*8核& 4G &146G*2 & 未使用,故障\\
 \hline
hty-compute3&HP proliant DL388 G7& E5620*2块*8核& 16G &300G*7 & 已使用,未做RAID\\
 \hline
hty-compute4&HP proliant DL580 G5 & E7420*8核& 8G &146G*2 & 已使用,未做RAID\\
 \hline
\end{tabular}
\end{center}
\section{安装恒天云3.8.1}
\subsection{安装工具}
\begin{itemize}
	\item[1.]U盘安装iso:由于服务器的bios启动是使用UEFI,携带的Legacy盘无法识别,并且服务器老旧无法调出Legacy模式,放弃
	\item[2.]DD安装:成功安装
\end{itemize}
\subsection{网络架构}
\begin{itemize}
	\item[1]控制节点hty-controller
	eth0管理网卡,eth1是trunk网卡,在eth1上虚拟出vlan网卡vlan2254
	\begin{lstlisting}
	auto eth0
	iface eth0 inet static
	address 10.10.10.10
	netmask 255.255.255.0

	auto eth1
	iface eth1 inet static
	address 0.0.0.0
	\end{lstlisting}
	虚拟vlan网卡
	\begin{lstlisting}
	ip link add link eth1 name vlan2254 type vlan id 2254
	ip addr add 172.16.254.81/24 dev vlan2254
	ip link set vlan2254 up 
	ip route add default via 172.16.254.254 dev vlan2254
	\end{lstlisting}
	因为上述命令是临时命令,重启后会失效,需将命令添加到/etc/rc.local

	\item[2]计算节点hty-compute1
	eth0管理网卡,eth1是trunk网卡,在eth1上虚拟出vlan网卡vlan2254
	\begin{lstlisting}
	auto eth0
	iface eth0 inet static
	address 10.10.10.11
	netmask 255.255.255.0

	auto eth1
	iface eth1 inet static
	address 0.0.0.0
	\end{lstlisting}
	虚拟vlan网卡
	\begin{lstlisting}
	ip link add link eth1 name vlan2254 type vlan id 2254
	ip addr add 172.16.254.82/24 dev vlan2254
	ip link set vlan2254 up 
	ip route add default via 172.16.254.254 dev vlan2254
	\end{lstlisting}
	因为上述命令是临时命令,重启后会失效,需将命令添加到/etc/rc.local

	\item[3]计算节点hty-compute2(故障,可忽略)
	eth0管理网卡,eth1是trunk网卡,在eth1上虚拟出vlan网卡vlan2254
	\begin{lstlisting}
	auto eth0
	iface eth0 inet static
	address 10.10.10.12
	netmask 255.255.255.0

	auto eth1
	iface eth1 inet static
	address 0.0.0.0
	\end{lstlisting}
	虚拟vlan网卡
	\begin{lstlisting}
	ip link add link eth1 name vlan2254 type vlan id 2254
	ip addr add 172.16.254.83/24 dev vlan2254
	ip link set vlan2254 up 
	ip route add default via 172.16.254.254 dev vlan2254
	\end{lstlisting}
	因为上述命令是临时命令,重启后会失效,需将命令添加到/etc/rc.local

	\item[4]计算节点hty-compute3
	eth0管理网卡,eth1是trunk网卡,在eth1上虚拟出vlan网卡vlan2254
	\begin{lstlisting}
	auto eth0
	iface eth0 inet static
	address 10.10.10.13
	netmask 255.255.255.0

	auto eth1
	iface eth1 inet static
	address 0.0.0.0
	\end{lstlisting}
	虚拟vlan网卡
	\begin{lstlisting}
	ip link add link eth1 name vlan2254 type vlan id 2254
	ip addr add 172.16.254.84/24 dev vlan2254
	ip link set vlan2254 up 
	ip route add default via 172.16.254.254 dev vlan2254
	\end{lstlisting}
	因为上述命令是临时命令,重启后会失效,需将命令添加到/etc/rc.local

	\item[5]计算节点hty-compute4
	eth0管理网卡,eth1是trunk网卡,在eth1上虚拟出vlan网卡vlan2254
	\begin{lstlisting}
	auto eth0
	iface eth0 inet static
	address 10.10.10.14
	netmask 255.255.255.0

	auto eth1
	iface eth1 inet static
	address 0.0.0.0
	\end{lstlisting}
	虚拟vlan网卡
	\begin{lstlisting}
	ip link add link eth1 name vlan2254 type vlan id 2254
	ip addr add 172.16.254.85/24 dev vlan2254
	ip link set vlan2254 up 
	ip route add default via 172.16.254.254 dev vlan2254
	\end{lstlisting}
	因为上述命令是临时命令,重启后会失效,需将命令添加到/etc/rc.local
\end{itemize}
\subsection{配置文件罗列}
\subsubsection{控制节点}
\begin{itemize}
	\item[1.]/etc/network/interfaces
	\item[2.]/etc/hosts
	\item[3.]/etc/nova/nova.conf
	\item[4.]/etc/cinder/cinder.conf
	\item[5]/usr/local/nagois/libexec/忘了/config.py
	\item[6]/usr/share/openstack-dashboard/openstack-dashboad/local/hty\_settings.py
	\item[7.]/usr/share/openstack-dashboard/openstack-dashboad/local/local\_settings.py
	\item[8.]/etc/hengtianyun/hengtianyun.conf
	\item[9.]/root/watch\_services/check.conf
	\item[10.]修改一改文件的MAX\_NUM=4,将4替换为base64编码NA==
\end{itemize}
\subsubsection{计算节点}
\begin{itemize}
	\item[1.]/etc/network/interfaces
	\item[2.]/etc/hosts
	\item[3.]/etc/nova/nova.conf
	\item[4.]/etc/cinder/cinder.conf
	\item[5.]/usr/local/nagois/libexec/忘了/config.py
	\item[6.]/root/watch\_services/check.conf
	\item[7.]配置计算节点nova用户无密码访问
	\begin{lstlisting}
	su nova
	cd /var/lib/nova/.ssh/
	for i in $(seq 1 4);do ssh-copy-id -i id_rsa.pub hty-compute$i;done
	\end{lstlisting}
\end{itemize}
\subsubsection{密码更改}
\begin{itemize}
	\item[1]控制节点
	\begin{lstlisting}
	用户名:root
	密码:htYun@wjzy
	\end{lstlisting}
	\item[2]计算节点
	\begin{lstlisting}
	用户名:root
	密码:htYun@wjzy
	\end{lstlisting}
\end{itemize}
\subsubsection{监控外网使用vlan号以及ip}
\begin{lstlisting}
vlan2255:172.16.255.110
gateway 172.16.255.254
\end{lstlisting}
\end{document}