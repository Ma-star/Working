% !TeX spellcheck = en_US
%% 字体:方正静蕾简体
%%		 方正粗宋
\documentclass[a4paper,left=1.5cm,right=1.5cm,11pt]{article}
\title{案例之吴江致远}
\author{周威光整理\footnote{简介:恒天云FTE}}
\date{2017-07-12} 
% 引用宏包
\usepackage[utf8]{inputenc}
\usepackage{fontspec}
\usepackage{cite}
\usepackage{xeCJK}
\usepackage{indentfirst}
\usepackage{titlesec}
\usepackage{etoolbox}%
\makeatletter
\patchcmd{\ttlh@hang}{\parindent\z@}{\parindent\z@\leavevmode}{}{}%
\patchcmd{\ttlh@hang}{\noindent}{}{}{}%
\makeatother

\usepackage{longtable}
\usepackage{empheq}
\usepackage{graphicx}
\usepackage{float}
\usepackage{rotating}
\usepackage{subfigure}
\usepackage{tabu}
\usepackage{amsmath}
\usepackage{setspace}
\usepackage{amsfonts}
\usepackage{appendix}
\usepackage{listings}
\usepackage{xcolor}
\usepackage{geometry}
\setcounter{secnumdepth}{4}
%\titleformat*{\section}{\LARGE}
%\renewcommand\refname{参考文献}
%\titleformat{\chapter}{\centering\bfseries\huge}{}{0.7em}{}{}
\titleformat{\section}{\LARGE\bf}{\thesection}{1em}{}{}
\titleformat{\subsection}{\Large\bfseries}{\thesubsection}{1em}{}{}
\titleformat{\subsubsection}{\large\bfseries}{\thesubsubsection}{1em}{}{}
\renewcommand{\contentsname}{{ \centerline{目{  } 录}}}
\setCJKfamilyfont{cjkhwxk}{STXINGKA.TTF}
%\setCJKfamilyfont{cjkhwxk}{华文行楷}
%\setCJKfamilyfont{cjkfzcs}{方正粗宋简体}
%\newcommand*{\cjkfzcs}{\CJKfamily{cjkfzcs}}
\newcommand*{\cjkhwxk}{\CJKfamily{cjkhwxk}}
%\newfontfamily\wryh{Microsoft YaHei}
%\newfontfamily\hwzs{华文中宋}
%\newfontfamily\hwst{华文宋体}
%\newfontfamily\hwfs{华文仿宋}
%\newfontfamily\jljt{方正静蕾简体}
%\newfontfamily\hwxk{华文行楷}
\newcommand{\verylarge}{\fontsize{60pt}{\baselineskip}\selectfont}  
\newcommand{\chuhao}{\fontsize{44.9pt}{\baselineskip}\selectfont}  
\newcommand{\xiaochu}{\fontsize{38.5pt}{\baselineskip}\selectfont}  
\newcommand{\yihao}{\fontsize{27.8pt}{\baselineskip}\selectfont}  
\newcommand{\xiaoyi}{\fontsize{25.7pt}{\baselineskip}\selectfont}  
\newcommand{\erhao}{\fontsize{23.5pt}{\baselineskip}\selectfont}  
\newcommand{\xiaoerhao}{\fontsize{19.3pt}{\baselineskip}\selectfont} 
\newcommand{\sihao}{\fontsize{14pt}{\baselineskip}\selectfont}      % 字号设置  
\newcommand{\xiaosihao}{\fontsize{12pt}{\baselineskip}\selectfont}  % 字号设置  
\newcommand{\wuhao}{\fontsize{10.5pt}{\baselineskip}\selectfont}    % 字号设置  
\newcommand{\xiaowuhao}{\fontsize{9pt}{\baselineskip}\selectfont}   % 字号设置  
\newcommand{\liuhao}{\fontsize{7.875pt}{\baselineskip}\selectfont}  % 字号设置  
\newcommand{\qihao}{\fontsize{5.25pt}{\baselineskip}\selectfont}    % 字号设置 

\usepackage{diagbox}
\usepackage{multirow}
\boldmath
\XeTeXlinebreaklocale "zh"
\XeTeXlinebreakskip = 0pt plus 1pt minus 0.1pt
\definecolor{cred}{rgb}{0.8,0.8,0.8}
\definecolor{cgreen}{rgb}{0,0.3,0}
\definecolor{cpurple}{rgb}{0.5,0,0.35}
\definecolor{cdocblue}{rgb}{0,0,0.3}
\definecolor{cdark}{rgb}{0.95,1.0,1.0}
\lstset{
	language=bash,
	numbers=left,
	numberstyle=\tiny\color{black},
	showspaces=false,
	showstringspaces=false,
	basicstyle=\scriptsize,
	keywordstyle=\color{purple},
	commentstyle=\itshape\color{cgreen},
	stringstyle=\color{blue},
	frame=lines,
	% escapeinside=``,
	extendedchars=true, 
	xleftmargin=1em,
	xrightmargin=1em, 
	backgroundcolor=\color{cred},
	aboveskip=1em,
	breaklines=true,
	tabsize=4
} 

%\newfontfamily{\consolas}{Consolas}
%\newfontfamily{\monaco}{Monaco}
%\setmonofont[Mapping={}]{Consolas}	%英文引号之类的正常显示,相当于设置英文字体
%\setsansfont{Consolas} %设置英文字体 Monaco, Consolas,  Fantasque Sans Mono
%\setmainfont{Times New Roman}
%\setCJKmainfont{STZHONGS.TTF}
%\setmonofont{Consolas}
% \newfontfamily{\consolas}{YaHeiConsolas.ttf}
\newfontfamily{\monaco}{MONACO.TTF}
\setCJKmainfont{STZHONGS.TTF}
%\setmainfont{MONACO.TTF}
%\setsansfont{MONACO.TTF}
% 自定义添加图片命令
\newcommand{\fic}[1]{\begin{figure}[H]
		\center
		\includegraphics[width=0.8\textwidth]{#1}
	\end{figure}}
	
\newcommand{\sizedfic}[2]{\begin{figure}[H]
		\center
		\includegraphics[width=#1\textwidth]{#2}
	\end{figure}}
% 
\newcommand{\codefile}[1]{\lstinputlisting{#1}}

\newcommand{\interval}{\vspace{0.5em}}

\newcommand{\tablestart}{
	\interval
	\begin{longtable}{p{2cm}p{10cm}}
	\hline}
\newcommand{\tableend}{
	\hline
	\end{longtable}
	\interval}

% 改变段间隔
\setlength{\parskip}{0.2em}
\linespread{1.1}

\usepackage{lastpage}
% 设置页眉页脚
\usepackage{fancyhdr}
\pagestyle{fancy}
\lhead{\space \qquad \space}
\chead{案例之无锡同步电子\qquad}
\rhead{\qquad\thepage/\pageref{LastPage}}

% 参考文献
\def\hang{\hangindent\parindent}
\def\textindent#1{\indent\llap{#1\enspace}\ignorespaces}
\def\re{\par\hang\textindent}
%超链接
\usepackage[CJKbookmarks=true,
            bookmarksnumbered=true,
            bookmarksopen=true,
            colorlinks, %注释掉此项则交叉引用为彩色边框(将colorlinks和pdfborder同时注释掉)
            pdfborder=001,   %注释掉此项则交叉引用为彩色边框
            linkcolor=green,
            anchorcolor=green,
            citecolor=green
            ]{hyperref}  

\begin{document}
\maketitle
\clearpage
\tableofcontents
\clearpage

\section{安装恒天云3.7.1}
\subsection{网络架构}
\begin{itemize}
	\item[1]控制节点hty-controller
	eth0管理网卡,eth1 eth4是ceph的网卡,eth3外网,eth5配置0.0.0.0允许虚拟机带vlan的包
	\begin{lstlisting}
	auto eth0
	iface eth0 inet static
	address 10.10.10.10
	netmask 255.255.255.0

	auto eth1
	iface eth1 inet static
	address 10.20.10.10
	netmask 255.255.255.0

	auto eth3
	iface eth3 inet static
	address 172.16.40.10
	netmask 255.255.255.0
	gateway 172.16.40.254
	dns-nameservers 8.8.8.8(忘了)
	
	auto eth4
	iface eth4 inet static
	address 10.30.10.10
	netmask 255.255.255.0

	auto eth5
	iface eth5 inet static
	address 0.0.0.0
	\end{lstlisting}

	\item[1]控制节点hty-controller
	eth0管理网卡,eth1 eth4是ceph的网卡,eth3外网,eth5配置0.0.0.0允许虚拟机带vlan的包
	\begin{lstlisting}
	auto eth0
	iface eth0 inet static
	address 10.10.10.10
	netmask 255.255.255.0

	auto eth1
	iface eth1 inet static
	address 10.20.10.10
	netmask 255.255.255.0

	auto eth3
	iface eth3 inet static
	address 172.16.40.10
	netmask 255.255.255.0
	gateway 172.16.40.254
	dns-nameservers 8.8.8.8(忘了)
	
	auto eth4
	iface eth4 inet static
	address 10.30.10.10
	netmask 255.255.255.0

	auto eth5
	iface eth5 inet static
	address 0.0.0.0
	\end{lstlisting}

	\item[1]控制节点hty-controller
	eth0管理网卡,eth1 eth4是ceph的网卡,eth3外网,eth5配置0.0.0.0允许虚拟机带vlan的包
	\begin{lstlisting}
	auto eth0
	iface eth0 inet static
	address 10.10.10.10
	netmask 255.255.255.0

	auto eth1
	iface eth1 inet static
	address 10.20.10.10
	netmask 255.255.255.0

	auto eth3
	iface eth3 inet static
	address 172.16.40.10
	netmask 255.255.255.0
	gateway 172.16.40.254
	dns-nameservers 8.8.8.8(忘了)
	
	auto eth4
	iface eth4 inet static
	address 10.30.10.10
	netmask 255.255.255.0

	auto eth5
	iface eth5 inet static
	address 0.0.0.0
	\end{lstlisting}

	\item[1]控制节点hty-controller
	eth0管理网卡,eth1 eth4是ceph的网卡,eth3外网,eth5配置0.0.0.0允许虚拟机带vlan的包
	\begin{lstlisting}
	auto eth0
	iface eth0 inet static
	address 10.10.10.10
	netmask 255.255.255.0

	auto eth1
	iface eth1 inet static
	address 10.20.10.10
	netmask 255.255.255.0

	auto eth3
	iface eth3 inet static
	address 172.16.40.10
	netmask 255.255.255.0
	gateway 172.16.40.254
	dns-nameservers 8.8.8.8(忘了)
	
	auto eth4
	iface eth4 inet static
	address 10.30.10.10
	netmask 255.255.255.0

	auto eth5
	iface eth5 inet static
	address 0.0.0.0
	\end{lstlisting}

	\item[2]计算节点hty-compute1
	eth0管理网卡,eth1 eth4是ceph的网卡,eth3外网,eth5配置0.0.0.0允许虚拟机带vlan的包
	\begin{lstlisting}
	auto eth0
	iface eth0 inet static
	address 10.10.10.11
	netmask 255.255.255.0

	auto eth1
	iface eth1 inet static
	address 10.20.10.11
	netmask 255.255.255.0

	auto eth3
	iface eth3 inet static
	address 172.16.40.11
	netmask 255.255.255.0
	gateway 172.16.40.254
	dns-nameservers 8.8.8.8(忘了)
	
	auto eth4
	iface eth4 inet static
	address 10.30.10.11
	netmask 255.255.255.0

	auto eth5
	iface eth5 inet static
	address 0.0.0.0
	\end{lstlisting}

	\item[3]计算节点hty-compute2
	eth0管理网卡,eth1 eth4是ceph的网卡,eth3外网,eth5配置0.0.0.0允许虚拟机带vlan的包
	\begin{lstlisting}
	auto eth0
	iface eth0 inet static
	address 10.10.10.12
	netmask 255.255.255.0

	auto eth1
	iface eth1 inet static
	address 10.20.10.12
	netmask 255.255.255.0

	auto eth3
	iface eth3 inet static
	address 172.16.40.12
	netmask 255.255.255.0
	gateway 172.16.40.254
	dns-nameservers 8.8.8.8(忘了)
	
	auto eth4
	iface eth4 inet static
	address 10.30.10.12
	netmask 255.255.255.0

	auto eth5
	iface eth5 inet static
	address 0.0.0.0
	\end{lstlisting}

	\item[4]计算节点hty-compute3
	eth0管理网卡,eth1 eth4是ceph的网卡,eth3外网,eth5配置0.0.0.0允许虚拟机带vlan的包
	\begin{lstlisting}
	auto eth0
	iface eth0 inet static
	address 10.10.10.13
	netmask 255.255.255.0

	auto eth1
	iface eth1 inet static
	address 10.20.10.13
	netmask 255.255.255.0

	auto eth3
	iface eth3 inet static
	address 172.16.40.13
	netmask 255.255.255.0
	gateway 172.16.40.254
	dns-nameservers 8.8.8.8(忘了)
	
	auto eth4
	iface eth4 inet static
	address 10.30.10.13
	netmask 255.255.255.0

	auto eth5
	iface eth5 inet static
	address 0.0.0.0
	\end{lstlisting}

	\item[5]计算节点hty-compute4
	eth0管理网卡,eth1 eth4是ceph的网卡,eth3外网,eth5配置0.0.0.0允许虚拟机带vlan的包
	\begin{lstlisting}
	auto eth0
	iface eth0 inet static
	address 10.10.10.14
	netmask 255.255.255.0

	auto eth1
	iface eth1 inet static
	address 10.20.10.14
	netmask 255.255.255.0

	auto eth3
	iface eth3 inet static
	address 172.16.40.14
	netmask 255.255.255.0
	gateway 172.16.40.254
	dns-nameservers 8.8.8.8(忘了)
	
	auto eth4
	iface eth4 inet static
	address 10.30.10.14
	netmask 255.255.255.0

	auto eth5
	iface eth5 inet static
	address 0.0.0.0
	\end{lstlisting}

	

\end{itemize}



\end{document}