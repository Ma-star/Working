% !TeX spellcheck = en_US
%% 字体:方正静蕾简体
%%		 方正粗宋
\documentclass[a4paper,left=1.5cm,right=1.5cm,11pt]{article}

\usepackage[utf8]{inputenc}
\usepackage{fontspec}
\usepackage{cite}
\usepackage{xeCJK}
\usepackage{indentfirst}
\usepackage{titlesec}
\usepackage{etoolbox}%
\makeatletter
\patchcmd{\ttlh@hang}{\parindent\z@}{\parindent\z@\leavevmode}{}{}%
\patchcmd{\ttlh@hang}{\noindent}{}{}{}%
\makeatother

\usepackage{longtable}
\usepackage{empheq}
\usepackage{graphicx}
\usepackage{float}
\usepackage{rotating}
\usepackage{subfigure}
\usepackage{tabu}
\usepackage{amsmath}
\usepackage{setspace}
\usepackage{amsfonts}
\usepackage{appendix}
\usepackage{listings}
\usepackage{xcolor}
\usepackage{geometry}
\setcounter{secnumdepth}{4}
%\titleformat*{\section}{\LARGE}
%\renewcommand\refname{参考文献}
%\titleformat{\chapter}{\centering\bfseries\huge}{}{0.7em}{}{}
\titleformat{\section}{\LARGE\bf}{\thesection}{1em}{}{}
\titleformat{\subsection}{\Large\bfseries}{\thesubsection}{1em}{}{}
\titleformat{\subsubsection}{\large\bfseries}{\thesubsubsection}{1em}{}{}
\renewcommand{\contentsname}{{ \centerline{目{  } 录}}}
\setCJKfamilyfont{cjkhwxk}{STXINGKA.TTF}
%\setCJKfamilyfont{cjkhwxk}{华文行楷}
%\setCJKfamilyfont{cjkfzcs}{方正粗宋简体}
%\newcommand*{\cjkfzcs}{\CJKfamily{cjkfzcs}}
\newcommand*{\cjkhwxk}{\CJKfamily{cjkhwxk}}
%\newfontfamily\wryh{Microsoft YaHei}
%\newfontfamily\hwzs{华文中宋}
%\newfontfamily\hwst{华文宋体}
%\newfontfamily\hwfs{华文仿宋}
%\newfontfamily\jljt{方正静蕾简体}
%\newfontfamily\hwxk{华文行楷}
\newcommand{\verylarge}{\fontsize{60pt}{\baselineskip}\selectfont}  
\newcommand{\chuhao}{\fontsize{44.9pt}{\baselineskip}\selectfont}  
\newcommand{\xiaochu}{\fontsize{38.5pt}{\baselineskip}\selectfont}  
\newcommand{\yihao}{\fontsize{27.8pt}{\baselineskip}\selectfont}  
\newcommand{\xiaoyi}{\fontsize{25.7pt}{\baselineskip}\selectfont}  
\newcommand{\erhao}{\fontsize{23.5pt}{\baselineskip}\selectfont}  
\newcommand{\xiaoerhao}{\fontsize{19.3pt}{\baselineskip}\selectfont} 
\newcommand{\sihao}{\fontsize{14pt}{\baselineskip}\selectfont}      % 字号设置  
\newcommand{\xiaosihao}{\fontsize{12pt}{\baselineskip}\selectfont}  % 字号设置  
\newcommand{\wuhao}{\fontsize{10.5pt}{\baselineskip}\selectfont}    % 字号设置  
\newcommand{\xiaowuhao}{\fontsize{9pt}{\baselineskip}\selectfont}   % 字号设置  
\newcommand{\liuhao}{\fontsize{7.875pt}{\baselineskip}\selectfont}  % 字号设置  
\newcommand{\qihao}{\fontsize{5.25pt}{\baselineskip}\selectfont}    % 字号设置 

\usepackage{diagbox}
\usepackage{multirow}
\boldmath
\XeTeXlinebreaklocale "zh"
\XeTeXlinebreakskip = 0pt plus 1pt minus 0.1pt
\definecolor{cred}{rgb}{0.8,0.8,0.8}
\definecolor{cgreen}{rgb}{0,0.3,0}
\definecolor{cpurple}{rgb}{0.5,0,0.35}
\definecolor{cdocblue}{rgb}{0,0,0.3}
\definecolor{cdark}{rgb}{0.95,1.0,1.0}
\lstset{
	language=bash,
	numbers=left,
	numberstyle=\tiny\color{black},
	showspaces=false,
	showstringspaces=false,
	basicstyle=\scriptsize,
	keywordstyle=\color{purple},
	commentstyle=\itshape\color{cgreen},
	stringstyle=\color{blue},
	frame=lines,
	% escapeinside=``,
	extendedchars=true, 
	xleftmargin=1em,
	xrightmargin=1em, 
	backgroundcolor=\color{cred},
	aboveskip=1em,
	breaklines=true,
	tabsize=4
} 

%\newfontfamily{\consolas}{Consolas}
%\newfontfamily{\monaco}{Monaco}
%\setmonofont[Mapping={}]{Consolas}	%英文引号之类的正常显示,相当于设置英文字体
%\setsansfont{Consolas} %设置英文字体 Monaco, Consolas,  Fantasque Sans Mono
%\setmainfont{Times New Roman}
%\setCJKmainfont{STZHONGS.TTF}
%\setmonofont{Consolas}
% \newfontfamily{\consolas}{YaHeiConsolas.ttf}
\newfontfamily{\monaco}{MONACO.TTF}
\setCJKmainfont{STZHONGS.TTF}
%\setmainfont{MONACO.TTF}
%\setsansfont{MONACO.TTF}

\newcommand{\fic}[1]{\begin{figure}[H]
		\center
		\includegraphics[width=0.8\textwidth]{#1}
	\end{figure}}
	
\newcommand{\sizedfic}[2]{\begin{figure}[H]
		\center
		\includegraphics[width=#1\textwidth]{#2}
	\end{figure}}

\newcommand{\codefile}[1]{\lstinputlisting{#1}}

\newcommand{\interval}{\vspace{0.5em}}

\newcommand{\tablestart}{
	\interval
	\begin{longtable}{p{2cm}p{10cm}}
	\hline}
\newcommand{\tableend}{
	\hline
	\end{longtable}
	\interval}

% 改变段间隔
\setlength{\parskip}{0.2em}
\linespread{1.1}

\usepackage{lastpage}
\usepackage{fancyhdr}
\pagestyle{fancy}
\lhead{\space \qquad \space}
\chead{恒天云集成neutron计划 \qquad}
\rhead{\qquad\thepage/\pageref{LastPage}}

\begin{document}

\tableofcontents

\clearpage


\section{openstack命令大全}

\subsection{rabbit命令大全}
\begin{lstlisting}
----------------------------------------------------------
rabbitmqctl [-n <node>] [-q] <command> [<command options>] 
Options:
    -n node #控制节点
    -q      #启用安静模式,压缩输出信息
Commands:
	add_user <username> <password>
    delete_user <username>
    change_password <username> <newpassword>
    clear_password <username>
    set_user_tags <username> <tag> ...
    list_users
----------------------------------------------------------
\end{lstlisting}

\subsection{keystone命令大全}
\begin{lstlisting}
--------------------------------------------------------------------------------------
usage: keystone [--version] [--debug] [--os-username <auth-user-name>]
                [--os-password <auth-password>]
                [--os-tenant-name <auth-tenant-name>]
                [--os-tenant-id <tenant-id>] [--os-auth-url <auth-url>]
                [--os-region-name <region-name>]
                [--os-identity-api-version <identity-api-version>]
                [--os-token <service-token>]
                [--os-endpoint <service-endpoint>] [--os-cache]
                [--force-new-token] [--stale-duration <seconds>] [--insecure]
                [--os-cacert <ca-certificate>] [--os-cert <certificate>]
                [--os-key <key>] [--timeout <seconds>]
                <subcommand> 
 <subcommand>
    catalog             List service catalog, possibly filtered by service.
    ec2-credentials-create
                        Create EC2-compatible credentials for user per tenant.
    ec2-credentials-delete
                        Delete EC2-compatible credentials.
    ec2-credentials-get
                        Display EC2-compatible credentials.
    ec2-credentials-list
                        List EC2-compatible credentials for a user.
    endpoint-create     Create a new endpoint associated with a service.
    endpoint-delete     Delete a service endpoint.
    endpoint-get        Find endpoint filtered by a specific attribute or
                        service type.
    endpoint-list       List configured service endpoints.
    password-update     Update own password.
    role-create         Create new role.
    role-delete         Delete role.
    role-get            Display role details.
    role-list           List all roles.
    service-create      Add service to Service Catalog.
    service-delete      Delete service from Service Catalog.
    service-get         Display service from Service Catalog.
    service-list        List all services in Service Catalog.
    tenant-create       Create new tenant.
    tenant-delete       Delete tenant.
    tenant-get          Display tenant details.
    tenant-list         List all tenants.
    tenant-update       Update tenant name, description, enabled status.
    token-get           Display the current user token.
    user-create         Create new user.
    user-delete         Delete user.
    user-get            Display user details.
    user-list           List users.
    user-password-update
                        Update user password.
    user-role-add       Add role to user.
    user-role-list      List roles granted to a user.
    user-role-remove    Remove role from user.
    user-update         Update user's name, email, and enabled status.
    discover            Discover Keystone servers, supported API versions and
                        extensions.
    bootstrap           Grants a new role to a new user on a new tenant, after
                        creating each.
    bash-completion     Prints all of the commands and options to stdout.
    help                Display help about this program or one of its
                        subcommands.
Optional arguments:
  --version             Shows the client version and exits.
  --debug               Prints debugging output onto the console, this
                        includes the curl request and response calls. Helpful
                        for debugging and understanding the API calls.
  --os-username <auth-user-name>
                        Name used for authentication with the OpenStack
                        Identity service. Defaults to env[OS_USERNAME].
  --os-password <auth-password>
                        Password used for authentication with the OpenStack
                        Identity service. Defaults to env[OS_PASSWORD].
  --os-tenant-name <auth-tenant-name>
                        Tenant to request authorization on. Defaults to
                        env[OS_TENANT_NAME].
  --os-tenant-id <tenant-id>
                        Tenant to request authorization on. Defaults to
                        env[OS_TENANT_ID].
  --os-auth-url <auth-url>
                        Specify the Identity endpoint to use for
                        authentication. Defaults to env[OS_AUTH_URL].
  --os-region-name <region-name>
                        Specify the region to use. Defaults to
                        env[OS_REGION_NAME].
  --os-identity-api-version <identity-api-version>
                        Specify Identity API version to use. Defaults to
                        env[OS_IDENTITY_API_VERSION] or 2.0.
  --os-token <service-token>
                        Specify an existing token to use instead of retrieving
                        one via authentication (e.g. with username &
                        password). Defaults to env[OS_SERVICE_TOKEN].
  --os-endpoint <service-endpoint>
                        Specify an endpoint to use instead of retrieving one
                        from the service catalog (via authentication).
                        Defaults to env[OS_SERVICE_ENDPOINT].
  --os-cache            Use the auth token cache. Defaults to env[OS_CACHE].
  --force-new-token     If the keyring is available and in use, token will
                        always be stored and fetched from the keyring until
                        the token has expired. Use this option to request a
                        new token and replace the existing one in the keyring.
  --stale-duration <seconds>
                        Stale duration (in seconds) used to determine whether
                        a token has expired when retrieving it from keyring.
                        This is useful in mitigating process or network
                        delays. Default is 30 seconds.
  --insecure            Explicitly allow client to perform "insecure" TLS
                        (https) requests. The server's certificate will not be
                        verified against any certificate authorities. This
                        option should be used with caution.
  --os-cacert <ca-certificate>
                        Specify a CA bundle file to use in verifying a TLS
                        (https) server certificate. Defaults to
                        env[OS_CACERT].
  --os-cert <certificate>
                        Defaults to env[OS_CERT].
  --os-key <key>        Defaults to env[OS_KEY].
  --timeout <seconds>   Set request timeout (in seconds).
  See "keystone help COMMAND" for help on a specific command
--------------------------------------------------------------------------------------
\end{lstlisting}

\subsection{glance命令大全}
\begin{lstlisting}
--------------------------------------------------------------------------------------
usage: glance [--version] [-d] [-v] [--get-schema] [--timeout TIMEOUT]
              [--no-ssl-compression] [-f] [--os-image-url OS_IMAGE_URL]
              [--os-image-api-version OS_IMAGE_API_VERSION]
              [--profile HMAC_KEY] [-k] [--os-cert OS_CERT]
              [--cert-file OS_CERT] [--os-key OS_KEY] [--key-file OS_KEY]
              [--os-cacert <ca-certificate-file>] [--ca-file OS_CACERT]
              [--os-username OS_USERNAME] [--os-user-id OS_USER_ID]
              [--os-user-domain-id OS_USER_DOMAIN_ID]
              [--os-user-domain-name OS_USER_DOMAIN_NAME]
              [--os-project-id OS_PROJECT_ID]
              [--os-project-name OS_PROJECT_NAME]
              [--os-project-domain-id OS_PROJECT_DOMAIN_ID]
              [--os-project-domain-name OS_PROJECT_DOMAIN_NAME]
              [--os-password OS_PASSWORD] [--os-tenant-id OS_TENANT_ID]
              [--os-tenant-name OS_TENANT_NAME] [--os-auth-url OS_AUTH_URL]
              [--os-region-name OS_REGION_NAME]
              [--os-auth-token OS_AUTH_TOKEN]
              [--os-service-type OS_SERVICE_TYPE]
              [--os-endpoint-type OS_ENDPOINT_TYPE]
              <subcommand> ...

Command-line interface to the OpenStack Images API.

Positional arguments:
  <subcommand>
    image-create        Create a new image.
    image-delete        Delete specified image(s).
    image-download      Download a specific image.
    image-list          List images you can access.
    image-show          Describe a specific image.
    image-update        Update a specific image.
    member-create       Share a specific image with a tenant.
    member-delete       Remove a shared image from a tenant.
    member-list         Describe sharing permissions by image or tenant.
    help                Display help about this program or one of its
                        subcommands.

Optional arguments:
  --version             show program's version number and exit
  -d, --debug           Defaults to env[GLANCECLIENT_DEBUG].
  -v, --verbose         Print more verbose output
  --get-schema          Ignores cached copy and forces retrieval of schema
                        that generates portions of the help text. Ignored with
                        API version 1.
  --timeout TIMEOUT     Number of seconds to wait for a response
  --no-ssl-compression  Disable SSL compression when using https.
  -f, --force           Prevent select actions from requesting user
                        confirmation.
  --os-image-url OS_IMAGE_URL
                        Defaults to env[OS_IMAGE_URL].
  --os-image-api-version OS_IMAGE_API_VERSION
                        Defaults to env[OS_IMAGE_API_VERSION] or 1.
  --profile HMAC_KEY    HMAC key to use for encrypting context data for
                        performance profiling of operation. This key should be
                        the value of HMAC key configured in osprofiler
                        middleware in glance, it is specified in paste
                        configuration file at /etc/glance/api-paste.ini and
                        /etc/glance/registry-paste.ini. Without key the
                        profiling will not be triggered even if osprofiler is
                        enabled on server side.
  -k, --insecure        Explicitly allow glanceclient to perform "insecure
                        SSL" (https) requests. The server's certificate will
                        not be verified against any certificate authorities.
                        This option should be used with caution.
  --os-cert OS_CERT     Path of certificate file to use in SSL connection.
                        This file can optionally be prepended with the private
                        key.
  --cert-file OS_CERT   DEPRECATED! Use --os-cert.
  --os-key OS_KEY       Path of client key to use in SSL connection. This
                        option is not necessary if your key is prepended to
                        your cert file.
  --key-file OS_KEY     DEPRECATED! Use --os-key.
  --os-cacert <ca-certificate-file>
                        Path of CA TLS certificate(s) used to verify the
                        remote server's certificate. Without this option
                        glance looks for the default system CA certificates.
  --ca-file OS_CACERT   DEPRECATED! Use --os-cacert.
  --os-username OS_USERNAME
                        Defaults to env[OS_USERNAME].
  --os-user-id OS_USER_ID
                        Defaults to env[OS_USER_ID].
  --os-user-domain-id OS_USER_DOMAIN_ID
                        Defaults to env[OS_USER_DOMAIN_ID].
  --os-user-domain-name OS_USER_DOMAIN_NAME
                        Defaults to env[OS_USER_DOMAIN_NAME].
  --os-project-id OS_PROJECT_ID
                        Another way to specify tenant ID. This option is
                        mutually exclusive with --os-tenant-id. Defaults to
                        env[OS_PROJECT_ID].
  --os-project-name OS_PROJECT_NAME
                        Another way to specify tenant name. This option is
                        mutually exclusive with --os-tenant-name. Defaults to
                        env[OS_PROJECT_NAME].
  --os-project-domain-id OS_PROJECT_DOMAIN_ID
                        Defaults to env[OS_PROJECT_DOMAIN_ID].
  --os-project-domain-name OS_PROJECT_DOMAIN_NAME
                        Defaults to env[OS_PROJECT_DOMAIN_NAME].
  --os-password OS_PASSWORD
                        Defaults to env[OS_PASSWORD].
  --os-tenant-id OS_TENANT_ID
                        Defaults to env[OS_TENANT_ID].
  --os-tenant-name OS_TENANT_NAME
                        Defaults to env[OS_TENANT_NAME].
  --os-auth-url OS_AUTH_URL
                        Defaults to env[OS_AUTH_URL].
  --os-region-name OS_REGION_NAME
                        Defaults to env[OS_REGION_NAME].
  --os-auth-token OS_AUTH_TOKEN
                        Defaults to env[OS_AUTH_TOKEN].
  --os-service-type OS_SERVICE_TYPE
                        Defaults to env[OS_SERVICE_TYPE].
  --os-endpoint-type OS_ENDPOINT_TYPE
                        Defaults to env[OS_ENDPOINT_TYPE].

See "glance help COMMAND" for help on a specific command.
--------------------------------------------------------------------------------------
\end{lstlisting}
\subsection{nova命令大全}
\begin{lstlisting}
--------------------------------------------------------------------------------------
usage: nova [--version] [--debug] [--os-cache] [--timings]
            [--timeout <seconds>] [--os-auth-token OS_AUTH_TOKEN]
            [--os-username <auth-user-name>] [--os-user-id <auth-user-id>]
            [--os-password <auth-password>]
            [--os-tenant-name <auth-tenant-name>]
            [--os-tenant-id <auth-tenant-id>] [--os-auth-url <auth-url>]
            [--os-region-name <region-name>] [--os-auth-system <auth-system>]
            [--service-type <service-type>] [--service-name <service-name>]
            [--volume-service-name <volume-service-name>]
            [--endpoint-type <endpoint-type>]
            [--os-compute-api-version <compute-api-ver>]
            [--os-cacert <ca-certificate>] [--insecure]
            [--bypass-url <bypass-url>]
            <subcommand> ...

Command-line interface to the OpenStack Nova API.

Positional arguments:
  <subcommand>
    absolute-limits             Print a list of absolute limits for a user
    add-fixed-ip                Add new IP address on a network to server.
    add-floating-ip             DEPRECATED, use floating-ip-associate instead.
    add-secgroup                Add a Security Group to a server.
    agent-create                Create new agent build.
    agent-delete                Delete existing agent build.
    agent-list                  List all builds.
    agent-modify                Modify existing agent build.
    aggregate-add-host          Add the host to the specified aggregate.
    aggregate-create            Create a new aggregate with the specified
                                details.
    aggregate-delete            Delete the aggregate.
    aggregate-details           Show details of the specified aggregate.
    aggregate-list              Print a list of all aggregates.
    aggregate-remove-host       Remove the specified host from the specified
                                aggregate.
    aggregate-set-metadata      Update the metadata associated with the
                                aggregate.
    aggregate-update            Update the aggregate's name and optionally
                                availability zone.
    availability-zone-list      List all the availability zones.
    backup                      Backup a server by creating a 'backup' type
                                snapshot.
    boot                        Boot a new server.
    clear-password              Clear password for a server.
    cloudpipe-configure         Update the VPN IP/port of a cloudpipe
                                instance.
    cloudpipe-create            Create a cloudpipe instance for the given
                                project.
    cloudpipe-list              Print a list of all cloudpipe instances.
    console-log                 Get console log output of a server.
    credentials                 Show user credentials returned from auth.
    delete                      Immediately shut down and delete specified
                                server(s).
    diagnostics                 Retrieve server diagnostics.
    dns-create                  Create a DNS entry for domain, name and ip.
    dns-create-private-domain   Create the specified DNS domain.
    dns-create-public-domain    Create the specified DNS domain.
    dns-delete                  Delete the specified DNS entry.
    dns-delete-domain           Delete the specified DNS domain.
    dns-domains                 Print a list of available dns domains.
    dns-list                    List current DNS entries for domain and ip or
                                domain and name.
    endpoints                   Discover endpoints that get returned from the
                                authenticate services.
    evacuate                    Evacuate server from failed host.
    fixed-ip-get                Retrieve info on a fixed ip.
    fixed-ip-reserve            Reserve a fixed IP.
    fixed-ip-unreserve          Unreserve a fixed IP.
    flavor-access-add           Add flavor access for the given tenant.
    flavor-access-list          Print access information about the given
                                flavor.
    flavor-access-remove        Remove flavor access for the given tenant.
    flavor-create               Create a new flavor
    flavor-delete               Delete a specific flavor
    flavor-key                  Set or unset extra_spec for a flavor.
    flavor-list                 Print a list of available 'flavors' (sizes of
                                servers).
    flavor-show                 Show details about the given flavor.
    floating-ip-associate       Associate a floating IP address to a server.
    floating-ip-bulk-create     Bulk create floating ips by range.
    floating-ip-bulk-delete     Bulk delete floating ips by range.
    floating-ip-bulk-list       List all floating ips.
    floating-ip-create          Allocate a floating IP for the current tenant.
    floating-ip-delete          De-allocate a floating IP.
    floating-ip-disassociate    Disassociate a floating IP address from a
                                server.
    floating-ip-list            List floating ips.
    floating-ip-pool-list       List all floating ip pools.
    get-password                Get password for a server.
    get-rdp-console             Get a rdp console to a server.
    get-spice-console           Get a spice console to a server.
    get-vnc-console             Get a vnc console to a server.
    host-action                 Perform a power action on a host.
    host-describe               Describe a specific host.
    host-list                   List all hosts by service.
    host-update                 Update host settings.
    hypervisor-list             List hypervisors.
    hypervisor-servers          List servers belonging to specific
                                hypervisors.
    hypervisor-show             Display the details of the specified
                                hypervisor.
    hypervisor-stats            Get hypervisor statistics over all compute
                                nodes.
    hypervisor-uptime           Display the uptime of the specified
                                hypervisor.
    image-create                Create a new image by taking a snapshot of a
                                running server.
    image-delete                Delete specified image(s).
    image-list                  Print a list of available images to boot from.
    image-meta                  Set or Delete metadata on an image.
    image-show                  Show details about the given image.
    interface-attach            Attach a network interface to a server.
    interface-detach            Detach a network interface from a server.
    interface-list              List interfaces attached to a server.
    keypair-add                 Create a new key pair for use with servers.
    keypair-delete              Delete keypair given by its name.
    keypair-list                Print a list of keypairs for a user
    keypair-show                Show details about the given keypair.
    list                        List active servers.
    list-secgroup               List Security Group(s) of a server.
    live-migration              Migrate running server to a new machine.
    lock                        Lock a server.
    meta                        Set or Delete metadata on a server.
    migrate                     Migrate a server. The new host will be
                                selected by the scheduler.
    network-associate-host      Associate host with network.
    network-associate-project   Associate project with network.
    network-create              Create a network.
    network-delete              Delete network by label or id.
    network-disassociate        Disassociate host and/or project from the
                                given network.
    network-list                Print a list of available networks.
    network-show                Show details about the given network.
    pause                       Pause a server.
    quota-class-show            List the quotas for a quota class.
    quota-class-update          Update the quotas for a quota class.
    quota-defaults              List the default quotas for a tenant.
    quota-delete                Delete quota for a tenant/user so their quota
                                will Revert back to default.
    quota-show                  List the quotas for a tenant/user.
    quota-update                Update the quotas for a tenant/user.
    rate-limits                 Print a list of rate limits for a user
    reboot                      Reboot a server.
    rebuild                     Shutdown, re-image, and re-boot a server.
    refresh-network             Refresh server network information.
    remove-fixed-ip             Remove an IP address from a server.
    remove-floating-ip          DEPRECATED, use floating-ip-disassociate
                                instead.
    remove-secgroup             Remove a Security Group from a server.
    rename                      Rename a server.
    rescue                      Reboots a server into rescue mode, which
                                starts the machine from the initial image,
                                attaching the current boot disk as secondary.
    reset-network               Reset network of a server.
    reset-state                 Reset the state of a server.
    resize                      Resize a server.
    resize-confirm              Confirm a previous resize.
    resize-revert               Revert a previous resize (and return to the
                                previous VM).
    resume                      Resume a server.
    root-password               Change the root password for a server.
    scrub                       Delete data associated with the project.
    secgroup-add-default-rule   Add a rule to the default security group.
    secgroup-add-group-rule     Add a source group rule to a security group.
    secgroup-add-rule           Add a rule to a security group.
    secgroup-create             Create a security group.
    secgroup-delete             Delete a security group.
    secgroup-delete-default-rule
                                Delete a rule from the default security group.
    secgroup-delete-group-rule  Delete a source group rule from a security
                                group.
    secgroup-delete-rule        Delete a rule from a security group.
    secgroup-list               List security groups for the current tenant.
    secgroup-list-default-rules
                                List rules for the default security group.
    secgroup-list-rules         List rules for a security group.
    secgroup-update             Update a security group.
    server-group-create         Create a new server group with the specified
                                details.
    server-group-delete         Delete specific server group(s).
    server-group-get            Get a specific server group.
    server-group-list           Print a list of all server groups.
    service-delete              Delete the service.
    service-disable             Disable the service.
    service-enable              Enable the service.
    service-list                Show a list of all running services. Filter by
                                host & binary.
    shelve                      Shelve a server.
    shelve-offload              Remove a shelved server from the compute node.
    show                        Show details about the given server.
    ssh                         SSH into a server.
    start                       Start a server.
    stop                        Stop a server.
    suspend                     Suspend a server.
    unlock                      Unlock a server.
    unpause                     Unpause a server.
    unrescue                    Restart the server from normal boot disk
                                again.
    unshelve                    Unshelve a server.
    usage                       Show usage data for a single tenant.
    usage-list                  List usage data for all tenants.
    version-list                List all API versions.
    volume-attach               Attach a volume to a server.
    volume-create               Add a new volume.
    volume-delete               Remove volume(s).
    volume-detach               Detach a volume from a server.
    volume-list                 List all the volumes.
    volume-show                 Show details about a volume.
    volume-snapshot-create      Add a new snapshot.
    volume-snapshot-delete      Remove a snapshot.
    volume-snapshot-list        List all the snapshots.
    volume-snapshot-show        Show details about a snapshot.
    volume-type-create          Create a new volume type.
    volume-type-delete          Delete a specific volume type.
    volume-type-list            Print a list of available 'volume types'.
    volume-update               Update volume attachment.
    x509-create-cert            Create x509 cert for a user in tenant.
    x509-get-root-cert          Fetch the x509 root cert.
    bash-completion             Prints all of the commands and options to
                                stdout so that the nova.bash_completion script
                                doesn't have to hard code them.
    help                        Display help about this program or one of its
                                subcommands.
    list-extensions             List all the os-api extensions that are
                                available.
    host-servers-migrate        Migrate all instances of the specified host to
                                other available hosts.
    cell-capacities             Get cell capacities for all cells or a given
                                cell.
    cell-show                   Show details of a given cell.
    force-delete                Force delete a server.
    restore                     Restore a soft-deleted server.
    baremetal-interface-add     Add a network interface to a baremetal node.
    baremetal-interface-list    List network interfaces associated with a
                                baremetal node.
    baremetal-interface-remove  Remove a network interface from a baremetal
                                node.
    baremetal-node-create       Create a baremetal node.
    baremetal-node-delete       Remove a baremetal node and any associated
                                interfaces.
    baremetal-node-list         Print list of available baremetal nodes.
    baremetal-node-show         Show information about a baremetal node.
    host-meta                   Set or Delete metadata on all instances of a
                                host.
    host-evacuate               Evacuate all instances from failed host.
    instance-action             Show an action.
    instance-action-list        List actions on a server.
    migration-list              Print a list of migrations.
    net                         Show a network
    net-create                  Create a network
    net-delete                  Delete a network
    net-list                    List networks

Optional arguments:
  --version                     show program's version number and exit
  --debug                       Print debugging output
  --os-cache                    Use the auth token cache. Defaults to False if
                                env[OS_CACHE] is not set.
  --timings                     Print call timing info
  --timeout <seconds>           Set HTTP call timeout (in seconds)
  --os-auth-token OS_AUTH_TOKEN
                                Defaults to env[OS_AUTH_TOKEN]
  --os-username <auth-user-name>
                                Defaults to env[OS_USERNAME].
  --os-user-id <auth-user-id>   Defaults to env[OS_USER_ID].
  --os-password <auth-password>
                                Defaults to env[OS_PASSWORD].
  --os-tenant-name <auth-tenant-name>
                                Defaults to env[OS_TENANT_NAME].
  --os-tenant-id <auth-tenant-id>
                                Defaults to env[OS_TENANT_ID].
  --os-auth-url <auth-url>      Defaults to env[OS_AUTH_URL].
  --os-region-name <region-name>
                                Defaults to env[OS_REGION_NAME].
  --os-auth-system <auth-system>
                                Defaults to env[OS_AUTH_SYSTEM].
  --service-type <service-type>
                                Defaults to compute for most actions
  --service-name <service-name>
                                Defaults to env[NOVA_SERVICE_NAME]
  --volume-service-name <volume-service-name>
                                Defaults to env[NOVA_VOLUME_SERVICE_NAME]
  --endpoint-type <endpoint-type>
                                Defaults to env[NOVA_ENDPOINT_TYPE] or
                                publicURL.
  --os-compute-api-version <compute-api-ver>
                                Accepts 1.1 or 3, defaults to
                                env[OS_COMPUTE_API_VERSION].
  --os-cacert <ca-certificate>  Specify a CA bundle file to use in verifying a
                                TLS (https) server certificate. Defaults to
                                env[OS_CACERT]
  --insecure                    Explicitly allow novaclient to perform
                                "insecure" SSL (https) requests. The server's
                                certificate will not be verified against any
                                certificate authorities. This option should be
                                used with caution.
  --bypass-url <bypass-url>     Use this API endpoint instead of the Service
                                Catalog. Defaults to
                                env[NOVACLIENT_BYPASS_URL]

See "nova help COMMAND" for help on a specific command.

--------------------------------------------------------------------------------------
\end{lstlisting}

\subsection{neutron命令大全}
\begin{lstlisting}
--------------------------------------------------------------------------------------
usage: neutron [--version] [-v] [-q] [-h] [-r NUM]
               [--os-service-type <os-service-type>]
               [--os-endpoint-type <os-endpoint-type>]
               [--service-type <service-type>]
               [--endpoint-type <endpoint-type>]
               [--os-auth-strategy <auth-strategy>] [--os-auth-url <auth-url>]
               [--os-tenant-name <auth-tenant-name> | --os-project-name <auth-project-name>]
               [--os-tenant-id <auth-tenant-id> | --os-project-id <auth-project-id>]
               [--os-username <auth-username>] [--os-user-id <auth-user-id>]
               [--os-user-domain-id <auth-user-domain-id>]
               [--os-user-domain-name <auth-user-domain-name>]
               [--os-project-domain-id <auth-project-domain-id>]
               [--os-project-domain-name <auth-project-domain-name>]
               [--os-cert <certificate>] [--os-cacert <ca-certificate>]
               [--os-key <key>] [--os-password <auth-password>]
               [--os-region-name <auth-region-name>] [--os-token <token>]
               [--http-timeout <seconds>] [--os-url <url>] [--insecure]

Command-line interface to the Neutron APIs

optional arguments:
  --version             show program's version number and exit
  -v, --verbose, --debug
                        Increase verbosity of output and show tracebacks on
                        errors. You can repeat this option.
  -q, --quiet           Suppress output except warnings and errors.
  -h, --help            Show this help message and exit.
  -r NUM, --retries NUM
                        How many times the request to the Neutron server
                        should be retried if it fails.
  --os-service-type <os-service-type>
                        Defaults to env[OS_NETWORK_SERVICE_TYPE] or network.
  --os-endpoint-type <os-endpoint-type>
                        Defaults to env[OS_ENDPOINT_TYPE] or publicURL.
  --service-type <service-type>
                        DEPRECATED! Use --os-service-type.
  --endpoint-type <endpoint-type>
                        DEPRECATED! Use --os-endpoint-type.
  --os-auth-strategy <auth-strategy>
                        DEPRECATED! Only keystone is supported.
  --os-auth-url <auth-url>
                        Authentication URL, defaults to env[OS_AUTH_URL].
  --os-tenant-name <auth-tenant-name>
                        Authentication tenant name, defaults to
                        env[OS_TENANT_NAME].
  --os-project-name <auth-project-name>
                        Another way to specify tenant name. This option is
                        mutually exclusive with --os-tenant-name. Defaults to
                        env[OS_PROJECT_NAME].
  --os-tenant-id <auth-tenant-id>
                        Authentication tenant ID, defaults to
                        env[OS_TENANT_ID].
  --os-project-id <auth-project-id>
                        Another way to specify tenant ID. This option is
                        mutually exclusive with --os-tenant-id. Defaults to
                        env[OS_PROJECT_ID].
  --os-username <auth-username>
                        Authentication username, defaults to env[OS_USERNAME].
  --os-user-id <auth-user-id>
                        Authentication user ID (Env: OS_USER_ID)
  --os-user-domain-id <auth-user-domain-id>
                        OpenStack user domain ID. Defaults to
                        env[OS_USER_DOMAIN_ID].
  --os-user-domain-name <auth-user-domain-name>
                        OpenStack user domain name. Defaults to
                        env[OS_USER_DOMAIN_NAME].
  --os-project-domain-id <auth-project-domain-id>
                        Defaults to env[OS_PROJECT_DOMAIN_ID].
  --os-project-domain-name <auth-project-domain-name>
                        Defaults to env[OS_PROJECT_DOMAIN_NAME].
  --os-cert <certificate>
                        Path of certificate file to use in SSL connection.
                        This file can optionally be prepended with the private
                        key. Defaults to env[OS_CERT].
  --os-cacert <ca-certificate>
                        Specify a CA bundle file to use in verifying a TLS
                        (https) server certificate. Defaults to
                        env[OS_CACERT].
  --os-key <key>        Path of client key to use in SSL connection. This
                        option is not necessary if your key is prepended to
                        your certificate file. Defaults to env[OS_KEY].
  --os-password <auth-password>
                        Authentication password, defaults to env[OS_PASSWORD].
  --os-region-name <auth-region-name>
                        Authentication region name, defaults to
                        env[OS_REGION_NAME].
  --os-token <token>    Authentication token, defaults to env[OS_TOKEN].
  --http-timeout <seconds>
                        Timeout in seconds to wait for an HTTP response.
                        Defaults to env[OS_NETWORK_TIMEOUT] or None if not
                        specified.
  --os-url <url>        Defaults to env[OS_URL].
  --insecure            Explicitly allow neutronclient to perform "insecure"
                        SSL (https) requests. The server's certificate will
                        not be verified against any certificate authorities.
                        This option should be used with caution.

Commands for API v2.0:
  agent-delete                   Delete a given agent.
  agent-list                     List agents.
  agent-show                     Show information of a given agent.
  agent-update                   Update a given agent.
  cisco-credential-create        Creates a credential.
  cisco-credential-delete        Delete a  given credential.
  cisco-credential-list          List credentials that belong to a given tenant.
  cisco-credential-show          Show information of a given credential.
  cisco-network-profile-create   Creates a network profile.
  cisco-network-profile-delete   Delete a given network profile.
  cisco-network-profile-list     List network profiles that belong to a given tenant.
  cisco-network-profile-show     Show information of a given network profile.
  cisco-network-profile-update   Update network profile's information.
  cisco-policy-profile-list      List policy profiles that belong to a given tenant.
  cisco-policy-profile-show      Show information of a given policy profile.
  cisco-policy-profile-update    Update policy profile's information.
  complete                       print bash completion command
  dhcp-agent-list-hosting-net    List DHCP agents hosting a network.
  dhcp-agent-network-add         Add a network to a DHCP agent.
  dhcp-agent-network-remove      Remove a network from a DHCP agent.
  ext-list                       List all extensions.
  ext-show                       Show information of a given resource.
  firewall-create                Create a firewall.
  firewall-delete                Delete a given firewall.
  firewall-list                  List firewalls that belong to a given tenant.
  firewall-policy-create         Create a firewall policy.
  firewall-policy-delete         Delete a given firewall policy.
  firewall-policy-insert-rule    Insert a rule into a given firewall policy.
  firewall-policy-list           List firewall policies that belong to a given tenant.
  firewall-policy-remove-rule    Remove a rule from a given firewall policy.
  firewall-policy-show           Show information of a given firewall policy.
  firewall-policy-update         Update a given firewall policy.
  firewall-rule-create           Create a firewall rule.
  firewall-rule-delete           Delete a given firewall rule.
  firewall-rule-list             List firewall rules that belong to a given tenant.
  firewall-rule-show             Show information of a given firewall rule.
  firewall-rule-update           Update a given firewall rule.
  firewall-show                  Show information of a given firewall.
  firewall-update                Update a given firewall.
  floatingip-associate           Create a mapping between a floating IP and a fixed IP.
  floatingip-create              Create a floating IP for a given tenant.
  floatingip-delete              Delete a given floating IP.
  floatingip-disassociate        Remove a mapping from a floating IP to a fixed IP.
  floatingip-list                List floating IPs that belong to a given tenant.
  floatingip-show                Show information of a given floating IP.
  gateway-device-create          Create a network gateway device.
  gateway-device-delete          Delete a given network gateway device.
  gateway-device-list            List network gateway devices for a given tenant.
  gateway-device-show            Show information for a given network gateway device.
  gateway-device-update          Update a network gateway device.
  help                           print detailed help for another command
  ipsec-site-connection-create   Create an IPsec site connection.
  ipsec-site-connection-delete   Delete a given IPsec site connection.
  ipsec-site-connection-list     List IPsec site connections that belong to a given tenant.
  ipsec-site-connection-show     Show information of a given IPsec site connection.
  ipsec-site-connection-update   Update a given IPsec site connection.
  l3-agent-list-hosting-router   List L3 agents hosting a router.
  l3-agent-router-add            Add a router to a L3 agent.
  l3-agent-router-remove         Remove a router from a L3 agent.
  lb-agent-hosting-pool          Get loadbalancer agent hosting a pool.
  lb-healthmonitor-associate     Create a mapping between a health monitor and a pool.
  lb-healthmonitor-create        Create a healthmonitor.
  lb-healthmonitor-delete        Delete a given healthmonitor.
  lb-healthmonitor-disassociate  Remove a mapping from a health monitor to a pool.
  lb-healthmonitor-list          List healthmonitors that belong to a given tenant.
  lb-healthmonitor-show          Show information of a given healthmonitor.
  lb-healthmonitor-update        Update a given healthmonitor.
  lb-member-create               Create a member.
  lb-member-delete               Delete a given member.
  lb-member-list                 List members that belong to a given tenant.
  lb-member-show                 Show information of a given member.
  lb-member-update               Update a given member.
  lb-pool-create                 Create a pool.
  lb-pool-delete                 Delete a given pool.
  lb-pool-list                   List pools that belong to a given tenant.
  lb-pool-list-on-agent          List the pools on a loadbalancer agent.
  lb-pool-show                   Show information of a given pool.
  lb-pool-stats                  Retrieve stats for a given pool.
  lb-pool-update                 Update a given pool.
  lb-vip-create                  Create a vip.
  lb-vip-delete                  Delete a given vip.
  lb-vip-list                    List vips that belong to a given tenant.
  lb-vip-show                    Show information of a given vip.
  lb-vip-update                  Update a given vip.
  meter-label-create             Create a metering label for a given tenant.
  meter-label-delete             Delete a given metering label.
  meter-label-list               List metering labels that belong to a given tenant.
  meter-label-rule-create        Create a metering label rule for a given label.
  meter-label-rule-delete        Delete a given metering label.
  meter-label-rule-list          List metering labels that belong to a given label.
  meter-label-rule-show          Show information of a given metering label rule.
  meter-label-show               Show information of a given metering label.
  nec-packet-filter-create       Create a packet filter for a given tenant.
  nec-packet-filter-delete       Delete a given packet filter.
  nec-packet-filter-list         List packet filters that belong to a given tenant.
  nec-packet-filter-show         Show information of a given packet filter.
  nec-packet-filter-update       Update packet filter's information.
  net-create                     Create a network for a given tenant.
  net-delete                     Delete a given network.
  net-external-list              List external networks that belong to a given tenant.
  net-gateway-connect            Add an internal network interface to a router.
  net-gateway-create             Create a network gateway.
  net-gateway-delete             Delete a given network gateway.
  net-gateway-disconnect         Remove a network from a network gateway.
  net-gateway-list               List network gateways for a given tenant.
  net-gateway-show               Show information of a given network gateway.
  net-gateway-update             Update the name for a network gateway.
  net-list                       List networks that belong to a given tenant.
  net-list-on-dhcp-agent         List the networks on a DHCP agent.
  net-show                       Show information of a given network.
  net-update                     Update network's information.
  nuage-netpartition-create      Create a netpartition for a given tenant.
  nuage-netpartition-delete      Delete a given netpartition.
  nuage-netpartition-list        List netpartitions that belong to a given tenant.
  nuage-netpartition-show        Show information of a given netpartition.
  port-create                    Create a port for a given tenant.
  port-delete                    Delete a given port.
  port-list                      List ports that belong to a given tenant.
  port-show                      Show information of a given port.
  port-update                    Update port's information.
  queue-create                   Create a queue.
  queue-delete                   Delete a given queue.
  queue-list                     List queues that belong to a given tenant.
  queue-show                     Show information of a given queue.
  quota-delete                   Delete defined quotas of a given tenant.
  quota-list                     List quotas of all tenants who have non-default quota values.
  quota-show                     Show quotas of a given tenant.
  quota-update                   Define tenant's quotas not to use defaults.
  router-create                  Create a router for a given tenant.
  router-delete                  Delete a given router.
  router-gateway-clear           Remove an external network gateway from a router.
  router-gateway-set             Set the external network gateway for a router.
  router-interface-add           Add an internal network interface to a router.
  router-interface-delete        Remove an internal network interface from a router.
  router-list                    List routers that belong to a given tenant.
  router-list-on-l3-agent        List the routers on a L3 agent.
  router-port-list               List ports that belong to a given tenant, with specified router.
  router-show                    Show information of a given router.
  router-update                  Update router's information.
  security-group-create          Create a security group.
  security-group-delete          Delete a given security group.
  security-group-list            List security groups that belong to a given tenant.
  security-group-rule-create     Create a security group rule.
  security-group-rule-delete     Delete a given security group rule.
  security-group-rule-list       List security group rules that belong to a given tenant.
  security-group-rule-show       Show information of a given security group rule.
  security-group-show            Show information of a given security group.
  security-group-update          Update a given security group.
  service-provider-list          List service providers.
  subnet-create                  Create a subnet for a given tenant.
  subnet-delete                  Delete a given subnet.
  subnet-list                    List subnets that belong to a given tenant.
  subnet-show                    Show information of a given subnet.
  subnet-update                  Update subnet's information.
  vpn-ikepolicy-create           Create an IKE policy.
  vpn-ikepolicy-delete           Delete a given IKE policy.
  vpn-ikepolicy-list             List IKE policies that belong to a tenant.
  vpn-ikepolicy-show             Show information of a given IKE policy.
  vpn-ikepolicy-update           Update a given IKE policy.
  vpn-ipsecpolicy-create         Create an IPsec policy.
  vpn-ipsecpolicy-delete         Delete a given IPsec policy.
  vpn-ipsecpolicy-list           List ipsecpolicies that belongs to a given tenant connection.
  vpn-ipsecpolicy-show           Show information of a given IPsec policy.
  vpn-ipsecpolicy-update         Update a given IPsec policy.
  vpn-service-create             Create a VPN service.
  vpn-service-delete             Delete a given VPN service.
  vpn-service-list               List VPN service configurations that belong to a given tenant.
  vpn-service-show               Show information of a given VPN service.
  vpn-service-update             Update a given VPN service.
--------------------------------------------------------------------------------------
\end{lstlisting}

\subsection{一些常用命令}
\begin{lstlisting}
-------------------------------------------------------------------------------------------
创建镜像:
glance image-create --file --name --disk-format --container-format --is-public --progress
创建配置模板:
nova flavor-create <name> <id> <ram内存,单位mb> <disk外存,单位GB> <vcpus>
创建密钥对:
ssh-keygen
nova keypair-add --pub-key 公钥路径 名字
创建安全组:
nova secgroup-create name
nova secgroup-add-rule groupname tcp/udp/icmp  起始port  结束port  cidr(0.0.0.0/0代表任意in的访问) 
nova secgroup-add-rule default icmp -1 -1 0.0.0.0/0
nova secgroup-add-rule default tcp 22 22 0.0.0.0/0
创建网络:
neutron net-create ext-net --shared --router:external True \
--provider:physical_network external --provier:network_type flat
neutron subnet-create ext-net --name ext-subnet \
--allocation-pool start=FLOATING_IP_START,end=FLOATING_IP_END  \
--disable-dhcp --gateway EXTERNAL_NETWORK_GATEWAY EXTERNAL_NETWORK_CIDR

neutron net-create demo-net
neutron subnet-create demo-net --name demo-subnet \
--gateway EXTERNAL_NETWORK_GATEWAY EXTERNAL_NETWORK_CIDR
创建路由并绑定端口:
neutron router-create demo-router
neutron router-interface-add demo-router demo-subnet
neutron router-gateway-set demo-router ext-net
创建浮动ip:
neutron floatingip-create ext-net
nova floating-ip-associate demo-instance1 ip地址
创建虚拟机:
nova boot --flavor --image  --nic net-id=netid --key-name --security-group 虚拟机名字
-------------------------------------------------------------------------------------------
\end{lstlisting}
\end{document}