% !TeX spellcheck = en_US
%% 字体:方正静蕾简体
%%		 方正粗宋
\documentclass[a4paper,left=1.5cm,right=1.5cm,11pt]{article}
\title{ubuntu虚拟网卡配置}
\author{周威光整理\footnote{简介:恒天云FTE}}
\date{2017-06-22} 
% 引用宏包
\usepackage[utf8]{inputenc}
\usepackage{fontspec}
\usepackage{cite}
\usepackage{xeCJK}
\usepackage{indentfirst}
\usepackage{titlesec}
\usepackage{etoolbox}%
\makeatletter
\patchcmd{\ttlh@hang}{\parindent\z@}{\parindent\z@\leavevmode}{}{}%
\patchcmd{\ttlh@hang}{\noindent}{}{}{}%
\makeatother

\usepackage{longtable}
\usepackage{empheq}
\usepackage{graphicx}
\usepackage{float}
\usepackage{rotating}
\usepackage{subfigure}
\usepackage{tabu}
\usepackage{amsmath}
\usepackage{setspace}
\usepackage{amsfonts}
\usepackage{appendix}
\usepackage{listings}
\usepackage{xcolor}
\usepackage{geometry}
\setcounter{secnumdepth}{4}
%\titleformat*{\section}{\LARGE}
%\renewcommand\refname{参考文献}
%\titleformat{\chapter}{\centering\bfseries\huge}{}{0.7em}{}{}
\titleformat{\section}{\LARGE\bf}{\thesection}{1em}{}{}
\titleformat{\subsection}{\Large\bfseries}{\thesubsection}{1em}{}{}
\titleformat{\subsubsection}{\large\bfseries}{\thesubsubsection}{1em}{}{}
\renewcommand{\contentsname}{{ \centerline{目{  } 录}}}
\setCJKfamilyfont{cjkhwxk}{STXINGKA.TTF}
%\setCJKfamilyfont{cjkhwxk}{华文行楷}
%\setCJKfamilyfont{cjkfzcs}{方正粗宋简体}
%\newcommand*{\cjkfzcs}{\CJKfamily{cjkfzcs}}
\newcommand*{\cjkhwxk}{\CJKfamily{cjkhwxk}}
%\newfontfamily\wryh{Microsoft YaHei}
%\newfontfamily\hwzs{华文中宋}
%\newfontfamily\hwst{华文宋体}
%\newfontfamily\hwfs{华文仿宋}
%\newfontfamily\jljt{方正静蕾简体}
%\newfontfamily\hwxk{华文行楷}
\newcommand{\verylarge}{\fontsize{60pt}{\baselineskip}\selectfont}  
\newcommand{\chuhao}{\fontsize{44.9pt}{\baselineskip}\selectfont}  
\newcommand{\xiaochu}{\fontsize{38.5pt}{\baselineskip}\selectfont}  
\newcommand{\yihao}{\fontsize{27.8pt}{\baselineskip}\selectfont}  
\newcommand{\xiaoyi}{\fontsize{25.7pt}{\baselineskip}\selectfont}  
\newcommand{\erhao}{\fontsize{23.5pt}{\baselineskip}\selectfont}  
\newcommand{\xiaoerhao}{\fontsize{19.3pt}{\baselineskip}\selectfont} 
\newcommand{\sihao}{\fontsize{14pt}{\baselineskip}\selectfont}      % 字号设置  
\newcommand{\xiaosihao}{\fontsize{12pt}{\baselineskip}\selectfont}  % 字号设置  
\newcommand{\wuhao}{\fontsize{10.5pt}{\baselineskip}\selectfont}    % 字号设置  
\newcommand{\xiaowuhao}{\fontsize{9pt}{\baselineskip}\selectfont}   % 字号设置  
\newcommand{\liuhao}{\fontsize{7.875pt}{\baselineskip}\selectfont}  % 字号设置  
\newcommand{\qihao}{\fontsize{5.25pt}{\baselineskip}\selectfont}    % 字号设置 

\usepackage{diagbox}
\usepackage{multirow}
\boldmath
\XeTeXlinebreaklocale "zh"
\XeTeXlinebreakskip = 0pt plus 1pt minus 0.1pt
\definecolor{cred}{rgb}{0.8,0.8,0.8}
\definecolor{cgreen}{rgb}{0,0.3,0}
\definecolor{cpurple}{rgb}{0.5,0,0.35}
\definecolor{cdocblue}{rgb}{0,0,0.3}
\definecolor{cdark}{rgb}{0.95,1.0,1.0}
\lstset{
	language=bash,
	numbers=left,
	numberstyle=\tiny\color{black},
	showspaces=false,
	showstringspaces=false,
	basicstyle=\scriptsize,
	keywordstyle=\color{purple},
	commentstyle=\itshape\color{cgreen},
	stringstyle=\color{blue},
	frame=lines,
	% escapeinside=``,
	extendedchars=true, 
	xleftmargin=1em,
	xrightmargin=1em, 
	backgroundcolor=\color{cred},
	aboveskip=1em,
	breaklines=true,
	tabsize=4
} 

%\newfontfamily{\consolas}{Consolas}
%\newfontfamily{\monaco}{Monaco}
%\setmonofont[Mapping={}]{Consolas}	%英文引号之类的正常显示,相当于设置英文字体
%\setsansfont{Consolas} %设置英文字体 Monaco, Consolas,  Fantasque Sans Mono
%\setmainfont{Times New Roman}
%\setCJKmainfont{STZHONGS.TTF}
%\setmonofont{Consolas}
% \newfontfamily{\consolas}{YaHeiConsolas.ttf}
\newfontfamily{\monaco}{MONACO.TTF}
\setCJKmainfont{STZHONGS.TTF}
%\setmainfont{MONACO.TTF}
%\setsansfont{MONACO.TTF}
% 自定义添加图片命令
\newcommand{\fic}[1]{\begin{figure}[H]
		\center
		\includegraphics[width=0.8\textwidth]{#1}
	\end{figure}}
	
\newcommand{\sizedfic}[2]{\begin{figure}[H]
		\center
		\includegraphics[width=#1\textwidth]{#2}
	\end{figure}}
% 
\newcommand{\codefile}[1]{\lstinputlisting{#1}}

\newcommand{\interval}{\vspace{0.5em}}

\newcommand{\tablestart}{
	\interval
	\begin{longtable}{p{2cm}p{10cm}}
	\hline}
\newcommand{\tableend}{
	\hline
	\end{longtable}
	\interval}

% 改变段间隔
\setlength{\parskip}{0.2em}
\linespread{1.1}

\usepackage{lastpage}
% 设置页眉页脚
\usepackage{fancyhdr}
\pagestyle{fancy}
\lhead{\space \qquad \space}
\chead{ubuntu虚拟网卡配置\qquad}
\rhead{\qquad\thepage/\pageref{LastPage}}

% 参考文献
\def\hang{\hangindent\parindent}
\def\textindent#1{\indent\llap{#1\enspace}\ignorespaces}
\def\re{\par\hang\textindent}
%超链接
\usepackage[CJKbookmarks=true,
            bookmarksnumbered=true,
            bookmarksopen=true,
            colorlinks, %注释掉此项则交叉引用为彩色边框(将colorlinks和pdfborder同时注释掉)
            pdfborder=001,   %注释掉此项则交叉引用为彩色边框
            linkcolor=green,
            anchorcolor=green,
            citecolor=green
            ]{hyperref}  

\begin{document}
\maketitle
\clearpage
\tableofcontents
\clearpage
\section{网卡类型介绍}
eth0 eth0:1 和eth0.1三者的关系对应于物理网卡、子网卡、虚拟VLAN网卡的关系\\
\begin{itemize}
	\item[1.]物理网卡:物理网卡这里指的是服务器上实际的网络接口设备,这里我服务器上双网卡,在系统中看到的2个物理网卡分别对应是eth0和eth1这两个网络接口。
	\item[2.]子网卡在这里并不是实际上的网络接口设备,但是可以作为网络接口在系统中出现,如eth0:1、eth1:2这种网络接口。
	它们必须要依赖于物理网卡,虽然可以与物理网卡的网络接口同时在系统中存在并使用不同的IP地址,而且也拥有它们自己的网络接口配置文件。
	但是当所依赖的物理网卡不启用时(Down状态)这些子网卡也将一同不能工作。
	\item[3.]虚拟VLAN网卡:这些虚拟VLAN网卡也不是实际上的网络接口设备,也可以作为网络接口在系统中出现,但是与子网卡不同的是,
	他们没有自己的配置文件。他们只是通过将物理网加入不同的VLAN而生成的VLAN虚拟网卡。如果将一个物理网卡通过vconfig命令添加到多个VLAN当中去的话,
	就会有多个VLAN虚拟网卡出现,他们的信息以及相关的VLAN信息都是保存在/proc/net/vlan/config这个临时文件中的,而没有独自的配置文件。
	它们的网络接口名是eth0.1、eth1.2这种名字。
\end{itemize}
注意:当需要启用VLAN虚拟网卡工作的时候,关联的物理网卡网络接口上必须没有IP地址的配置信息,并且,这些主物理网卡的子网卡也必须不能被启用和必须不能有IP地址配置信息。
这个在网上看到的结论根据我的实际测试结果来看是不准确的,物理网卡本身可以绑定IP,并且给本征vlan提供通信网关的功能,但必须是在802.1q下。
\section{Linux添加虚拟网卡的方法}
\subsection{子网卡}
\begin{itemize}
	\item[1.]命令行添加\\
	创建虚拟网卡
	\begin{lstlisting}
	sudo ifconfig eth0:0 192.168.10.10 netmask 255.255.255.0 up
	\end{lstlisting}
	关闭虚拟网卡
	\begin{lstlisting}
	sudo ifconfig eth0:0 down
	\end{lstlisting}
	缺点:重启服务器,网卡配置会丢失
	\item[2.]配置文件中/etc/network/interfaces添加
	\begin{lstlisting}
	auto eth0:0
	iface eth0:0 inet static
	address 192.168.10.10
	netmask 255.255.255.0
	#network 192.168.10.1
	#broadcast 192.168.1.255
	\end{lstlisting}
	启动网卡
	\begin{lstlisting}
	ifup eth0:0
	\end{lstlisting}
	优点:重启服务器,网卡配置不会丢失\\
	配置子网卡的主物理网卡必须有ip,否则ping不通
\end{itemize}
\subsection{vlan虚拟网卡}
\begin{itemize}
	\item[1.]命令行添加vlan虚拟网卡\\
	\begin{lstlisting}
	#添加vlan
	ip link add link eth0 name eth0.100 type vlan id 100
	#查看虚拟vlan网卡信息
	ip -d link show eth0.100
	#添加ip地址
	ip addr add 192.168.100.1/24 dev eth0.100
	#启动虚拟网卡
	ip link set dev eth0.100 up
	#关闭虚拟网卡
	ip link set dev eth0.100 down
	#删除虚拟网卡
	ip link delete eth0.100	
	\end{lstlisting}
	缺点:重启服务器,网卡配置会丢失(可将命令添加到/etc/rc.local中,开机自动运行)
	\item[2.]物理网卡配置为vlan模式
	前提条件
	\begin{lstlisting}
	#开启内核支持
	modprobe 8021q
	echo "8021q" >>/etc/modules
	#安装配置工具
	aptitude install vlan
	#增加虚拟接口
	vconfig add eth1 30
	\end{lstlisting}
	配置文件中配置
	\begin{lstlisting}
	auto eth1.30
	iface eth1.30 inet static
    address 10.20.30.19
    netmask 255.255.255.0
    network 10.20.30.0
    broadcast 10.20.30.255
	\end{lstlisting}
	启动网卡
	\begin{lstlisting}
	ifup eth1.30
	\end{lstlisting}
	优点:重启服务器,网卡配置不会丢失
\end{itemize}
 
\clearpage
\begin{center}%参考文献的书写
参考文献
\end{center}
\re{[1]} \href{http://blog.csdn.net/hzhsan/article/details/44677867}{Linux添加虚拟网卡的多种方法}  
\re{[2]} \href{http://www.anrip.com/post/1766}{Ubuntu/Linux 配置 Vlan}
\re{[3]} \href{http://blog.sina.com.cn/s/blog_54559518010123u9.html}{Ubuntu Linux中VLAN的配置}
\re{[4]} \href{https://blog.blacate.me/2017/01/12/linux-vlan/}{linux下vlan的学习}
\re{[5]} \href{http://www.cnblogs.com/JohnABC/p/5951340.html}{Linux-eth0 eth0:1 和eth0.1关系、ifconfig以及虚拟IP实现介绍}

\end{document}