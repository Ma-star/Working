% !TeX spellcheck = en_US
%% 字体:方正静蕾简体
%%		 方正粗宋
\documentclass[a4paper,left=1.5cm,right=1.5cm,11pt]{article}
\title{Openstack之SDN性能测试方案}
\author{周威光整理}
\date{2017-06-26} 
% 引用宏包
\usepackage[utf8]{inputenc}
\usepackage{fontspec}
\usepackage{cite}
\usepackage{xeCJK}
\usepackage{indentfirst}
\usepackage{titlesec}
\usepackage{etoolbox}%
\makeatletter
\patchcmd{\ttlh@hang}{\parindent\z@}{\parindent\z@\leavevmode}{}{}%
\patchcmd{\ttlh@hang}{\noindent}{}{}{}%
\makeatother

\usepackage{longtable}
\usepackage{empheq}
\usepackage{graphicx}
\usepackage{float}
\usepackage{rotating}
\usepackage{subfigure}
\usepackage{tabu}
\usepackage{amsmath}
\usepackage{setspace}
\usepackage{amsfonts}
\usepackage{appendix}
\usepackage{listings}
\usepackage{xcolor}
\usepackage{geometry}
\setcounter{secnumdepth}{4}
%\titleformat*{\section}{\LARGE}
%\renewcommand\refname{参考文献}
%\titleformat{\chapter}{\centering\bfseries\huge}{}{0.7em}{}{}
\titleformat{\section}{\LARGE\bf}{\thesection}{1em}{}{}
\titleformat{\subsection}{\Large\bfseries}{\thesubsection}{1em}{}{}
\titleformat{\subsubsection}{\large\bfseries}{\thesubsubsection}{1em}{}{}
\renewcommand{\contentsname}{{ \centerline{目{  } 录}}}
\setCJKfamilyfont{cjkhwxk}{STXINGKA.TTF}
%\setCJKfamilyfont{cjkhwxk}{华文行楷}
%\setCJKfamilyfont{cjkfzcs}{方正粗宋简体}
%\newcommand*{\cjkfzcs}{\CJKfamily{cjkfzcs}}
\newcommand*{\cjkhwxk}{\CJKfamily{cjkhwxk}}
%\newfontfamily\wryh{Microsoft YaHei}
%\newfontfamily\hwzs{华文中宋}
%\newfontfamily\hwst{华文宋体}
%\newfontfamily\hwfs{华文仿宋}
%\newfontfamily\jljt{方正静蕾简体}
%\newfontfamily\hwxk{华文行楷}
\newcommand{\verylarge}{\fontsize{60pt}{\baselineskip}\selectfont}  
\newcommand{\chuhao}{\fontsize{44.9pt}{\baselineskip}\selectfont}  
\newcommand{\xiaochu}{\fontsize{38.5pt}{\baselineskip}\selectfont}  
\newcommand{\yihao}{\fontsize{27.8pt}{\baselineskip}\selectfont}  
\newcommand{\xiaoyi}{\fontsize{25.7pt}{\baselineskip}\selectfont}  
\newcommand{\erhao}{\fontsize{23.5pt}{\baselineskip}\selectfont}  
\newcommand{\xiaoerhao}{\fontsize{19.3pt}{\baselineskip}\selectfont} 
\newcommand{\sihao}{\fontsize{14pt}{\baselineskip}\selectfont}      % 字号设置  
\newcommand{\xiaosihao}{\fontsize{12pt}{\baselineskip}\selectfont}  % 字号设置  
\newcommand{\wuhao}{\fontsize{10.5pt}{\baselineskip}\selectfont}    % 字号设置  
\newcommand{\xiaowuhao}{\fontsize{9pt}{\baselineskip}\selectfont}   % 字号设置  
\newcommand{\liuhao}{\fontsize{7.875pt}{\baselineskip}\selectfont}  % 字号设置  
\newcommand{\qihao}{\fontsize{5.25pt}{\baselineskip}\selectfont}    % 字号设置 

\usepackage{diagbox}
\usepackage{multirow}
\boldmath
\XeTeXlinebreaklocale "zh"
\XeTeXlinebreakskip = 0pt plus 1pt minus 0.1pt
\definecolor{cred}{rgb}{0.8,0.8,0.8}
\definecolor{cgreen}{rgb}{0,0.3,0}
\definecolor{cpurple}{rgb}{0.5,0,0.35}
\definecolor{cdocblue}{rgb}{0,0,0.3}
\definecolor{cdark}{rgb}{0.95,1.0,1.0}
\lstset{
	language=bash,
	numbers=left,
	numberstyle=\tiny\color{black},
	showspaces=false,
	showstringspaces=false,
	basicstyle=\scriptsize,
	keywordstyle=\color{purple},
	commentstyle=\itshape\color{cgreen},
	stringstyle=\color{blue},
	frame=lines,
	% escapeinside=``,
	extendedchars=true, 
	xleftmargin=1em,
	xrightmargin=1em, 
	backgroundcolor=\color{cred},
	aboveskip=1em,
	breaklines=true,
	tabsize=4
} 

%\newfontfamily{\consolas}{Consolas}
%\newfontfamily{\monaco}{Monaco}
%\setmonofont[Mapping={}]{Consolas}	%英文引号之类的正常显示,相当于设置英文字体
%\setsansfont{Consolas} %设置英文字体 Monaco, Consolas,  Fantasque Sans Mono
%\setmainfont{Times New Roman}
%\setCJKmainfont{STZHONGS.TTF}
%\setmonofont{Consolas}
% \newfontfamily{\consolas}{YaHeiConsolas.ttf}
\newfontfamily{\monaco}{MONACO.TTF}
\setCJKmainfont{STZHONGS.TTF}
%\setmainfont{MONACO.TTF}
%\setsansfont{MONACO.TTF}
% 自定义添加图片命令
\newcommand{\fic}[1]{\begin{figure}[H]
		\center
		\includegraphics[width=0.8\textwidth]{#1}
	\end{figure}}
	
\newcommand{\sizedfic}[2]{\begin{figure}[H]
		\center
		\includegraphics[width=#1\textwidth]{#2}
	\end{figure}}
% 
\newcommand{\codefile}[1]{\lstinputlisting{#1}}

\newcommand{\interval}{\vspace{0.5em}}

\newcommand{\tablestart}{
	\interval
	\begin{longtable}{p{2cm}p{10cm}}
	\hline}
\newcommand{\tableend}{
	\hline
	\end{longtable}
	\interval}

% 改变段间隔
\setlength{\parskip}{0.2em}
\linespread{1.1}

\usepackage{lastpage}
% 设置页眉页脚
\usepackage{fancyhdr}
\pagestyle{fancy}
\lhead{\space \qquad \space}
\chead{Openstack之SDN性能测试方案\qquad}
\rhead{\qquad\thepage/\pageref{LastPage}}

% 参考文献
\def\hang{\hangindent\parindent}
\def\textindent#1{\indent\llap{#1\enspace}\ignorespaces}
\def\re{\par\hang\textindent}

\begin{document}
\maketitle
\clearpage
\tableofcontents
\clearpage
\section{Neutron network划分}
\subsection{根据创建网络的用户权限划分}
\begin{itemize}
	\item[(1).]Provider network:管理员创建的和物理网络有直接映射关系的虚拟网络;
	\item[(2).]Tenant network:租户普通用户创建的网络,物理网络对创建者透明,其配置由 Neutron根据管理员在系统中的配置决定;
\end{itemize}
\subsection{根据网络类型划分}
\begin{itemize}
	\item[(1).]local network(本地网络):一个只允许在本服务器内通信的虚拟网络,不知道跨服务器的通信。主要用于单节点上测试。
	\item[(2).]flat network:基于不使用 vlan 的物理网络实现的虚拟网络。每个物理网络最多只能实现一个虚拟网络。
	\item[(3).]vlan network(虚拟局域网) :基于物理 vlan 网络实现的虚拟网络。共享同一个物理网络的多个 vlan 网络是相互隔离的,甚至可以使用重叠的 ip 地址空间。
		每个支持 vlan network 的物理网络可以被视为一个分离的 vlan trunk,它使用一组独占的 vlan ID。有效的 vlan ID 范围是 1 到 4094。
	\item[(4).]gre network (通用路由封装网络):一个使用 gre 封装网络包的虚拟网络。gre 封装的数据包基于 IP 路由表来进行路由,因此 gre network 不和具体的物理网络绑定。
	\item[(5).]vxlan network(虚拟可扩展网络):基于 vxlan 实现的虚拟网络。同 gre network 一样, vxlan network 中 ip 包的路由也基于 ip 路由表,也不和具体的物理网络绑定。
\end{itemize}
总结:根据定义可知,local网络主要用在单点测试,不在讨论范围之内。租户网络(Tenant network)包括gre、vxlan,提供者网络(Provider network)包括flat、vlan。

\section{网络模式初步确定}
\begin{itemize}
	\item[(1).]vxlan和gre同时解决了vlan id个数限制和跨机房互通问题,但vxlan避免了gre的点对点必须有连接的缺点,同时实现了大2层网络,可用于vm在机房之间的的无缝迁移。基于这些特点,
	选择vxlan作为租户网络
	\item[(2).]根据以往市场调研情况,对vlan的需求会比较大,选择vlan为提供者网络
\end{itemize}
\begin{center}
\begin{tabular}[c]{|l|l|}
\hline
租户网络 & 提供者网络\\
\hline
vxlan & vlan\\
\hline
\end{tabular}
\end{center}

\section{性能测试目标}
测试目标是两个虚拟机传输数据包时的几个值,包括:数据包传输往返时延(RTT),数据包丢失率(Lost rate),数据包传输时的带宽(Bandwidth)。数据包又分为TCP包、UDP包、ICMP(ping)包。
表格形式如下:
\begin{center}
\begin{tabular}[c]{l|l|l|l}
 & TCP包 & UDP包 & ICMP(ping)包\\
\hline
延迟 & & &\\
\hline
带宽 & & & \\
\hline
丢包 & & & \\
\hline
\end{tabular}
\end{center}
\section{性能测试场景}
\subsection{场景一:同时挂在提供者网络的两个vm之间}
进行如下细分:
\begin{itemize}
	\item[(1).]测试同一租户下,传输数据包的延迟、丢包、带宽
	\item[(2).]测试不同租户下,传输数据包的延迟、丢包、带宽
	\item[(3).]与物理机接相同vlan下,对比vm之间和物理机之间的性能
	\item[(4).]测试vm与物理机之间的数据包传输性能
\end{itemize}
\subsection{场景二:一个vm挂在租户网络,一个vm在提供者网络上}
细分如下:
\begin{itemize}
	\item[(1).]测试同一租户下,传输数据包的延迟、丢包、带宽
	\item[(2).]测试不同租户下,传输数据包的延迟、丢包、带宽
\end{itemize}
\subsection{场景三:同时挂在租户网络且同一网段的两个vm之间}
\subsection{场景四:同时挂在租户网络但不同网段的两个vm之间}

\end{document}