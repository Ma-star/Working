% !TeX spellcheck = en_US
%% 字体:方正静蕾简体
%%		 方正粗宋
\documentclass[a4paper,left=1.5cm,right=1.5cm,11pt]{article}
\title{Openstack之SDN性能测试具体实施及结果}
\author{周威光整理\footnote{简介:恒天云FTE}}
\date{2017-07-21} 
% 引用宏包
\usepackage[utf8]{inputenc}
\usepackage{fontspec}
\usepackage{cite}
\usepackage{xeCJK}
\usepackage{indentfirst}
\usepackage{titlesec}
\usepackage{etoolbox}%
\makeatletter
\patchcmd{\ttlh@hang}{\parindent\z@}{\parindent\z@\leavevmode}{}{}%
\patchcmd{\ttlh@hang}{\noindent}{}{}{}%
\makeatother

\usepackage{longtable}
\usepackage{empheq}
\usepackage{graphicx}
\usepackage{float}
\usepackage{rotating}
\usepackage{subfigure}
\usepackage{tabu}
\usepackage{amsmath}
\usepackage{setspace}
\usepackage{amsfonts}
\usepackage{appendix}
\usepackage{listings}
\usepackage{xcolor}
\usepackage{geometry}
\setcounter{secnumdepth}{4}
%\titleformat*{\section}{\LARGE}
%\renewcommand\refname{参考文献}
%\titleformat{\chapter}{\centering\bfseries\huge}{}{0.7em}{}{}
\titleformat{\section}{\LARGE\bf}{\thesection}{1em}{}{}
\titleformat{\subsection}{\Large\bfseries}{\thesubsection}{1em}{}{}
\titleformat{\subsubsection}{\large\bfseries}{\thesubsubsection}{1em}{}{}
\renewcommand{\contentsname}{{ \centerline{目{  } 录}}}
\setCJKfamilyfont{cjkhwxk}{STXINGKA.TTF}
%\setCJKfamilyfont{cjkhwxk}{华文行楷}
%\setCJKfamilyfont{cjkfzcs}{方正粗宋简体}
%\newcommand*{\cjkfzcs}{\CJKfamily{cjkfzcs}}
\newcommand*{\cjkhwxk}{\CJKfamily{cjkhwxk}}
%\newfontfamily\wryh{Microsoft YaHei}
%\newfontfamily\hwzs{华文中宋}
%\newfontfamily\hwst{华文宋体}
%\newfontfamily\hwfs{华文仿宋}
%\newfontfamily\jljt{方正静蕾简体}
%\newfontfamily\hwxk{华文行楷}
\newcommand{\verylarge}{\fontsize{60pt}{\baselineskip}\selectfont}  
\newcommand{\chuhao}{\fontsize{44.9pt}{\baselineskip}\selectfont}  
\newcommand{\xiaochu}{\fontsize{38.5pt}{\baselineskip}\selectfont}  
\newcommand{\yihao}{\fontsize{27.8pt}{\baselineskip}\selectfont}  
\newcommand{\xiaoyi}{\fontsize{25.7pt}{\baselineskip}\selectfont}  
\newcommand{\erhao}{\fontsize{23.5pt}{\baselineskip}\selectfont}  
\newcommand{\xiaoerhao}{\fontsize{19.3pt}{\baselineskip}\selectfont} 
\newcommand{\sihao}{\fontsize{14pt}{\baselineskip}\selectfont}      % 字号设置  
\newcommand{\xiaosihao}{\fontsize{12pt}{\baselineskip}\selectfont}  % 字号设置  
\newcommand{\wuhao}{\fontsize{10.5pt}{\baselineskip}\selectfont}    % 字号设置  
\newcommand{\xiaowuhao}{\fontsize{9pt}{\baselineskip}\selectfont}   % 字号设置  
\newcommand{\liuhao}{\fontsize{7.875pt}{\baselineskip}\selectfont}  % 字号设置  
\newcommand{\qihao}{\fontsize{5.25pt}{\baselineskip}\selectfont}    % 字号设置 

\usepackage{diagbox}
\usepackage{multirow}
\boldmath
\XeTeXlinebreaklocale "zh"
\XeTeXlinebreakskip = 0pt plus 1pt minus 0.1pt
\definecolor{cred}{rgb}{0.8,0.8,0.8}
\definecolor{cgreen}{rgb}{0,0.3,0}
\definecolor{cpurple}{rgb}{0.5,0,0.35}
\definecolor{cdocblue}{rgb}{0,0,0.3}
\definecolor{cdark}{rgb}{0.95,1.0,1.0}
\lstset{
	language=bash,
	numbers=left,
	numberstyle=\tiny\color{black},
	showspaces=false,
	showstringspaces=false,
	basicstyle=\scriptsize,
	keywordstyle=\color{purple},
	commentstyle=\itshape\color{cgreen},
	stringstyle=\color{blue},
	frame=lines,
	% escapeinside=``,
	extendedchars=true, 
	xleftmargin=1em,
	xrightmargin=1em, 
	backgroundcolor=\color{cred},
	aboveskip=1em,
	breaklines=true,
	tabsize=4
} 

%\newfontfamily{\consolas}{Consolas}
%\newfontfamily{\monaco}{Monaco}
%\setmonofont[Mapping={}]{Consolas}	%英文引号之类的正常显示,相当于设置英文字体
%\setsansfont{Consolas} %设置英文字体 Monaco, Consolas,  Fantasque Sans Mono
%\setmainfont{Times New Roman}
%\setCJKmainfont{STZHONGS.TTF}
%\setmonofont{Consolas}
% \newfontfamily{\consolas}{YaHeiConsolas.ttf}
\newfontfamily{\monaco}{MONACO.TTF}
\setCJKmainfont{STZHONGS.TTF}
%\setmainfont{MONACO.TTF}
%\setsansfont{MONACO.TTF}
% 自定义添加图片命令
\newcommand{\fic}[1]{\begin{figure}[H]
		\center
		\includegraphics[width=0.8\textwidth]{#1}
	\end{figure}}
	
\newcommand{\sizedfic}[2]{\begin{figure}[H]
		\center
		\includegraphics[width=#1\textwidth]{#2}
	\end{figure}}
% 
\newcommand{\codefile}[1]{\lstinputlisting{#1}}

\newcommand{\interval}{\vspace{0.5em}}

\newcommand{\tablestart}{
	\interval
	\begin{longtable}{p{2cm}p{10cm}}
	\hline}
\newcommand{\tableend}{
	\hline
	\end{longtable}
	\interval}

% 改变段间隔
\setlength{\parskip}{0.2em}
\linespread{1.1}

\usepackage{lastpage}
% 设置页眉页脚
\usepackage{fancyhdr}
\pagestyle{fancy}
\lhead{\space \qquad \space}
\chead{Openstack之SDN性能测试具体实施及结果\qquad}
\rhead{\qquad\thepage/\pageref{LastPage}}

% 参考文献
\def\hang{\hangindent\parindent}
\def\textindent#1{\indent\llap{#1\enspace}\ignorespaces}
\def\re{\par\hang\textindent}
%超链接
\usepackage[CJKbookmarks=true,
            bookmarksnumbered=true,
            bookmarksopen=true,
            colorlinks, %注释掉此项则交叉引用为彩色边框(将colorlinks和pdfborder同时注释掉)
            pdfborder=001,   %注释掉此项则交叉引用为彩色边框
            linkcolor=green,
            anchorcolor=green,
            citecolor=green
            ]{hyperref} 
\begin{document}
\maketitle
\clearpage
\tableofcontents
\clearpage
\section{网络性能测试指标}
常见的网络性能测试指标包含:网络吞吐量(Throughput)、网络延迟(latency)、抖动(jitter)、丢包率等
\begin{itemize}
	\item[1.]网络吞吐量:单位时间内通过某个网络(或信道、接口)的数据量,吞吐量受网络的带宽或者网络的额定速率限制的,例如家庭带宽为10M网络,
	表明网络吞吐量不可能超过10Mbits/s,吞吐量的单位通常表示为位元每秒(bit/s或bps)。
	\item[2.]网络延迟:通俗的讲,就是数据从电脑这边传到那边所用的时间。这儿有个问题需要确认,数据是指一个数据包的传输还是任意大小,和你传输的数据量相关。
	可以明显的看到,从A到B传送1个字节的时间和传送100MB的时间肯定是不一样的。标准意义上的延迟,应该仅仅指1个字节的传输时间,类似网络课上讲到的传播时延。
	(不同意见欢迎讨论)。同样存在一个名词叫做传播延时,这个应该可以标识整个数据包的传输时间,不论包大小为多少。
	\item[3.]抖动:用于描述包在网络中的传输延时的变化,抖动越小,说明网络质量越稳定越好。抖动是评价一个网络性能的最重要的因素。
	\item[4.]丢包率:测试中所丢失的数据包数量占所发送的数据包的比率,因为我们知道TCP协议是可靠的,所以,一般在使用UDP传输时,才会统计丢包率。
\end{itemize}
\section{网络性能测试工具选择}
\subsection{常见的网络性能测试工具简介}
常用的开源网络性能测试工具有两个:iperf 和 netperf,iperf是美国伊利诺斯大学(University of Illinois)开发的一种开源的网络性能测试工具,
netperf是由惠普公司开发的一种网络性能的测量工具,测试网络栈。这两种工具都可以测试TCP协议和UDP协议,从可测试的网络性能指标,我们对两种工具进行下对比:
\begin{center}
\begin{tabular}[c]{|l|l|l|l|l|}
\hline
工具 & 带宽 & 网络延迟 & 抖动 & 丢包 \\
\hline
iperf & 是 & 否 & 是 & 是 \\
\hline
netperf & 是 & 是 & 是 & 否 \\
\hline
\end{tabular}
\end{center}\par
可见iperf和netperf都可以完成基本的网络性能测试,但netperf更倾向于测试不同网络模式的数据传输,与本次性能测试需求不符。
而iperf可以经过简单的参数设置,比较直观的给出带宽、抖动和丢包,
并且还能设置测试时间、发送包的大小,以及带宽。虽然无法对网络延迟进行统一测试,但可以使用ping进行弥补
\subsection{测试方法确定}
\begin{center}
\begin{tabular}[c]{|l|l|l|l|l|}
\hline
测试对象 & 带宽 & 延时 & 抖动(Jitter) & 丢包 \\
\hline
测试工具 & iperf tcp包测试 & ping测试 & iperf UDP测试 & iperf UDP测试 \\
\hline
\end{tabular}
\end{center}

\section{网络性能测试工具iperf的使用}
\subsection{使用原理}
使用Iperf测试时必须将一台主机设置为客户端,一台主机设置为服务器。
\subsection{iperf使用方法与参数说明}
\begin{lstlisting}
参数说明
-s 以server模式启动,eg:iperf -s
-c host以client模式启动,host是server端地址,eg:iperf -c 222.35.11.23
 
通用参数
-f [kmKM] 分别表示以Kbits, Mbits, KBytes, MBytes显示报告,默认以Mbits为单位,eg:iperf -c 222.35.11.23 -f K
-i sec 以秒为单位显示报告间隔,eg:iperf -c 222.35.11.23 -i 2
-l 缓冲区大小,默认是8KB,eg:iperf -c 222.35.11.23 -l 16
-m 显示tcp最大mtu值
-o 将报告和错误信息输出到文件eg:iperf -c 222.35.11.23 -o ciperflog.txt
-p 指定服务器端使用的端口或客户端所连接的端口eg:iperf -s -p 9999;iperf -c 222.35.11.23 -p 9999
-u 使用udp协议
-w 指定TCP窗口大小,默认是8KB
-B 绑定一个主机地址或接口(当主机有多个地址或接口时使用该参数)
-C 兼容旧版本(当server端和client端版本不一样时使用)
-M 设定TCP数据包的最大mtu值
-N 设定TCP不延时
-V 传输ipv6数据包
 
server专用参数
-D 以服务方式运行iperf,eg:iperf -s -D
-R 停止iperf服务,针对-D,eg:iperf -s -R
 
client端专用参数
-d 同时进行双向传输测试
-n 指定传输的字节数,eg:iperf -c 222.35.11.23 -n 100000
-r 单独进行双向传输测试
-t 测试时间,默认10秒,eg:iperf -c 222.35.11.23 -t 5
-F 指定需要传输的文件
-T 指定ttl值
\end{lstlisting}
\subsection{tcp包测试带宽}
\begin{lstlisting}
iperf -s #服务器默认等待接收tcp数据包
iperf -c 172.16.133.10 -i 1 #客户端向服务器默认发送tcp数据包
\end{lstlisting}
服务器给出的数据如下:
\sizedfic{0.8}{iperf_tcp.png}
客户端给出的数据如下:
\sizedfic{0.8}{iperf_tcp2.png}
\subsection{udp包同时测试抖动和丢包}
\begin{lstlisting}
 iperf -s -u -i 1 #服务器等待接收udp数据包,每隔一秒钟显示信息
 iperf -c 172.16.133.10 -u -i 1 #客户端向服务器发送udp数据包,每隔一秒钟显示信息
\end{lstlisting}
服务器给出的数据如下:
\sizedfic{0.8}{iperf_udp1.png}
客户端给出的数据如下:
\sizedfic{0.8}{iperf_udp2.png}

\section{ping命令的使用}
\subsection{ping使用方法与参数说明}
\begin{lstlisting}
语法:
ping(选项)(参数)

选项:
-b :后面接的是 broadcast 的 IP,用在你『需要对整个网域的主机进行 ping 』时;
-c :后面接的是运行 ping 的次数,例如 -c 5 ;
-n :不进行 IP 与主机名的反查,直接使用 IP ;
-s :发送出去的 ICMP 封包大小,默认为 56(bytes),再加 8 bytes 的 ICMP 表头数据
-t :TTL 的数值,默认是 255,每经过一个节点就会少一;
-M [do|dont] :主要在侦测网络的 MTU 数值大小,两个常见的项目是:
   do  :代表传送一个 DF (Don't Fragment) 旗标,让封包不能重新拆包与打包;
   dont:代表不要传送 DF 旗标,表示封包可以在其他主机上拆包与打包

参数:
目的主机:指定发送ICMP报文的目的主机。
\end{lstlisting}
\subsection{icmp包测试延迟}
\begin{lstlisting}
ping -c 10 172.16.133.10
\end{lstlisting}
ping服务器显示如下:
\sizedfic{0.8}{ping1.png}
\clearpage
\begin{center}%参考文献的书写
参考文献
\end{center}
\re{[1]} \href{http://www.sdnlab.com/2961.html}{网络性能测试工具Iperf介绍} 
\re{[2]} \href{http://www.51testing.com/html/11/255511-805117.html}{使用iperf测试网络性能}
\re{[3]} \href{http://man.linuxde.net/ping}{ping命令}
\end{document}