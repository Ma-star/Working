% !TeX spellcheck = en_US
%% 字体:方正静蕾简体
%%		 方正粗宋
\documentclass[a4paper,left=1.5cm,right=1.5cm,11pt]{article}
\title{恒天云之SDN性能测试报告}
\author{周威光}
\date{2017-08-2} 
% 引用宏包
\usepackage[utf8]{inputenc}
\usepackage{fontspec}
\usepackage{cite}
\usepackage{xeCJK}
\usepackage{indentfirst}
\usepackage{titlesec}
\usepackage{etoolbox}%
\makeatletter
\patchcmd{\ttlh@hang}{\parindent\z@}{\parindent\z@\leavevmode}{}{}%
\patchcmd{\ttlh@hang}{\noindent}{}{}{}%
\makeatother

\usepackage{longtable}
\usepackage{empheq}
\usepackage{graphicx}
\usepackage{float}
\usepackage{rotating}
\usepackage{subfigure}
\usepackage{tabu}
\usepackage{amsmath}
\usepackage{setspace}
\usepackage{amsfonts}
\usepackage{appendix}
\usepackage{listings}
\usepackage{xcolor}
\usepackage{geometry}
\setcounter{secnumdepth}{4}
%\titleformat*{\section}{\LARGE}
%\renewcommand\refname{参考文献}
%\titleformat{\chapter}{\centering\bfseries\huge}{}{0.7em}{}{}
\titleformat{\section}{\LARGE\bf}{\thesection}{1em}{}{}
\titleformat{\subsection}{\Large\bfseries}{\thesubsection}{1em}{}{}
\titleformat{\subsubsection}{\large\bfseries}{\thesubsubsection}{1em}{}{}
\renewcommand{\contentsname}{{ \centerline{目{  } 录}}}
\setCJKfamilyfont{cjkhwxk}{STXINGKA.TTF}
%\setCJKfamilyfont{cjkhwxk}{华文行楷}
%\setCJKfamilyfont{cjkfzcs}{方正粗宋简体}
%\newcommand*{\cjkfzcs}{\CJKfamily{cjkfzcs}}
\newcommand*{\cjkhwxk}{\CJKfamily{cjkhwxk}}
%\newfontfamily\wryh{Microsoft YaHei}
%\newfontfamily\hwzs{华文中宋}
%\newfontfamily\hwst{华文宋体}
%\newfontfamily\hwfs{华文仿宋}
%\newfontfamily\jljt{方正静蕾简体}
%\newfontfamily\hwxk{华文行楷}
\newcommand{\verylarge}{\fontsize{60pt}{\baselineskip}\selectfont}  
\newcommand{\chuhao}{\fontsize{44.9pt}{\baselineskip}\selectfont}  
\newcommand{\xiaochu}{\fontsize{38.5pt}{\baselineskip}\selectfont}  
\newcommand{\yihao}{\fontsize{27.8pt}{\baselineskip}\selectfont}  
\newcommand{\xiaoyi}{\fontsize{25.7pt}{\baselineskip}\selectfont}  
\newcommand{\erhao}{\fontsize{23.5pt}{\baselineskip}\selectfont}  
\newcommand{\xiaoerhao}{\fontsize{19.3pt}{\baselineskip}\selectfont} 
\newcommand{\sihao}{\fontsize{14pt}{\baselineskip}\selectfont}      % 字号设置  
\newcommand{\xiaosihao}{\fontsize{12pt}{\baselineskip}\selectfont}  % 字号设置  
\newcommand{\wuhao}{\fontsize{10.5pt}{\baselineskip}\selectfont}    % 字号设置  
\newcommand{\xiaowuhao}{\fontsize{9pt}{\baselineskip}\selectfont}   % 字号设置  
\newcommand{\liuhao}{\fontsize{7.875pt}{\baselineskip}\selectfont}  % 字号设置  
\newcommand{\qihao}{\fontsize{5.25pt}{\baselineskip}\selectfont}    % 字号设置 

\usepackage{diagbox}
\usepackage{multirow}
\boldmath
\XeTeXlinebreaklocale "zh"
\XeTeXlinebreakskip = 0pt plus 1pt minus 0.1pt
\definecolor{cred}{rgb}{0.8,0.8,0.8}
\definecolor{cgreen}{rgb}{0,0.3,0}
\definecolor{cpurple}{rgb}{0.5,0,0.35}
\definecolor{cdocblue}{rgb}{0,0,0.3}
\definecolor{cdark}{rgb}{0.95,1.0,1.0}
\lstset{
	language=bash,
	numbers=left,
	numberstyle=\tiny\color{black},
	showspaces=false,
	showstringspaces=false,
	basicstyle=\scriptsize,
	keywordstyle=\color{purple},
	commentstyle=\itshape\color{cgreen},
	stringstyle=\color{blue},
	frame=lines,
	% escapeinside=``,
	extendedchars=true, 
	xleftmargin=1em,
	xrightmargin=1em, 
	backgroundcolor=\color{cred},
	aboveskip=1em,
	breaklines=true,
	tabsize=4
} 

%\newfontfamily{\consolas}{Consolas}
%\newfontfamily{\monaco}{Monaco}
%\setmonofont[Mapping={}]{Consolas}	%英文引号之类的正常显示,相当于设置英文字体
%\setsansfont{Consolas} %设置英文字体 Monaco, Consolas,  Fantasque Sans Mono
%\setmainfont{Times New Roman}
%\setCJKmainfont{STZHONGS.TTF}
%\setmonofont{Consolas}
% \newfontfamily{\consolas}{YaHeiConsolas.ttf}
\newfontfamily{\monaco}{MONACO.TTF}
\setCJKmainfont{STZHONGS.TTF}
%\setmainfont{MONACO.TTF}
%\setsansfont{MONACO.TTF}
% 自定义添加图片命令
\newcommand{\fic}[1]{\begin{figure}[H]
		\center
		\includegraphics[width=0.8\textwidth]{#1}
	\end{figure}}
	
\newcommand{\sizedfic}[2]{\begin{figure}[H]
		\center
		\includegraphics[width=#1\textwidth]{#2}
	\end{figure}}
% 
\newcommand{\codefile}[1]{\lstinputlisting{#1}}

\newcommand{\interval}{\vspace{0.5em}}

\newcommand{\tablestart}{
	\interval
	\begin{longtable}{p{2cm}p{10cm}}
	\hline}
\newcommand{\tableend}{
	\hline
	\end{longtable}
	\interval}

% 改变段间隔
\setlength{\parskip}{0.2em}
\linespread{1.1}

\usepackage{lastpage}
% 设置页眉页脚
\usepackage{fancyhdr}
\pagestyle{fancy}
\lhead{\space \qquad \space}
\chead{恒天云之SDN性能测试报告\qquad}
\rhead{\qquad\thepage/\pageref{LastPage}}

% 参考文献
\def\hang{\hangindent\parindent}
\def\textindent#1{\indent\llap{#1\enspace}\ignorespaces}
\def\re{\par\hang\textindent}
%超链接
\usepackage[CJKbookmarks=true,
            bookmarksnumbered=true,
            bookmarksopen=true,
            colorlinks, %注释掉此项则交叉引用为彩色边框(将colorlinks和pdfborder同时注释掉)
            pdfborder=001,   %注释掉此项则交叉引用为彩色边框
            linkcolor=black,
            anchorcolor=black,
            citecolor=black
            ]{hyperref}  

\begin{document}
\maketitle
\clearpage
\tableofcontents
\clearpage
\section{网络性能测试指标}
常见的网络性能测试指标包含:网络带宽(bandwidth)、网络延迟(latency)、丢包率等
\begin{itemize}
	\item[1.]网络带宽:网络带宽是指在一个固定的时间内(1秒),能通过的最大位数据。
	\item[2.]网络延迟:通俗的讲,就是数据从电脑这边传到那边所用的时间。
	\item[3.]丢包率:测试中所丢失的数据包数量占所发送的数据包的比率。
\end{itemize}

\section{网络性能测试工具选择}
\subsection{常见的网络性能测试工具}
常用的开源网络性能测试工具有:iperf、netperf、qperf,这三种种工具都可以测试TCP协议和UDP协议,从可测试的网络性能指标,我们对三种工具进行下对比:
\begin{center}
\begin{tabular}[c]{|l|l|l|l|}
\hline
工具 & 带宽 & 网络延迟 & 丢包 \\
\hline
iperf & 是 & 否 & 是 \\
\hline
netperf & 是 & 否 & 是 \\
\hline
qperf & 是 & 是 & 否 \\
\hline
\end{tabular}
\end{center}\par
可见三个工具都可以完成基本的网络性能测试,但netperf更倾向于测试不同网络模式的数据传输,与本次性能测试需求不符。
而iperf无法测试网络延迟,因此本次性能测试选择工具qperf。
\subsection{测试方法确定}
\begin{center}
\begin{tabular}[c]{|l|l|l|}
\hline
测试对象 & TCP包和UDP包带宽、延迟 & ICMP包延迟  \\
\hline
测试工具 & qperf & ping \\
\hline
\end{tabular}
\end{center}

\section{网络性能测试工具qperf的使用}
\subsection{qperf的工作原理}
使用两台机器,一台机器运行qperf充当服务器角色,一台机器运行qperf充当客户端角色
\subsection{qperf测试使用命令}
\begin{itemize}
	\item[1.]服务器端运行命令如下:
	\begin{lstlisting}
	qperf
	\end{lstlisting}
	\item[2.]客户端运行命令测试tcp的极限带宽和延迟
	\begin{lstlisting}
	qperf  服务器端ip tcp_bw tcp_lat
	\end{lstlisting}
	替换服务器端ip为具体的ip,得到结果如下
	\begin{lstlisting}
	tcp_bw:
		bw  =  117 MB/sec
	tcp_lat:
		latency  =  47.1 us
	\end{lstlisting}
	其中,tcp\_bw代表tcp带宽,tcp\_lat代表tcp延迟。
	\item[3]客户端命令测试udp的带宽和延迟
	\begin{lstlisting}
	qperf  -oo msg_size:1:64k:*2 服务器端ip udp_bw udp_lat
	\end{lstlisting}
	替换服务器端ip,改变udp包的大小(msg\_size),从1个字节到64K,每次倍增的方式,带宽和延迟情况如下
	\begin{lstlisting}
	udp_bw:
		send_bw  =  404 KB/sec
		recv_bw  =  291 KB/sec
	udp_bw:
		send_bw  =  1.01 MB/sec
		recv_bw  =   728 KB/sec
	udp_bw:
		send_bw  =  1.98 MB/sec
		recv_bw  =  1.37 MB/sec
	udp_bw:
		send_bw  =  3.95 MB/sec
		recv_bw  =  3.06 MB/sec
	udp_bw:
		send_bw  =  7.91 MB/sec
		recv_bw  =  6.51 MB/sec
	udp_bw:
		send_bw  =  16.2 MB/sec
		recv_bw  =  10.7 MB/sec
	udp_bw:
		send_bw  =  31.5 MB/sec
		recv_bw  =  24.8 MB/sec
	udp_bw:
		send_bw  =  62.5 MB/sec
		recv_bw  =  42.3 MB/sec
	udp_bw:
		send_bw  =  98.1 MB/sec
		recv_bw  =    95 MB/sec
	udp_bw:
		send_bw  =  110 MB/sec
		recv_bw  =  110 MB/sec
	udp_bw:
		send_bw  =  117 MB/sec
		recv_bw  =  117 MB/sec
	udp_bw:
		send_bw  =  117 MB/sec
		recv_bw  =  117 MB/sec
	udp_bw:
		send_bw  =  119 MB/sec
		recv_bw  =  119 MB/sec
	udp_bw:
		send_bw  =  120 MB/sec
		recv_bw  =  119 MB/sec
	udp_bw:
		send_bw  =  120 MB/sec
		recv_bw  =  119 MB/sec
	udp_bw:
		send_bw  =  120 MB/sec
		recv_bw  =  120 MB/sec
	udp_lat:
		latency  =  51.4 us
	udp_lat:
		latency  =  50.3 us
	udp_lat:
		latency  =  52.6 us
	udp_lat:
		latency  =  49.5 us
	udp_lat:
		latency  =  50 us
	udp_lat:
		latency  =  55.3 us
	udp_lat:
		latency  =  54.9 us
	udp_lat:
		latency  =  53.2 us
	udp_lat:
		latency  =  56.5 us
	udp_lat:
		latency  =  63.8 us
	udp_lat:
		latency  =  78.2 us
	udp_lat:
		latency  =  96.2 us
	udp_lat:
		latency  =  122 us
	udp_lat:
		latency  =  155 us
	udp_lat:
		latency  =  225 us
	udp_lat:
		latency  =  373 us
	\end{lstlisting}\par
	由于udp是不可靠传输,传输过程会出现丢包的现象,在测带宽的时候,会有发送带宽和接收带宽两组数据。
	发送带宽等于发送的总数据除以发送总时间,接收带宽等于接收的总数据除以接收总时间。两者之间的差值,
	可以侧面表现传输过程的丢包情况,差距越大网路性能越差。
\end{itemize}

\section{测试场景罗列}
\subsection{场景一:同时挂在提供者网络的两个 vm 之间}
\begin{itemize}
	\item[1.]同一租户相同节点提供者虚拟机
	\item[2.]同一租户不同节点提供者虚拟机
	\item[3.]不同租户相同节点提供者虚拟机
	\item[4.]不同租户不同节点提供者虚拟机
	\item[5.]计算节点1和计算节点2
	\item[6.]同一节点提供者虚拟机和物理机
	\item[7.]不同节点提供者虚拟机和物理机
\end{itemize}
\subsection{场景二:一个 vm 挂在租户网络,一个 vm 在提供者网络上}
\begin{itemize}
	\item[1.]同一租户相同节点的租户虚拟机和提供者虚拟机
	\item[2.]同一租户不同节点的租户虚拟机和提供者虚拟机
	\item[3.]不同租户相同节点的租户虚拟机和提供者虚拟机
	\item[4.]不同租户不同节点的租户虚拟机和提供者虚拟机
\end{itemize}
\subsection{场景三:同时挂在租户网络且同一网段的两个 vm 之间}
\begin{itemize}
	\item[1.]相同节点同一网段的虚拟机
	\item[2.]不同节点同一网段的虚拟机
\end{itemize}
\subsection{场景四:同时挂在租户网络且不同一网段的两个 vm 之间}
\begin{itemize}
	\item[1.]相同节点不同网段的虚拟机
	\item[2.]不同节点不同网段的虚拟机
\end{itemize}

\section{测试环境准备}
\begin{center}
\begin{tabular}[c]{|l|l|l|l|}
\hline
节点 & hty-controller & hty-compute1 & hty-compute2  \\
\hline
ip  & 172.16.133.10 & 172.16.133.15 & 172.16.133.16 \\
\hline
软件环境  & 恒天云3.8.1集成M版neutron & 同控制节点 & 同控制节点\\
\hline
网络模式  & vxlan+vlan & 同控制节点 & 同控制节点 \\
\hline
\end{tabular}
\end{center}

\section{具体测试结果及分析}
\subsection{集成M版neutron的恒天云tcp带宽和延迟测试}
测试方法:对每个应用场景的每个情况,创建对应虚拟机,使用上面提到的测试tcp命令,测试5组数据,求出均值。结果如下。为了方便分析,将所有的情况分为在同一个计算节点和不在同一个计算节点。
\begin{center}
\begin{tabular}[c]{|l|l|l|}
\hline
 &  带宽(MByte/s)&延迟(us) \\
\hline
 情况一:在同一个计算节点 &  &\\
\hline
服务器的计算节点1和计算节点1 & 4342&7.394 \\
\hline
 同一节点提供者虚拟机和物理机& 185.4&95.96 \\
 \hline
 不同租户相同节点提供者虚拟机& 97.3&239.8 \\
\hline
 同一租户相同节点提供者虚拟机 & 91.72&257.2 \\
 \hline
 相同节点同一网段的租户虚拟机& 89.6&255.4 \\
 \hline
 相同节点不同网段的租户虚拟机&  78.82&266\\
 \hline
 同一租户相同节点的租户虚拟机和提供者虚拟机& 55.82&371.2 \\
 \hline
 不同租户相同节点的租户虚拟机和提供者虚拟机& 51.52&382.6 \\
 \hline
 \hline

 情况二:不在同一个计算节点&  &\\
 \hline
 服务器的计算节点1和计算节点2& 117&47.84 \\
 \hline
 同一租户不同节点提供者虚拟机 & 67.96&310.8 \\
 \hline
 不同租户不同节点提供者虚拟机的带宽& 66.42&337 \\
 \hline
 同一租户不同节点的租户虚拟机和提供者虚拟机& 56.4&392.2 \\
 \hline
 不同租户不同节点的租户虚拟机和提供者虚拟机& 51.84&376.2 \\
 \hline
 不同节点不同网段的租户虚拟机&  44.4&320.4\\
 \hline
 不同节点同一网段的租户虚拟机&  39.6&339.8\\
 \hline
 不同节点提供者虚拟机和物理机& 35.22&259.6 \\
\hline
\end{tabular}
\end{center}

现根据上表的测试数据,统计两种情况的tcp带宽和延迟:\\
情况一:在同一节点,各测试带宽占物理机带宽(最大带宽)百分比,如下图所示\\
\sizedfic{0.9}{tcp_bw_1.png}
对应上面测试带宽的各延迟,如下图所示\\
\sizedfic{0.9}{tcp_lat_1.png}
结论一:由上图,可得同一节点的两个进程之间传输tcp数据的带宽远远大于其他情况,能够达到GByte/sec的量级。\par
结论二:同一节点,是否同一租户并不影响网络性能。\par
结论三:相同节点,提供者虚拟机到物理机的带宽 > 提供者虚拟机之间的带宽 > 租户虚拟机之间的带宽 > 租户虚拟机到提供者虚拟机之间的带宽 \par
结论四:在环境一致的情况下,经过路由的信息传输带宽始终小于未经过路由的,会有节点处理延迟以及排队延迟。\par
结论五:延迟大小,遵循带宽越大,延迟越小的规律\\\\
情况二:在不同节点,各测试带宽占物理机带宽(最大带宽)百分比,如下图所示\\
\sizedfic{0.9}{tcp_bw_2.png}
对应上面测试带宽的各延迟,如下图所示\\
\sizedfic{0.9}{tcp_lat_2.png}
结论一:由上图,可得不同物理节点传输tcp数据的带宽大于虚拟机之间的传输带宽。\par
结论二:不同节点,是否同一租户并不影响网络性能。\par
结论三:不同节点,物理机之间的传输带宽 > 提供者虚拟机之间的带宽 > 租户虚拟机到提供者虚拟机之间的带宽 > 
租户虚拟机之间的带宽 > 提供者虚拟机到物理机的带宽 \par
结论四:延迟大小并没有完全遵循带宽越大,延迟越小的规律。可能带宽相差不大的时候,延迟会有波动,网络不是很稳定\par
疑问一:提供者虚拟机之间传输数据,一个提供者虚拟机访问另一节点的提供者虚拟机,需要先访问另一个节点的物理机,按常理其带宽应该小于到物理机的带宽,但实际测试结果却是大于,有待考量。\par
疑问二:租户虚拟机之间传输数据,不同网段的虚拟机之间因为走路由的缘故,其带宽应该小于同一网段虚拟机的带宽,但结果相反,有待考量 \par
疑问三:提供者虚拟机到物理机的带宽不应该是最小的,有待考量。


\subsection{集成M版neutron的恒天云udp带宽和延迟测试}
测试方法:对每个应用场景的每种情况,测试5组数据,每组改变udp包的大小(msg\_size),从1个字节到64K,每次倍增的方式,获取对应的带宽和延迟,最后对5组数据再次求均值。由于每种情况会有16个数据,无法全部展示,这里只给出最大稳定带宽。
\begin{center}
\begin{tabular}[c]{|l|l|l|}
\hline
 &发送(MB/s) &接收(MB/s)\\
\hline
 场景一:同时挂在提供者网络的两个 vm 之间 &  & \\
\hline
 同一租户相同节点提供者虚拟机 & 12.772&11.835  \\
\hline
 同一租户不同节点提供者虚拟机 & 14.2822&13.9678  \\
 \hline
 不同租户相同节点提供者虚拟机& 14.318&13.3892  \\
 \hline
 不同租户不同节点提供者虚拟机&14.1038 &11.57678 \\
 \hline
 服务器的计算节点1和计算节点2&119.884 &119.616  \\
 \hline
 同一节点提供者虚拟机和物理机&126.964 &126.976  \\
 \hline
 同一节点物理机到提供者虚拟机&0 &0  \\
 \hline
 不同节点提供者虚拟机和物理机&136.98 &119.726  \\
 \hline
 不同节点物理机到提供者虚拟机&0 &0  \\
 \hline
 \hline
 场景二:一个 vm 挂在租户网络,一个 vm 在提供者网络上& & \\
 \hline
 同一租户相同节点的租户虚拟机和提供者虚拟机&12.7682 &12.64  \\
 \hline
 同一租户不同节点的租户虚拟机和提供者虚拟机&13.7534 &13.2396 \\
 \hline
 不同租户相同节点的租户虚拟机和提供者虚拟机&12.7824 &12.7678  \\
 \hline
 不同租户不同节点的租户虚拟机和提供者虚拟机&11.8034 &11.8034  \\
 \hline
 \hline
 场景三:同时挂在租户网络且同一网段的两个 vm 之间& & \\
 \hline
 相同节点同一网段的虚拟机& 11.63304 &11.61194 \\
 \hline
 不同节点同一网段的虚拟机&  13.7186 &13.732 \\
 \hline
 \hline
 场景四:同时挂在租户网络且不同网段的两个 vm 之间& &\\
 \hline
 相同节点不同网段的虚拟机&11.2912  &10.932 \\
 \hline
 不同节点不同网段的虚拟机&11.7046  &11.7048 \\

\hline
\end{tabular}
\end{center}

根据上表的测试数据,可以有如下总结:\par

结论一:虚拟机之间发送udp包的稳定带宽只能达到物理机的十分之一左右,带宽变化如下图所示\\
\sizedfic{0.9}{udp1.png}
上图只是代表性的图片,展示物理机之间以及某两个虚拟机之间的的发送和接收带宽对比。\par
基本走势:物理机之间随着传送数据包的增大,接收和发送带宽都在不断变大,并且能够保持稳定。
而虚拟机之间随着传送数据包的增大,发送和接收带宽也在不断变大,但当达到物理机极限带宽的十分之一左右之后,发送带宽和接收差距不断变大,出现丢包情况。\par
结论二:在提供者网络的虚拟机向物理机发送数据包,基本上可以达到极限带宽\par
结论三:物理机向提供者网络虚拟机发送udp数据包,基本被全部丢弃,带宽变化如下图所示\\
\sizedfic{0.9}{udp2.png}
上图展示了物理机到提供者网络虚拟机以及提供者网络虚拟机到物理机之间的的发送和接收带宽对比。
与结论二和三基本吻合。


\subsection{icmp延迟测试}
测试方法:对每个应用场景的每个情况,测试5组数据,每组使用ping命令默认发送大小56字节的icmp包30次的平均延迟,最后对5组数据再次求均值。结果如下。
\begin{center}
\begin{tabular}[c]{|l|l|}
\hline
 &  平均延迟(ms) \\
\hline
 场景一:同时挂在提供者网络的两个 vm 之间 &  \\
\hline
 同一租户相同节点提供者虚拟机 & 0.6694 \\
\hline
 同一租户不同节点提供者虚拟机 & 0.6456 \\
 \hline
 不同租户相同节点提供者虚拟机& 0.5406 \\
 \hline
 不同租户不同节点提供者虚拟机& 0.6428 \\
 \hline
 服务器的计算节点1和计算节点2& 0.1638 \\
 \hline
 同一节点提供者虚拟机和物理机& 0.36 \\
 \hline
 不同节点提供者虚拟机和物理机& 0.439 \\
 \hline
 \hline
 场景二:一个 vm 挂在租户网络,一个 vm 在提供者网络上& \\
 \hline
 同一租户相同节点的租户虚拟机和提供者虚拟机& 0.8736 \\
 \hline
 同一租户不同节点的租户虚拟机和提供者虚拟机& 0.951 \\
 \hline
 不同租户相同节点的租户虚拟机和提供者虚拟机& 0.8826 \\
 \hline
 不同租户不同节点的租户虚拟机和提供者虚拟机& 0.9796 \\
 \hline
 \hline
 场景三:同时挂在租户网络且同一网段的两个 vm 之间& \\
 \hline
 相同节点同一网段的虚拟机& 0.6242 \\
 \hline
 不同节点同一网段的虚拟机&  0.8342\\
 \hline
 \hline
 场景四:同时挂在租户网络且不同网段的两个 vm 之间& \\
 \hline
 相同节点不同网段的虚拟机&  0.645\\
 \hline
 不同节点不同网段的虚拟机&  0.8132\\

\hline
\end{tabular}
\end{center}

根据上表的测试数据可以有如下总结:\\
结论:延迟较小,与上面tcp测试的延迟变化基本一致

\subsection{恒天云tcp带宽和延迟测试}
测试方法同上
\begin{center}
\begin{tabular}[c]{|l|l|l|}
\hline
 &  带宽(MByte/s)&延迟(us) \\
\hline
 情况一:在同一个计算节点 &  &\\
\hline
 恒天云同一物理节点 & 3901.9&13.87 \\
\hline
 恒天云同节点虚拟机到物理机& 2867.4&29.812 \\
\hline
 恒天云相同节点虚拟机之间& 2647.4&55.545 \\
 \hline
 \hline

 情况二:不在同一个计算节点&  &\\
 \hline
 恒天云两个物理机之间& 117.67&44.908 \\
 \hline
 恒天云不同节点虚拟机到物理机 & 117.67&86.472 \\
 \hline
 恒天云不同节点虚拟机之间& 117&121.06 \\
 \hline
\hline
\end{tabular}
\end{center}

现根据上表的测试数据,统计两种情况的tcp带宽和延迟:\\
情况一:各测试带宽占物理机带宽(最大带宽)百分比,如下图所示\\
\sizedfic{0.6}{tcp_bw_11.png}
对应上面测试带宽的各延迟,如下图所示\\
\sizedfic{0.6}{tcp_lat_11.png}
结论一:同一节点,物理机之间带宽 > 虚拟机到物理机带宽 > 虚拟机之间带宽\par
结论二:上面三种情况,全都达到GByte/sec的量级\par
结论三:延迟大小遵循带宽越大,延迟越小\\
情况二:各测试带宽占物理机带宽(最大带宽)百分比,如下图所示\\
\sizedfic{0.6}{tcp_bw_22.png}
对应上面测试带宽的各延迟,如下图所示\\
\sizedfic{0.6}{tcp_lat_22.png}
结论一:不同节点,虚拟机之间的传输带宽与物理机之间的传输带宽,相差不大

\subsection{恒天云udp带宽和延迟测试}
测试方法同上,由于篇幅问题,下面仅给出带宽走势图
\sizedfic{0.9}{udp3.png}
注意:图表中有三条线重合为一条黄线\par
结论:物理节点之间的发送带宽和接受带宽走势一致,不同虚拟机之间随着发送数据量的增大,发送带宽和接收带宽相差不断增大。集成M版neutron的恒天云和投入生产的恒天云,当带宽超过一定稳定值,都会发生丢包现象。投入生产的
恒天云网络性能优于最新集成的。
\subsection{测试总结}
物理网络的极限带宽是1000Mbps,换算成字节就是120MByte/sec
\begin{itemize}
	\item[1.]对于tcp包的传输:集成M版neutron的恒天云根据不同的应用场景进行测试,带宽在30M到70M之间,投入生产的恒天云基本接近极限带宽
	\item[2.]对于udp包的传输:集成M版neutron的恒天云当带宽超过10M左右之后,就开始出现丢包,且随着发送带宽的增大,接收带宽在极速下降。投入生产的恒天云在最开始的时候也在丢包,发送带宽不断增大,接收带宽也在小幅度增加,当接收带宽达到极限带宽120M就无法增大,但发送带宽仍在增大。
\end{itemize}
\clearpage

\end{document}