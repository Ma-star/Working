% !TeX spellcheck = en_US
%% 字体:方正静蕾简体
%%		 方正粗宋
\documentclass[a4paper,left=1.5cm,right=1.5cm,11pt]{article}

\usepackage[utf8]{inputenc}
\usepackage{fontspec}
\usepackage{cite}
\usepackage{xeCJK}
\usepackage{indentfirst}
\usepackage{titlesec}
\usepackage{etoolbox}%
\makeatletter
\patchcmd{\ttlh@hang}{\parindent\z@}{\parindent\z@\leavevmode}{}{}%
\patchcmd{\ttlh@hang}{\noindent}{}{}{}%
\makeatother

\usepackage{longtable}
\usepackage{empheq}
\usepackage{graphicx}
\usepackage{float}
\usepackage{rotating}
\usepackage{subfigure}
\usepackage{tabu}
\usepackage{amsmath}
\usepackage{setspace}
\usepackage{amsfonts}
\usepackage{appendix}
\usepackage{listings}
\usepackage{xcolor}
\usepackage{geometry}
\setcounter{secnumdepth}{4}
%\titleformat*{\section}{\LARGE}
%\renewcommand\refname{参考文献}
%\titleformat{\chapter}{\centering\bfseries\huge}{}{0.7em}{}{}
\titleformat{\section}{\LARGE\bf}{\thesection}{1em}{}{}
\titleformat{\subsection}{\Large\bfseries}{\thesubsection}{1em}{}{}
\titleformat{\subsubsection}{\large\bfseries}{\thesubsubsection}{1em}{}{}
\renewcommand{\contentsname}{{ \centerline{目{  } 录}}}
\setCJKfamilyfont{cjkhwxk}{STXINGKA.TTF}
%\setCJKfamilyfont{cjkhwxk}{华文行楷}
%\setCJKfamilyfont{cjkfzcs}{方正粗宋简体}
%\newcommand*{\cjkfzcs}{\CJKfamily{cjkfzcs}}
\newcommand*{\cjkhwxk}{\CJKfamily{cjkhwxk}}
%\newfontfamily\wryh{Microsoft YaHei}
%\newfontfamily\hwzs{华文中宋}
%\newfontfamily\hwst{华文宋体}
%\newfontfamily\hwfs{华文仿宋}
%\newfontfamily\jljt{方正静蕾简体}
%\newfontfamily\hwxk{华文行楷}
\newcommand{\verylarge}{\fontsize{60pt}{\baselineskip}\selectfont}  
\newcommand{\chuhao}{\fontsize{44.9pt}{\baselineskip}\selectfont}  
\newcommand{\xiaochu}{\fontsize{38.5pt}{\baselineskip}\selectfont}  
\newcommand{\yihao}{\fontsize{27.8pt}{\baselineskip}\selectfont}  
\newcommand{\xiaoyi}{\fontsize{25.7pt}{\baselineskip}\selectfont}  
\newcommand{\erhao}{\fontsize{23.5pt}{\baselineskip}\selectfont}  
\newcommand{\xiaoerhao}{\fontsize{19.3pt}{\baselineskip}\selectfont} 
\newcommand{\sihao}{\fontsize{14pt}{\baselineskip}\selectfont}      % 字号设置  
\newcommand{\xiaosihao}{\fontsize{12pt}{\baselineskip}\selectfont}  % 字号设置  
\newcommand{\wuhao}{\fontsize{10.5pt}{\baselineskip}\selectfont}    % 字号设置  
\newcommand{\xiaowuhao}{\fontsize{9pt}{\baselineskip}\selectfont}   % 字号设置  
\newcommand{\liuhao}{\fontsize{7.875pt}{\baselineskip}\selectfont}  % 字号设置  
\newcommand{\qihao}{\fontsize{5.25pt}{\baselineskip}\selectfont}    % 字号设置 

\usepackage{diagbox}
\usepackage{multirow}
\boldmath
\XeTeXlinebreaklocale "zh"
\XeTeXlinebreakskip = 0pt plus 1pt minus 0.1pt
\definecolor{cred}{rgb}{0.8,0.8,0.8}
\definecolor{cgreen}{rgb}{0,0.3,0}
\definecolor{cpurple}{rgb}{0.5,0,0.35}
\definecolor{cdocblue}{rgb}{0,0,0.3}
\definecolor{cdark}{rgb}{0.95,1.0,1.0}
\lstset{
	language=bash,
	numbers=left,
	numberstyle=\tiny\color{black},
	showspaces=false,
	showstringspaces=false,
	basicstyle=\scriptsize,
	keywordstyle=\color{purple},
	commentstyle=\itshape\color{cgreen},
	stringstyle=\color{blue},
	frame=lines,
	% escapeinside=``,
	extendedchars=true, 
	xleftmargin=1em,
	xrightmargin=1em, 
	backgroundcolor=\color{cred},
	aboveskip=1em,
	breaklines=true,
	tabsize=4
} 

%\newfontfamily{\consolas}{Consolas}
%\newfontfamily{\monaco}{Monaco}
%\setmonofont[Mapping={}]{Consolas}	%英文引号之类的正常显示,相当于设置英文字体
%\setsansfont{Consolas} %设置英文字体 Monaco, Consolas,  Fantasque Sans Mono
%\setmainfont{Times New Roman}
%\setCJKmainfont{STZHONGS.TTF}
%\setmonofont{Consolas}
% \newfontfamily{\consolas}{YaHeiConsolas.ttf}
\newfontfamily{\monaco}{MONACO.TTF}
\setCJKmainfont{STZHONGS.TTF}
%\setmainfont{MONACO.TTF}
%\setsansfont{MONACO.TTF}

\newcommand{\fic}[1]{\begin{figure}[H]
		\center
		\includegraphics[width=0.8\textwidth]{#1}
	\end{figure}}
	
\newcommand{\sizedfic}[2]{\begin{figure}[H]
		\center
		\includegraphics[width=#1\textwidth]{#2}
	\end{figure}}

\newcommand{\codefile}[1]{\lstinputlisting{#1}}

\newcommand{\interval}{\vspace{0.5em}}

\newcommand{\tablestart}{
	\interval
	\begin{longtable}{p{2cm}p{10cm}}
	\hline}
\newcommand{\tableend}{
	\hline
	\end{longtable}
	\interval}

% 改变段间隔
\setlength{\parskip}{0.2em}
\linespread{1.1}

\usepackage{lastpage}
\usepackage{fancyhdr}
\pagestyle{fancy}
\lhead{\space \qquad \space}
\chead{20170301完成的任务和问题 \qquad}
\rhead{\qquad\thepage/\pageref{LastPage}}

\begin{document}

\tableofcontents

\clearpage


\section{20170301完成的任务和问题}

\subsection{任务一:对四台机器重装ubuntu14.04.1的系统}
    四台机器的ip地址分别是\par
\begin{itemize}
	\item[1.]hty-controller:172.16.19.147,对应网卡eth0(机器最上面的网口,一共3个网口)
	\item[2.]hty-compute1:172.16.19.149,对应网卡eth0(机器最上面的网口,一共2个网口)
	\item[3.]hty-compute2:172.16.19.151,对应网卡eth0(机器最下面,右边的网口,一共3个网口)
	\item[4.]hty-compute3:172.16.19.156,对应网卡eth0(机器中间的网口,一共3个网口)
\end{itemize}

\subsection{问题一:重装机器后,发现网卡名并非传统的eth*,需要重命名}
\begin{itemize}
	\item[1.] 新建 /etc/udev/rules.d/70-persistent-net.rules
        \begin{lstlisting}
            vi /etc/udev/rules.d/70-persistent-net.rules
        \end{lstlisting}
	\item[2.] 修改文件内容,下面更改了3个网卡的名字
        \begin{lstlisting}
            SUBSYSTEM=="net",ACTION=="add",DRIVERS=="?*",ATTR{address}=="2c:53:4a:02:d6:f8",ATTR{dev_id}=="0x0",ATTR{type}=="1",KERNEL=="p1p1",NAME="eth0"
            SUBSYSTEM=="net",ACTION=="add",DRIVERS=="?*",ATTR{address}=="2c:53:4a:02:d6:f9",ATTR{dev_id}=="0x0",ATTR{type}=="1",KERNEL=="p1p2",NAME="eth1"
            SUBSYSTEM=="net",ACTION=="add",DRIVERS=="?*",ATTR{address}=="c8:1f:66:32:f7:ff",ATTR{dev_id}=="0x0",ATTR{type}=="1",KERNEL=="p4p1",NAME="eth2"
        \end{lstlisting}
	\item[3.]KERNEL对应原网卡名,ATTR{address}对应网卡的MAC地址,NAME对应想要修改成的网卡名
	\item[4.]注意:文件中的==和=严格参照上面,否则会更改不成功。
\end{itemize}

\subsection{问题二:ip绑定的网卡与网线插上去的网口不对应}
\begin{itemize}
	\item[1.]需要不断插拔网线切换网口进行测试
	\item[2.]使用命令ip link list查看网线是否插对
	\item[3.]命令与结果如下
        \begin{lstlisting}
        ip link list
        1: lo: <LOOPBACK,UP,LOWER_UP> mtu 65536 qdisc noqueue state UNKNOWN mode DEFAULT group default 
            link/loopback 00:00:00:00:00:00 brd 00:00:00:00:00:00
        2: eth2: <BROADCAST,MULTICAST> mtu 1500 qdisc noop state DOWN mode DEFAULT group default qlen 1000
            link/ether c8:1f:66:32:f7:ff brd ff:ff:ff:ff:ff:ff
        3: eth0: <BROADCAST,MULTICAST,UP,LOWER_UP> mtu 1500 qdisc mq state UP mode DEFAULT group default qlen 1000
            link/ether 2c:53:4a:02:d6:f8 brd ff:ff:ff:ff:ff:ff
        4: eth1: <BROADCAST,MULTICAST> mtu 1500 qdisc noop state DOWN mode DEFAULT group default qlen 1000
            link/ether 2c:53:4a:02:d6:f9 brd ff:ff:ff:ff:ff:ff
        \end{lstlisting}
	\item[4.]当上面有一条显示state UP mode就表明插对了,例如上面第3条
\end{itemize}
\end{document}