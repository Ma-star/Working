% !TeX spellcheck = en_US
%% 字体:方正静蕾简体
%%		 方正粗宋
\documentclass[a4paper,left=1.5cm,right=1.5cm,11pt]{article}

\usepackage[utf8]{inputenc}
\usepackage{fontspec}
\usepackage{cite}
\usepackage{xeCJK}
\usepackage{indentfirst}
\usepackage{titlesec}
\usepackage{etoolbox}%
\makeatletter
\patchcmd{\ttlh@hang}{\parindent\z@}{\parindent\z@\leavevmode}{}{}%
\patchcmd{\ttlh@hang}{\noindent}{}{}{}%
\makeatother

\usepackage{longtable}
\usepackage{empheq}
\usepackage{graphicx}
\usepackage{float}
\usepackage{rotating}
\usepackage{subfigure}
\usepackage{tabu}
\usepackage{amsmath}
\usepackage{setspace}
\usepackage{amsfonts}
\usepackage{appendix}
\usepackage{listings}
\usepackage{xcolor}
\usepackage{geometry}
\setcounter{secnumdepth}{4}
%\titleformat*{\section}{\LARGE}
%\renewcommand\refname{参考文献}
%\titleformat{\chapter}{\centering\bfseries\huge}{}{0.7em}{}{}
\titleformat{\section}{\LARGE\bf}{\thesection}{1em}{}{}
\titleformat{\subsection}{\Large\bfseries}{\thesubsection}{1em}{}{}
\titleformat{\subsubsection}{\large\bfseries}{\thesubsubsection}{1em}{}{}
\renewcommand{\contentsname}{{ \centerline{目{  } 录}}}
\setCJKfamilyfont{cjkhwxk}{STXINGKA.TTF}
%\setCJKfamilyfont{cjkhwxk}{华文行楷}
%\setCJKfamilyfont{cjkfzcs}{方正粗宋简体}
%\newcommand*{\cjkfzcs}{\CJKfamily{cjkfzcs}}
\newcommand*{\cjkhwxk}{\CJKfamily{cjkhwxk}}
%\newfontfamily\wryh{Microsoft YaHei}
%\newfontfamily\hwzs{华文中宋}
%\newfontfamily\hwst{华文宋体}
%\newfontfamily\hwfs{华文仿宋}
%\newfontfamily\jljt{方正静蕾简体}
%\newfontfamily\hwxk{华文行楷}
\newcommand{\verylarge}{\fontsize{60pt}{\baselineskip}\selectfont}  
\newcommand{\chuhao}{\fontsize{44.9pt}{\baselineskip}\selectfont}  
\newcommand{\xiaochu}{\fontsize{38.5pt}{\baselineskip}\selectfont}  
\newcommand{\yihao}{\fontsize{27.8pt}{\baselineskip}\selectfont}  
\newcommand{\xiaoyi}{\fontsize{25.7pt}{\baselineskip}\selectfont}  
\newcommand{\erhao}{\fontsize{23.5pt}{\baselineskip}\selectfont}  
\newcommand{\xiaoerhao}{\fontsize{19.3pt}{\baselineskip}\selectfont} 
\newcommand{\sihao}{\fontsize{14pt}{\baselineskip}\selectfont}      % 字号设置  
\newcommand{\xiaosihao}{\fontsize{12pt}{\baselineskip}\selectfont}  % 字号设置  
\newcommand{\wuhao}{\fontsize{10.5pt}{\baselineskip}\selectfont}    % 字号设置  
\newcommand{\xiaowuhao}{\fontsize{9pt}{\baselineskip}\selectfont}   % 字号设置  
\newcommand{\liuhao}{\fontsize{7.875pt}{\baselineskip}\selectfont}  % 字号设置  
\newcommand{\qihao}{\fontsize{5.25pt}{\baselineskip}\selectfont}    % 字号设置 

\usepackage{diagbox}
\usepackage{multirow}
\boldmath
\XeTeXlinebreaklocale "zh"
\XeTeXlinebreakskip = 0pt plus 1pt minus 0.1pt
\definecolor{cred}{rgb}{0.8,0.8,0.8}
\definecolor{cgreen}{rgb}{0,0.3,0}
\definecolor{cpurple}{rgb}{0.5,0,0.35}
\definecolor{cdocblue}{rgb}{0,0,0.3}
\definecolor{cdark}{rgb}{0.95,1.0,1.0}
\lstset{
	language=bash,
	numbers=left,
	numberstyle=\tiny\color{black},
	showspaces=false,
	showstringspaces=false,
	basicstyle=\scriptsize,
	keywordstyle=\color{purple},
	commentstyle=\itshape\color{cgreen},
	stringstyle=\color{blue},
	frame=lines,
	% escapeinside=``,
	extendedchars=true, 
	xleftmargin=1em,
	xrightmargin=1em, 
	backgroundcolor=\color{cred},
	aboveskip=1em,
	breaklines=true,
	tabsize=4
} 

%\newfontfamily{\consolas}{Consolas}
%\newfontfamily{\monaco}{Monaco}
%\setmonofont[Mapping={}]{Consolas}	%英文引号之类的正常显示,相当于设置英文字体
%\setsansfont{Consolas} %设置英文字体 Monaco, Consolas,  Fantasque Sans Mono
%\setmainfont{Times New Roman}
%\setCJKmainfont{STZHONGS.TTF}
%\setmonofont{Consolas}
% \newfontfamily{\consolas}{YaHeiConsolas.ttf}
\newfontfamily{\monaco}{MONACO.TTF}
\setCJKmainfont{STZHONGS.TTF}
%\setmainfont{MONACO.TTF}
%\setsansfont{MONACO.TTF}

\newcommand{\fic}[1]{\begin{figure}[H]
		\center
		\includegraphics[width=0.8\textwidth]{#1}
	\end{figure}}
	
\newcommand{\sizedfic}[2]{\begin{figure}[H]
		\center
		\includegraphics[width=#1\textwidth]{#2}
	\end{figure}}

\newcommand{\codefile}[1]{\lstinputlisting{#1}}

\newcommand{\interval}{\vspace{0.5em}}

\newcommand{\tablestart}{
	\interval
	\begin{longtable}{p{2cm}p{10cm}}
	\hline}
\newcommand{\tableend}{
	\hline
	\end{longtable}
	\interval}

% 改变段间隔
\setlength{\parskip}{0.2em}
\linespread{1.1}

\usepackage{lastpage}
\usepackage{fancyhdr}
\pagestyle{fancy}
\lhead{\space \qquad \space}
\chead{手动源码安装neutron\qquad}
\rhead{\qquad\thepage/\pageref{LastPage}}

\begin{document}

\tableofcontents

\clearpage

\subsection{neutron源码各文件夹作用}
    \begin{itemize}
        \item[1.]bin/:可执行的二进制脚本文件
		\item[2.]etc/:配置文件
		\item[2.]build/:没有理解到位
		\item[3.]neutron/:源码文件
		\item[4.]tools/:工具文件夹,例如install_venv.sh,安装virtualenv,建立独立的pyhton开发环境(安装neutron需要的第三方类库)
		\item[5.]run_tests.sh:安装virtualenv,并进行单元测试
		\item[6.]setup.py:利用setuptools工具,安装neutron。
		\item[7.]setup.cfg:pbr工具,解析过滤该文件,并将解析结果作为setup.py中setup函数的默认参数。
		\item[8.]requirements.txt:neutron的第三方依赖包
		\item[9.]test-requirements.txt:neutron的测试依赖包
		\item[10.]在install_venv.sh就是,利用pip安装requirements.txt,test-requirements.txt,virtualenv
    \end{itemize}
\subsection{手动源码安装步骤}
    \begin{itemize}
		\item[1.]需要安装pip版本在8.1.2之上,否则会报错
			apt-get install python-pip
			pip install --upgrade 'pip>=8.1.2'
		\item[2.]安装virtualenv
			begin{lstlisting}
				pip install virtualenv
			end{lstlisting}
		\item[3.]创建名为venv的虚拟环境,并切换	进venv目录
			begin{lstlisting}
				virtualenv venv
				cd venv
			end{lstlisting}
			安装之后使用ls命令,查看当前文件夹的内容,会多出来bin,lib,local等文件夹。以后安装的包会放在
			这些文件夹,将不会和系统的混合在一起。这就是隔离的意义。
		\item[4.]启动虚拟环境
			begin{lstlisting}
				source bin/activate
			end{lstlisting}
        \item[5.]切换到neutron源码包,安装neutron依赖包
			begin{lstlisting}
				pip install -r requirements.txt
			end{lstlisting}
			出现报错:netifaces.c:1:20: fatal error: Python.h: No such file or directory
			执行下面命令即可
			begin{lstlisting}
				sudo apt-get install python2.7-dev
			end{lstlisting}
		\item[6.]安装neutron组件
			begin{lstlisting}
				python setup.py install
			end{lstlisting}
		\item[7.]通过源码的tools文件夹下的generate\_config\_file\_samples.sh脚本生成配置文件,并拷贝到/etc/neutron文件夹下
			begin{lstlisting}
				./tools/generate_config_file_samples.sh
			end{lstlisting}	
		\item[8.]本以为已经安装成功,可以直接启动服务,投入使用。下面是遇到的问题
			#直接运行neutron-server,TypeError: set\_defaults() got an unexpected keyword argument 'sqlite_db'?
			解决办法:/root/venv/local/lib/python2.7/site-packages/neutron/common/config.py,将该参数删除,即可。

			#运行neutron-server,提示neutron中的某个表是不存在的?
			思路:通过mysql -uroot -p123456,进入数据库,使用use neutron切换到neutron,使用show tables命令,发现neutron数据库
			确实是空的。
			解决办法:运行文档中的命令
			su -s /bin/sh -c "neutron-db-manage --config-file /etc/neutron/neutron.conf \
			--config-file /etc/neutron/plugins/ml2/ml2\_conf.ini upgrade head" neutron
			提示no passwd entry for neutron,但neutron用户是已经存在的,google查看/etc/passwd果然没有neutron那一栏。
			了解su 命令之后,发现直接运行下面命令
			apt
			再次执行neutron-server,不会报错

			#运行neutron-openvswitch-agent,提示/usr/bin/neutron-rootwrap command not found?
			思路:由于neutron是装在虚拟环境中,所有对应的neutron-rootwrap的路经不能使用默认的
			解决办法:vi /etc/neutron/neutron.conf 修改agent节点的root\_helper参数
			原:root\_helper = sudo /usr/bin/neutron-rootwrap /etc/neutron/rootwrap.conf
			改为:root\_helper = sudo /root/wgzhou/test/bin/neutron-rootwrap /etc/neutron/rootwrap.conf

			#运行neutron-openvswitch-agent,提示neutron-rootwrap: Executable not found: ovs-vsctl?
			解决办法:通过apt-get install openvswitch-switch安装switch

			#运行neutron ext-list 提示No module named memcache
			原因:由于之前的服务都是按照j版openstack进行安装,并没有在之前安装memcached。但m版的neutron是需要memcached服务的支持
			解决办法:pip install python-memcached

			#由于neutron-openvswitch-agent的配置文件中,bridge_mappings = external:br-ex,需要用命令手动添加br-ex
			解决办法:ovs-vsctl add-br br-ex

			#计算节点运行python setup.py install的时候,报错It's also possible that there is a mismatch between the package name in setup.cfg and the argument given to pbr.version.VersionInfo
		    原因:PBR从git标签生成“版本”
			解决办法:安装git,执行命令apt-get install git 

			#无法进行neutron数据库迁移,需要安装pymysql
			解决办法:pip install pymysql
		
    \end{itemize}


\end{document}