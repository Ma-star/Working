% !TeX spellcheck = en_US
%% 字体:方正静蕾简体
%%		 方正粗宋
\documentclass[a4paper,left=1.5cm,right=1.5cm,11pt]{article}

\usepackage[utf8]{inputenc}
\usepackage{fontspec}
\usepackage{cite}
\usepackage{xeCJK}
\usepackage{indentfirst}
\usepackage{titlesec}
\usepackage{etoolbox}%
\makeatletter
\patchcmd{\ttlh@hang}{\parindent\z@}{\parindent\z@\leavevmode}{}{}%
\patchcmd{\ttlh@hang}{\noindent}{}{}{}%
\makeatother

\usepackage{longtable}
\usepackage{empheq}
\usepackage{graphicx}
\usepackage{float}
\usepackage{rotating}
\usepackage{subfigure}
\usepackage{tabu}
\usepackage{amsmath}
\usepackage{setspace}
\usepackage{amsfonts}
\usepackage{appendix}
\usepackage{listings}
\usepackage{xcolor}
\usepackage{geometry}
\setcounter{secnumdepth}{4}
%\titleformat*{\section}{\LARGE}
%\renewcommand\refname{参考文献}
%\titleformat{\chapter}{\centering\bfseries\huge}{}{0.7em}{}{}
\titleformat{\section}{\LARGE\bf}{\thesection}{1em}{}{}
\titleformat{\subsection}{\Large\bfseries}{\thesubsection}{1em}{}{}
\titleformat{\subsubsection}{\large\bfseries}{\thesubsubsection}{1em}{}{}
\renewcommand{\contentsname}{{ \centerline{目{  } 录}}}
\setCJKfamilyfont{cjkhwxk}{STXINGKA.TTF}
%\setCJKfamilyfont{cjkhwxk}{华文行楷}
%\setCJKfamilyfont{cjkfzcs}{方正粗宋简体}
%\newcommand*{\cjkfzcs}{\CJKfamily{cjkfzcs}}
\newcommand*{\cjkhwxk}{\CJKfamily{cjkhwxk}}
%\newfontfamily\wryh{Microsoft YaHei}
%\newfontfamily\hwzs{华文中宋}
%\newfontfamily\hwst{华文宋体}
%\newfontfamily\hwfs{华文仿宋}
%\newfontfamily\jljt{方正静蕾简体}
%\newfontfamily\hwxk{华文行楷}
\newcommand{\verylarge}{\fontsize{60pt}{\baselineskip}\selectfont}  
\newcommand{\chuhao}{\fontsize{44.9pt}{\baselineskip}\selectfont}  
\newcommand{\xiaochu}{\fontsize{38.5pt}{\baselineskip}\selectfont}  
\newcommand{\yihao}{\fontsize{27.8pt}{\baselineskip}\selectfont}  
\newcommand{\xiaoyi}{\fontsize{25.7pt}{\baselineskip}\selectfont}  
\newcommand{\erhao}{\fontsize{23.5pt}{\baselineskip}\selectfont}  
\newcommand{\xiaoerhao}{\fontsize{19.3pt}{\baselineskip}\selectfont} 
\newcommand{\sihao}{\fontsize{14pt}{\baselineskip}\selectfont}      % 字号设置  
\newcommand{\xiaosihao}{\fontsize{12pt}{\baselineskip}\selectfont}  % 字号设置  
\newcommand{\wuhao}{\fontsize{10.5pt}{\baselineskip}\selectfont}    % 字号设置  
\newcommand{\xiaowuhao}{\fontsize{9pt}{\baselineskip}\selectfont}   % 字号设置  
\newcommand{\liuhao}{\fontsize{7.875pt}{\baselineskip}\selectfont}  % 字号设置  
\newcommand{\qihao}{\fontsize{5.25pt}{\baselineskip}\selectfont}    % 字号设置 

\usepackage{diagbox}
\usepackage{multirow}
\boldmath
\XeTeXlinebreaklocale "zh"
\XeTeXlinebreakskip = 0pt plus 1pt minus 0.1pt
\definecolor{cred}{rgb}{0.8,0.8,0.8}
\definecolor{cgreen}{rgb}{0,0.3,0}
\definecolor{cpurple}{rgb}{0.5,0,0.35}
\definecolor{cdocblue}{rgb}{0,0,0.3}
\definecolor{cdark}{rgb}{0.95,1.0,1.0}
\lstset{
	language=bash,
	numbers=left,
	numberstyle=\tiny\color{black},
	showspaces=false,
	showstringspaces=false,
	basicstyle=\scriptsize,
	keywordstyle=\color{purple},
	commentstyle=\itshape\color{cgreen},
	stringstyle=\color{blue},
	frame=lines,
	% escapeinside=``,
	extendedchars=true, 
	xleftmargin=1em,
	xrightmargin=1em, 
	backgroundcolor=\color{cred},
	aboveskip=1em,
	breaklines=true,
	tabsize=4
} 

%\newfontfamily{\consolas}{Consolas}
%\newfontfamily{\monaco}{Monaco}
%\setmonofont[Mapping={}]{Consolas}	%英文引号之类的正常显示,相当于设置英文字体
%\setsansfont{Consolas} %设置英文字体 Monaco, Consolas,  Fantasque Sans Mono
%\setmainfont{Times New Roman}
%\setCJKmainfont{STZHONGS.TTF}
%\setmonofont{Consolas}
% \newfontfamily{\consolas}{YaHeiConsolas.ttf}
\newfontfamily{\monaco}{MONACO.TTF}
\setCJKmainfont{STZHONGS.TTF}
%\setmainfont{MONACO.TTF}
%\setsansfont{MONACO.TTF}

\newcommand{\fic}[1]{\begin{figure}[H]
		\center
		\includegraphics[width=0.8\textwidth]{#1}
	\end{figure}}
	
\newcommand{\sizedfic}[2]{\begin{figure}[H]
		\center
		\includegraphics[width=#1\textwidth]{#2}
	\end{figure}}

\newcommand{\codefile}[1]{\lstinputlisting{#1}}

\newcommand{\interval}{\vspace{0.5em}}

\newcommand{\tablestart}{
	\interval
	\begin{longtable}{p{2cm}p{10cm}}
	\hline}
\newcommand{\tableend}{
	\hline
	\end{longtable}
	\interval}

% 改变段间隔
\setlength{\parskip}{0.2em}
\linespread{1.1}

\usepackage{lastpage}
\usepackage{fancyhdr}
\pagestyle{fancy}
\lhead{\space \qquad \space}
\chead{软件源相关命令\qquad}
\rhead{\qquad\thepage/\pageref{LastPage}}

\begin{document}

\tableofcontents

\clearpage


\section{软件源相关命令}

\subsection{apt的常用命令}
\begin{lstlisting}
	apt-cache search # ------(package 搜索包)
	apt-cache show #------(package 获取包的相关信息,如说明、大小、版本等)
	apt-get install # ------(package 安装包)
	apt-get install # -----(package --reinstall 重新安装包)
	apt-get -f install # -----(强制安装, "-f = --fix-missing"当是修复安装吧...)
	apt-get remove #-----(package 删除包)
	apt-get remove --purge # ------(package 删除包,包括删除配置文件等)
	apt-get autoremove --purge # ----(package 删除包及其依赖的软件包+配置文件等(只对6.10有效,强烈推荐))
	apt-get update #------更新源
	apt-get upgrade #------更新已安装的包
	apt-get dist-upgrade # ---------升级系统
	apt-get dselect-upgrade #------使用 dselect 升级
	apt-cache depends #-------(package 了解使用依赖)
	apt-cache rdepends # ------(package 了解某个具体的依赖,当是查看该包被哪些包依赖吧...)
	apt-get build-dep # ------(package 安装相关的编译环境)
	apt-get source #------(package 下载该包的源代码)
	apt-get clean && apt-get autoclean # --------清理下载文件的存档 && 只清理过时的包
	apt-get check #-------检查是否有损坏的依赖
	dpkg -S filename -----查找filename属于哪个软件包
	apt-file search filename -----查找filename属于哪个软件包
	apt-file list packagename -----列出软件包的内容
	apt-file update --更新apt-file的数据库

	dpkg --info "软件包名" --列出软件包解包后的包名称.
	dpkg -l --列出当前系统中所有的包.可以和参数less一起使用在分屏查看. (类似于rpm -qa)
	dpkg -l |grep -i "软件包名" --查看系统中与"软件包名"相关联的包.
	dpkg -s 查询已安装的包的详细信息.
	dpkg -L 查询系统中已安装的软件包所安装的位置. (类似于rpm -ql)
	dpkg -S 查询系统中某个文件属于哪个软件包. (类似于rpm -qf)
	dpkg -I 查询deb包的详细信息,在一个软件包下载到本地之后看看用不用安装(看一下呗).
	dpkg -i 手动安装软件包(这个命令并不能解决软件包之前的依赖性问题),如果在安装某一个软件包的时候遇到了软件依赖的问题,可以用apt-get -f install在解决信赖性这个问题.
	dpkg -r 卸载软件包.不是完全的卸载,它的配置文件还存在.
	dpkg -P 全部卸载(但是还是不能解决软件包的依赖性的问题)
	dpkg -reconfigure 重新配置
	dpkg --purge 完全卸载
\end{lstlisting}

\subsection{apt详细描述如下}
\begin{itemize}
	\item[1.]apt-get install \par
		下载软件包,以及所有依赖的包,同时进行包的安装或升级。如果某个包被设置了 hold (停止标志,就会被搁在一边(即不会被升级)。更多 hold 细节请看下面。
	\item[2.]apt-get remove [--purge]\par
		移除 以及任何依赖这个包的其它包。
		--purge 指明这个包应该被完全清除 (purged) ,更多信息请看 dpkg -P。
	\item[3.]apt-get update\par
		升级来自 Debian 镜像的包列表,如果你想安装当天的任何软件,至少每天运行一次,而且每次修改了
		/etc/apt/sources.list 後,必须执行。
	\item[4.]apt-get upgrade [-u]\par
		升级所有已经安装的包为最新可用版本。不会安装新的或移除老的包。如果一个包改变了依赖关系而需要安装一个新的包,那么它将不会被升级,而是标志为 hold。
		apt-get update 不会升级被标志为 hold 的包 (这个也就是 hold 的意思)。请看下文如何手动设置包为 hold。我建议同时使用 '-u' 选项,因为这样你就能看到哪些包将会被升级。
	\item[5.]apt-get dist-upgrade [-u]\par	
		和 apt-get upgrade 类似,除了 dist-upgrade 会安装和移除包来满足依赖关系。因此具有一定的危险性。
	\item[6.]apt-cache search\par
		在软件包名称和描述中,搜索包含xxx的软件包。
	\item[7.]apt-cache show\par
		显示某个软件包的完整的描述。
	\item[8.]apt-cache showpkg\par
		显示软件包更多细节,以及和其它包的关系。
	\item[9.]apt-get clean\par
		经过apt-get下载并安装的 DEB 套件, 会存放在 /var/cache/apt/archives/, 不会自动刪除, 使用 apt-get clean会将 /var/cache/apt/archives/ 的 所有 deb 刪掉.	
	\item[10.]apt-get autoclean\par
		apt-get autoclean 只会刪除 /var/cache/apt/archives/ 已经过期的 DEB套件档
	\item[11.]apt-get install package=version\par
		安装指定版本的软件
\end{itemize}


\end{document}