% !TeX spellcheck = en_US
%% 字体:方正静蕾简体
%%		 方正粗宋
\documentclass[a4paper,left=1.5cm,right=1.5cm,11pt]{article}

\usepackage[utf8]{inputenc}
\usepackage{fontspec}
\usepackage{cite}
\usepackage{xeCJK}
\usepackage{indentfirst}
\usepackage{titlesec}
\usepackage{etoolbox}%
\makeatletter
\patchcmd{\ttlh@hang}{\parindent\z@}{\parindent\z@\leavevmode}{}{}%
\patchcmd{\ttlh@hang}{\noindent}{}{}{}%
\makeatother

\usepackage{longtable}
\usepackage{empheq}
\usepackage{graphicx}
\usepackage{float}
\usepackage{rotating}
\usepackage{subfigure}
\usepackage{tabu}
\usepackage{amsmath}
\usepackage{setspace}
\usepackage{amsfonts}
\usepackage{appendix}
\usepackage{listings}
\usepackage{xcolor}
\usepackage{geometry}
\setcounter{secnumdepth}{4}
%\titleformat*{\section}{\LARGE}
%\renewcommand\refname{参考文献}
%\titleformat{\chapter}{\centering\bfseries\huge}{}{0.7em}{}{}
\titleformat{\section}{\LARGE\bf}{\thesection}{1em}{}{}
\titleformat{\subsection}{\Large\bfseries}{\thesubsection}{1em}{}{}
\titleformat{\subsubsection}{\large\bfseries}{\thesubsubsection}{1em}{}{}
\renewcommand{\contentsname}{{ \centerline{目{  } 录}}}
\setCJKfamilyfont{cjkhwxk}{STXINGKA.TTF}
%\setCJKfamilyfont{cjkhwxk}{华文行楷}
%\setCJKfamilyfont{cjkfzcs}{方正粗宋简体}
%\newcommand*{\cjkfzcs}{\CJKfamily{cjkfzcs}}
\newcommand*{\cjkhwxk}{\CJKfamily{cjkhwxk}}
%\newfontfamily\wryh{Microsoft YaHei}
%\newfontfamily\hwzs{华文中宋}
%\newfontfamily\hwst{华文宋体}
%\newfontfamily\hwfs{华文仿宋}
%\newfontfamily\jljt{方正静蕾简体}
%\newfontfamily\hwxk{华文行楷}
\newcommand{\verylarge}{\fontsize{60pt}{\baselineskip}\selectfont}  
\newcommand{\chuhao}{\fontsize{44.9pt}{\baselineskip}\selectfont}  
\newcommand{\xiaochu}{\fontsize{38.5pt}{\baselineskip}\selectfont}  
\newcommand{\yihao}{\fontsize{27.8pt}{\baselineskip}\selectfont}  
\newcommand{\xiaoyi}{\fontsize{25.7pt}{\baselineskip}\selectfont}  
\newcommand{\erhao}{\fontsize{23.5pt}{\baselineskip}\selectfont}  
\newcommand{\xiaoerhao}{\fontsize{19.3pt}{\baselineskip}\selectfont} 
\newcommand{\sihao}{\fontsize{14pt}{\baselineskip}\selectfont}      % 字号设置  
\newcommand{\xiaosihao}{\fontsize{12pt}{\baselineskip}\selectfont}  % 字号设置  
\newcommand{\wuhao}{\fontsize{10.5pt}{\baselineskip}\selectfont}    % 字号设置  
\newcommand{\xiaowuhao}{\fontsize{9pt}{\baselineskip}\selectfont}   % 字号设置  
\newcommand{\liuhao}{\fontsize{7.875pt}{\baselineskip}\selectfont}  % 字号设置  
\newcommand{\qihao}{\fontsize{5.25pt}{\baselineskip}\selectfont}    % 字号设置 

\usepackage{diagbox}
\usepackage{multirow}
\boldmath
\XeTeXlinebreaklocale "zh"
\XeTeXlinebreakskip = 0pt plus 1pt minus 0.1pt
\definecolor{cred}{rgb}{0.8,0.8,0.8}
\definecolor{cgreen}{rgb}{0,0.3,0}
\definecolor{cpurple}{rgb}{0.5,0,0.35}
\definecolor{cdocblue}{rgb}{0,0,0.3}
\definecolor{cdark}{rgb}{0.95,1.0,1.0}
\lstset{
	language=bash,
	numbers=left,
	numberstyle=\tiny\color{black},
	showspaces=false,
	showstringspaces=false,
	basicstyle=\scriptsize,
	keywordstyle=\color{purple},
	commentstyle=\itshape\color{cgreen},
	stringstyle=\color{blue},
	frame=lines,
	% escapeinside=``,
	extendedchars=true, 
	xleftmargin=1em,
	xrightmargin=1em, 
	backgroundcolor=\color{cred},
	aboveskip=1em,
	breaklines=true,
	tabsize=4
} 

%\newfontfamily{\consolas}{Consolas}
%\newfontfamily{\monaco}{Monaco}
%\setmonofont[Mapping={}]{Consolas}	%英文引号之类的正常显示,相当于设置英文字体
%\setsansfont{Consolas} %设置英文字体 Monaco, Consolas,  Fantasque Sans Mono
%\setmainfont{Times New Roman}
%\setCJKmainfont{STZHONGS.TTF}
%\setmonofont{Consolas}
% \newfontfamily{\consolas}{YaHeiConsolas.ttf}
\newfontfamily{\monaco}{MONACO.TTF}
\setCJKmainfont{STZHONGS.TTF}
%\setmainfont{MONACO.TTF}
%\setsansfont{MONACO.TTF}

\newcommand{\fic}[1]{\begin{figure}[H]
		\center
		\includegraphics[width=0.8\textwidth]{#1}
	\end{figure}}
	
\newcommand{\sizedfic}[2]{\begin{figure}[H]
		\center
		\includegraphics[width=#1\textwidth]{#2}
	\end{figure}}

\newcommand{\codefile}[1]{\lstinputlisting{#1}}

\newcommand{\interval}{\vspace{0.5em}}

\newcommand{\tablestart}{
	\interval
	\begin{longtable}{p{2cm}p{10cm}}
	\hline}
\newcommand{\tableend}{
	\hline
	\end{longtable}
	\interval}

% 改变段间隔
\setlength{\parskip}{0.2em}
\linespread{1.1}

\usepackage{lastpage}
\usepackage{fancyhdr}
\pagestyle{fancy}
\lhead{\space \qquad \space}
\chead{容器网络\qquad}
\rhead{\qquad\thepage/\pageref{LastPage}}

\begin{document}

\tableofcontents

\clearpage

\subsection{外部访问容器的方式(web访问db,web父容器,db子容器)}
\begin{itemize}
	\item[1.]通过端口映射-p hostport:containerport
	\item[2.]通过容器互联 --link 要连接的容器名:别名 例如db:db
	\item[3.]查看连接信息方式一:在创建web容器后添加env命令
	\item[4.]查看连接信息方式二:查看web容器的/etc/hosts文件
	\item[5.]用户可以连接多个父容器到子容器,多个web容器,连接到db容器
\end{itemize}

\sbusection{Docker网络相关的命令列表}
\begin{itemize}
	\item[1.]-b BRIDGE or --bridge=BRIDGE --指定容器挂载的网桥
	\item[2.]--bip=CIDR --定制 docker0 的掩码
	\item[3.]-H SOCKET... or --host=SOCKET... --Docker 服务端接收命令的通道
	\item[4.]--icc=true|false --是否支持容器之间进行通信
	\item[5.]--ip-forward=true|false --ip地址转发
	\item[6.]--iptables=true|false --是否允许 Docker 添加 iptables 规则
	\item[7.]--mtu=BYTES --容器网络中的 MTU
	\item[8.]--dns=IP_ADDRESS... --使用指定的DNS服务器
	\item[9.]--dns-search=DOMAIN... --指定DNS搜索域
	\item[10.]-h HOSTNAME or --hostname=HOSTNAME --配置容器主机名
	\item[11.]--link=CONTAINER_NAME:ALIAS --添加到另一个容器的连接
	\item[12.]--net=bridge|none|container:NAME_or_ID|host --配置容器的桥接模式
	\item[13.]-p SPEC or --publish=SPEC --映射容器端口到宿主主机
	\item[14.]-P or --publish-all=true|false --映射容器所有端口到宿主主机
\end{itemize}

\subsection{使用pipework,将docker容器配置到本地网络环境中}
\begin{itemize}
	\item[1.]安装pipework
		\begin{lstlisting}
			git clone https://github.com/jpetazzo/pipework
			cp ~/pipework/pipework /usr/local/bin/
		\end{lstlisting}
	\item[2.]启动Docker容器
		\begin{lstlisting}
			docker run -itd --name test1 ubuntu /bin/bash
		\end{lstlisting}
	\item[3.]配置容器网络,并连到网桥docker0上。网关在IP地址后面加@指定
		\begin{lstlisting}
			pipework docker0 test1 10.0.2.100/24@10.0.2.1
		\end{lstlisting}
	\item[4.]将主机eth0桥接到docker0上,并把eth0的IP配置在docker0上。\par
	这里由于是远程操作,中间网络会断掉,所以放在一条命令中执行。
		\begin{lstlisting}
			ip addr add 10.0.2.7/24 dev docker0; \ 
			ip addr del 10.0.2.7/24 dev eth0; \
			brctl addif docker0 eth0; \
			ip route del default; \
			ip route add default via 10.0.2.1 dev docker0
		\end{lstlisting}
	
\end{itemize}
\subsection{doker的四种网络模式}
\begin{itemize}
	\item[1.]host模式:与宿主机共用命名空间,不会生成新的网卡。假如宿主机10.10.101.105/24\par
	用host模式启动一个含有web应用的Docker容器,可以直接使用10.10.101.105:80访问
	\item[2.]container模式:新创建的容器不会创建自己的网卡,配置自己的ip,而是和指定的容器共享ip和端口访问。
	\item[3.]none模式:拥有自己的命名空间,但需要手动添加网卡和ip
	\item[4.]桥接模式:当docker服务启动时,会在主机上创建一个默认docker0的虚拟网卡。容器\par
	使用他和来和主机相互通信。当创建一个docker的容器的时候,它就创建了一个对接口。这对接口在容器的一端\par
	的名字是eth0,宿主机会指定一个指定的名字。例如veth**的名字。所有的veth的接口都会桥接到docker,这样\par
\end{itemize}

\end{document}