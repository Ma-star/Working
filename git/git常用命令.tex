% !TeX spellcheck = en_US
%% 字体:方正静蕾简体
%%		 方正粗宋
\documentclass[a4paper,left=1.5cm,right=1.5cm,11pt]{article}
\title{git常用命令}
\author{周威光整理\footnote{简介:恒天云FTE}}
\date{2017-7-2} 
% 引用宏包
\usepackage[utf8]{inputenc}
\usepackage{fontspec}
\usepackage{cite}
\usepackage{xeCJK}
\usepackage{indentfirst}
\usepackage{titlesec}
\usepackage{etoolbox}%
\makeatletter
\patchcmd{\ttlh@hang}{\parindent\z@}{\parindent\z@\leavevmode}{}{}%
\patchcmd{\ttlh@hang}{\noindent}{}{}{}%
\makeatother

\usepackage{longtable}
\usepackage{empheq}
\usepackage{graphicx}
\usepackage{float}
\usepackage{rotating}
\usepackage{subfigure}
\usepackage{tabu}
\usepackage{amsmath}
\usepackage{setspace}
\usepackage{amsfonts}
\usepackage{appendix}
\usepackage{listings}
\usepackage{xcolor}
\usepackage{geometry}
\setcounter{secnumdepth}{4}
%\titleformat*{\section}{\LARGE}
%\renewcommand\refname{参考文献}
%\titleformat{\chapter}{\centering\bfseries\huge}{}{0.7em}{}{}
\titleformat{\section}{\LARGE\bf}{\thesection}{1em}{}{}
\titleformat{\subsection}{\Large\bfseries}{\thesubsection}{1em}{}{}
\titleformat{\subsubsection}{\large\bfseries}{\thesubsubsection}{1em}{}{}
\renewcommand{\contentsname}{{ \centerline{目{  } 录}}}
\setCJKfamilyfont{cjkhwxk}{STXINGKA.TTF}
%\setCJKfamilyfont{cjkhwxk}{华文行楷}
%\setCJKfamilyfont{cjkfzcs}{方正粗宋简体}
%\newcommand*{\cjkfzcs}{\CJKfamily{cjkfzcs}}
\newcommand*{\cjkhwxk}{\CJKfamily{cjkhwxk}}
%\newfontfamily\wryh{Microsoft YaHei}
%\newfontfamily\hwzs{华文中宋}
%\newfontfamily\hwst{华文宋体}
%\newfontfamily\hwfs{华文仿宋}
%\newfontfamily\jljt{方正静蕾简体}
%\newfontfamily\hwxk{华文行楷}
\newcommand{\verylarge}{\fontsize{60pt}{\baselineskip}\selectfont}  
\newcommand{\chuhao}{\fontsize{44.9pt}{\baselineskip}\selectfont}  
\newcommand{\xiaochu}{\fontsize{38.5pt}{\baselineskip}\selectfont}  
\newcommand{\yihao}{\fontsize{27.8pt}{\baselineskip}\selectfont}  
\newcommand{\xiaoyi}{\fontsize{25.7pt}{\baselineskip}\selectfont}  
\newcommand{\erhao}{\fontsize{23.5pt}{\baselineskip}\selectfont}  
\newcommand{\xiaoerhao}{\fontsize{19.3pt}{\baselineskip}\selectfont} 
\newcommand{\sihao}{\fontsize{14pt}{\baselineskip}\selectfont}      % 字号设置  
\newcommand{\xiaosihao}{\fontsize{12pt}{\baselineskip}\selectfont}  % 字号设置  
\newcommand{\wuhao}{\fontsize{10.5pt}{\baselineskip}\selectfont}    % 字号设置  
\newcommand{\xiaowuhao}{\fontsize{9pt}{\baselineskip}\selectfont}   % 字号设置  
\newcommand{\liuhao}{\fontsize{7.875pt}{\baselineskip}\selectfont}  % 字号设置  
\newcommand{\qihao}{\fontsize{5.25pt}{\baselineskip}\selectfont}    % 字号设置 

\usepackage{diagbox}
\usepackage{multirow}
\boldmath
\XeTeXlinebreaklocale "zh"
\XeTeXlinebreakskip = 0pt plus 1pt minus 0.1pt
\definecolor{cred}{rgb}{0.8,0.8,0.8}
\definecolor{cgreen}{rgb}{0,0.3,0}
\definecolor{cpurple}{rgb}{0.5,0,0.35}
\definecolor{cdocblue}{rgb}{0,0,0.3}
\definecolor{cdark}{rgb}{0.95,1.0,1.0}
\lstset{
	language=bash,
	numbers=left,
	numberstyle=\tiny\color{black},
	showspaces=false,
	showstringspaces=false,
	basicstyle=\scriptsize,
	keywordstyle=\color{purple},
	commentstyle=\itshape\color{cgreen},
	stringstyle=\color{blue},
	frame=lines,
	% escapeinside=``,
	extendedchars=true, 
	xleftmargin=1em,
	xrightmargin=1em, 
	backgroundcolor=\color{cred},
	aboveskip=1em,
	breaklines=true,
	tabsize=4
} 

%\newfontfamily{\consolas}{Consolas}
%\newfontfamily{\monaco}{Monaco}
%\setmonofont[Mapping={}]{Consolas}	%英文引号之类的正常显示,相当于设置英文字体
%\setsansfont{Consolas} %设置英文字体 Monaco, Consolas,  Fantasque Sans Mono
%\setmainfont{Times New Roman}
%\setCJKmainfont{STZHONGS.TTF}
%\setmonofont{Consolas}
% \newfontfamily{\consolas}{YaHeiConsolas.ttf}
\newfontfamily{\monaco}{MONACO.TTF}
\setCJKmainfont{STZHONGS.TTF}
%\setmainfont{MONACO.TTF}
%\setsansfont{MONACO.TTF}
% 自定义添加图片命令
\newcommand{\fic}[1]{\begin{figure}[H]
		\center
		\includegraphics[width=0.8\textwidth]{#1}
	\end{figure}}
	
\newcommand{\sizedfic}[2]{\begin{figure}[H]
		\center
		\includegraphics[width=#1\textwidth]{#2}
	\end{figure}}
% 
\newcommand{\codefile}[1]{\lstinputlisting{#1}}

\newcommand{\interval}{\vspace{0.5em}}

\newcommand{\tablestart}{
	\interval
	\begin{longtable}{p{2cm}p{10cm}}
	\hline}
\newcommand{\tableend}{
	\hline
	\end{longtable}
	\interval}

% 改变段间隔
\setlength{\parskip}{0.2em}
\linespread{1.1}

\usepackage{lastpage}
% 设置页眉页脚
\usepackage{fancyhdr}
\pagestyle{fancy}
\lhead{\space \qquad \space}
\chead{git常用命令\qquad}
\rhead{\qquad\thepage/\pageref{LastPage}}

% 参考文献
\def\hang{\hangindent\parindent}
\def\textindent#1{\indent\llap{#1\enspace}\ignorespaces}
\def\re{\par\hang\textindent}
%超链接
\usepackage[CJKbookmarks=true,
            bookmarksnumbered=true,
            bookmarksopen=true,
            colorlinks, %注释掉此项则交叉引用为彩色边框(将colorlinks和pdfborder同时注释掉)
            pdfborder=001,   %注释掉此项则交叉引用为彩色边框
            linkcolor=green,
            anchorcolor=green,
            citecolor=green
            ]{hyperref}  

\begin{document}
\maketitle
\clearpage
\tableofcontents
\clearpage
\section{git常用命令}
\subsection{创建版本库}
\begin{itemize}
	\item[1.]ubuntu系统安装git
	\begin{lstlisting}
	apt-get install git
	\end{lstlisting}
	\item[2.]创建本地版本库
	\begin{lstlisting}
	mkdir my_repository
	cd my_repository
	git init
	\end{lstlisting}
	\item[3]将文件提交到版本库基本步骤
	\begin{lstlisting}
	touch readme.txt
	git add readme.txt
	git commit -m "wrote readme.txt"
	\end{lstlisting}
\end{itemize}
\subsection{版本回退}
\begin{itemize}
	\item[1.]查看本仓库当前所有提交commit
	\begin{lstlisting}
	git log
	\end{lstlisting}
	\item[2.]回退到上一个版本或多个版本
	\begin{lstlisting}
	git reset --hard HEAD^  #回退上个版本
	git reset --hard HEAD~n #回退n个版本
	git reset --hard commit_id  #回退到指定版本号
	\end{lstlisting}
	\item[3.]查看本仓库HEAD指针改变所有记录
	\begin{lstlisting}
	git reflog
	\end{lstlisting}
\end{itemize}
\subsection{工作区和暂存区}
\sizedfic{0.6}{git.png}
\begin{itemize}
	\item[1.]工作区:my\_repository文件夹
	\item[2.]版本库:工作区的的隐藏目录“.git”,这个不算工作区,而是git的版本库。\\\\
	Git的版本库里存了很多东西,其中最重要的就是称为stage(或者叫index)的暂存区,
	还有Git为我们自动创建的第一个分支master,以及指向master的一个指针叫HEAD。\\\\
	文件往Git版本库里添加的时候,是分两步执行的: \par
	第一步是用“git add”把文件添加进去,实际上就是把文件修改添加到暂存区; \par
	第二步是用“git commit”提交更改,实际上就是把暂存区的所有内容提交到当前分支。\\\\
	因为我们创建Git版本库时,Git自动为我们创建了唯一一个master分支,所以,现在,commit就是往master分支上提交更
	改。你可以简单理解为,需要提交的文件修改通通放到暂存区,然后,一次性提交暂存区的所有修改。
\end{itemize}
\subsection{管理修改}
\begin{itemize}
	\item[1.]查看工作区对比版本库中的状态变化
	\begin{lstlisting}
	git status
	\end{lstlisting}
	\item[2.]查看指定文件的具体修改
	\begin{lstlisting}
	git diff readme.txt #如果暂存区中存在该文件,则与暂存区对比,否则与master分支对比
	git diff HEAD -- readme.txt # 指定和master分支对比
	\end{lstlisting}
\end{itemize}
\subsection{撤销修改}
\begin{itemize}
	\item[1.]丢弃工作区的修改
	\begin{lstlisting}
	git checkout -- readme.txt #工作区会被还原
	\end{lstlisting} 
	命令 git checkout -- readme.txt 意思就是,把readme.txt文件在工作区的修改全部撤销,这里有两种情况:\par
	一种是readme.txt自修改后还没有被放到暂存区,现在,撤销修改就回到和版本库一模一样的状态;\\
	一种是readme.txt已经添加到暂存区后,又作了修改,现在,撤销修改就回到添加到暂存区后的状态。\\
	总之,就是让这个文件回到最近一次git commit或git add时的状态。
	\item[2.]把暂存区的修改回退到工作区
	\begin{lstlisting}
	git reset HEAD readme.txt #暂存区被清理掉,不改变工作区的内容
	\end{lstlisting} 
	git reset命令既可以回退版本,也可以把暂存区的修改回退到工作区
\end{itemize}
\subsection{删除文件}
\begin{itemize}
	\item[1.]若不小心删除了文件,想恢复
	\begin{lstlisting}
	rm test.txt
	git checkout -- test.txt #用版本库中的恢复
	\end{lstlisting}
	\item[2.]若删除了文件,想同步到版本库
	\begin{lstlisting}
	rm test.txt
	git rm test.txt 
	git commit -m "remove test.txt"
	\end{lstlisting}
\end{itemize}
\subsection{远程仓库}
\begin{itemize}
	\item[1.]建立本地库与远程库的关联
	\begin{lstlisting}
	git remote add Life git@github.com:weiguangzhou/Life.git
	git pull Life master #将远程的Life仓库的master分支更新到本地,并与本地master合并
	git push Life master #将本地文件推送到远程库
	\end{lstlisting}
	\item[2.]克隆远程库
	\begin{lstlisting}
	git clone git@github.com:weiguangzhou/Life.git
	\end{lstlisting}
\end{itemize}
\subsection{分支管理}
\begin{itemize}
	\item[1.]创建分支
	\begin{lstlisting}
	git branch #看看当前分支
	git branch -a #查看所有分支
	git branch dev #创建dev分支
	git checkout dev #切换到dev分支
	git checkout -b dev #创建dev分支,并切换到该分支,相当于上面两个命令
	\end{lstlisting}
	\item[2.]合并和删除分支
	\begin{lstlisting}
	git merge dev
	git branch -d dev
	\end{lstlisting}
	\item[3.]解决冲突
	\begin{lstlisting}
	git branch -b feature1 #创建feature1分支,并切换,并修改readme.txt内容,保存提交
	git checkout master #切回主分支,同样修改readme.txt内容,保存提交。
	git merge feature1 #合并新建分支时,会冲突
	git status #查看冲突文件,并手动修改
	git add readme.txt 
	git commit -m "conflict merge" #提交合并
	git log --graph --pretty=oneline --abbrev-commit #查看分支合并情况
	git log --graph #同样可以查看分支合并情况
	\end{lstlisting}
\end{itemize}
\clearpage
\begin{center}
参考文献
\end{center}
\re{[1]} \href{http://www.liaoxuefeng.com/wiki/0013739516305929606dd18361248578c67b8067c8c017b000}{廖学峰之Git教程}  
\end{document}