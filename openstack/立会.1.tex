% !TeX spellcheck = en_US
%% 字体:方正静蕾简体
%%		 方正粗宋
\documentclass[a4paper,left=1.5cm,right=1.5cm,11pt]{article}

\usepackage[utf8]{inputenc}
\usepackage{fontspec}
\usepackage{cite}
\usepackage{xeCJK}
\usepackage{indentfirst}
\usepackage{titlesec}
\usepackage{etoolbox}%
\makeatletter
\patchcmd{\ttlh@hang}{\parindent\z@}{\parindent\z@\leavevmode}{}{}%
\patchcmd{\ttlh@hang}{\noindent}{}{}{}%
\makeatother

\usepackage{longtable}
\usepackage{empheq}
\usepackage{graphicx}
\usepackage{float}
\usepackage{rotating}
\usepackage{subfigure}
\usepackage{tabu}
\usepackage{amsmath}
\usepackage{setspace}
\usepackage{amsfonts}
\usepackage{appendix}
\usepackage{listings}
\usepackage{xcolor}
\usepackage{geometry}
\setcounter{secnumdepth}{4}
%\titleformat*{\section}{\LARGE}
%\renewcommand\refname{参考文献}
%\titleformat{\chapter}{\centering\bfseries\huge}{}{0.7em}{}{}
\titleformat{\section}{\LARGE\bf}{\thesection}{1em}{}{}
\titleformat{\subsection}{\Large\bfseries}{\thesubsection}{1em}{}{}
\titleformat{\subsubsection}{\large\bfseries}{\thesubsubsection}{1em}{}{}
\renewcommand{\contentsname}{{ \centerline{目{  } 录}}}
\setCJKfamilyfont{cjkhwxk}{STXINGKA.TTF}
%\setCJKfamilyfont{cjkhwxk}{华文行楷}
%\setCJKfamilyfont{cjkfzcs}{方正粗宋简体}
%\newcommand*{\cjkfzcs}{\CJKfamily{cjkfzcs}}
\newcommand*{\cjkhwxk}{\CJKfamily{cjkhwxk}}
%\newfontfamily\wryh{Microsoft YaHei}
%\newfontfamily\hwzs{华文中宋}
%\newfontfamily\hwst{华文宋体}
%\newfontfamily\hwfs{华文仿宋}
%\newfontfamily\jljt{方正静蕾简体}
%\newfontfamily\hwxk{华文行楷}
\newcommand{\verylarge}{\fontsize{60pt}{\baselineskip}\selectfont}  
\newcommand{\chuhao}{\fontsize{44.9pt}{\baselineskip}\selectfont}  
\newcommand{\xiaochu}{\fontsize{38.5pt}{\baselineskip}\selectfont}  
\newcommand{\yihao}{\fontsize{27.8pt}{\baselineskip}\selectfont}  
\newcommand{\xiaoyi}{\fontsize{25.7pt}{\baselineskip}\selectfont}  
\newcommand{\erhao}{\fontsize{23.5pt}{\baselineskip}\selectfont}  
\newcommand{\xiaoerhao}{\fontsize{19.3pt}{\baselineskip}\selectfont} 
\newcommand{\sihao}{\fontsize{14pt}{\baselineskip}\selectfont}      % 字号设置  
\newcommand{\xiaosihao}{\fontsize{12pt}{\baselineskip}\selectfont}  % 字号设置  
\newcommand{\wuhao}{\fontsize{10.5pt}{\baselineskip}\selectfont}    % 字号设置  
\newcommand{\xiaowuhao}{\fontsize{9pt}{\baselineskip}\selectfont}   % 字号设置  
\newcommand{\liuhao}{\fontsize{7.875pt}{\baselineskip}\selectfont}  % 字号设置  
\newcommand{\qihao}{\fontsize{5.25pt}{\baselineskip}\selectfont}    % 字号设置 

\usepackage{diagbox}
\usepackage{multirow}
\boldmath
\XeTeXlinebreaklocale "zh"
\XeTeXlinebreakskip = 0pt plus 1pt minus 0.1pt
\definecolor{cred}{rgb}{0.8,0.8,0.8}
\definecolor{cgreen}{rgb}{0,0.3,0}
\definecolor{cpurple}{rgb}{0.5,0,0.35}
\definecolor{cdocblue}{rgb}{0,0,0.3}
\definecolor{cdark}{rgb}{0.95,1.0,1.0}
\lstset{
	language=bash,
	numbers=left,
	numberstyle=\tiny\color{black},
	showspaces=false,
	showstringspaces=false,
	basicstyle=\scriptsize,
	keywordstyle=\color{purple},
	commentstyle=\itshape\color{cgreen},
	stringstyle=\color{blue},
	frame=lines,
	% escapeinside=``,
	extendedchars=true, 
	xleftmargin=1em,
	xrightmargin=1em, 
	backgroundcolor=\color{cred},
	aboveskip=1em,
	breaklines=true,
	tabsize=4
} 

%\newfontfamily{\consolas}{Consolas}
%\newfontfamily{\monaco}{Monaco}
%\setmonofont[Mapping={}]{Consolas}	%英文引号之类的正常显示,相当于设置英文字体
%\setsansfont{Consolas} %设置英文字体 Monaco, Consolas,  Fantasque Sans Mono
%\setmainfont{Times New Roman}
%\setCJKmainfont{STZHONGS.TTF}
%\setmonofont{Consolas}
% \newfontfamily{\consolas}{YaHeiConsolas.ttf}
\newfontfamily{\monaco}{MONACO.TTF}
\setCJKmainfont{STZHONGS.TTF}
%\setmainfont{MONACO.TTF}
%\setsansfont{MONACO.TTF}

\newcommand{\fic}[1]{\begin{figure}[H]
		\center
		\includegraphics[width=0.8\textwidth]{#1}
	\end{figure}}
	
\newcommand{\sizedfic}[2]{\begin{figure}[H]
		\center
		\includegraphics[width=#1\textwidth]{#2}
	\end{figure}}

\newcommand{\codefile}[1]{\lstinputlisting{#1}}

\newcommand{\interval}{\vspace{0.5em}}

\newcommand{\tablestart}{
	\interval
	\begin{longtable}{p{2cm}p{10cm}}
	\hline}
\newcommand{\tableend}{
	\hline
	\end{longtable}
	\interval}

% 改变段间隔
\setlength{\parskip}{0.2em}
\linespread{1.1}

\usepackage{lastpage}
\usepackage{fancyhdr}
\pagestyle{fancy}
\lhead{\space \qquad \space}
\chead{立会\qquad}
\rhead{\qquad\thepage/\pageref{LastPage}}

\begin{document}

\tableofcontents

\clearpage

\subsection{一周目标}
	\begin{itemize}
        \item[1.]威光:解决虚拟机时常获取不到ip问题 
		\item[2.]伟琪:info升级方案完善
		\item[3.]明星:熟悉并接手网盘部署,安装
		\item[4.]李比:ui界面,area,历史数据统计
		\item[5.]吴芮:调研混合云,提出一些想法
		\item[6.]文枫:完成恒天云前台环境搭建
		\item[7.]艺弥:恒天云3.8.1 release,测试bug,完成80%
		\item[8.]瑜静:修改恒天云3.8.1bug
		\item[9.]江斌:修改恒天云3.8.1bug
		\item[10.]王宇:完成ipv6后续工作
		\item[11.]小军:恒天云3.8.1 release,测试bug,完成80%
		\item[12.]建鹏:
		\item[13.]修琳:完成快照优化
    \end{itemize}
\subsection{周三}
    \begin{itemize}
       \item[1.]威光:分析日志,查找虚拟机获取不到ip的原因
		\item[2.]伟琪:写无锡同步电子的文档
		\item[3.]明星:对照网盘文档,学习网盘的操作
		\item[4.]李比:日志系统bug修改,以及功能完善
		\item[5.]吴芮:使用阿里云的sdk,api,进行小实验
		\item[6.]文枫:django开发blog,完成部分
		\item[7.]艺弥:测试恒天云3.8.1bug
		\item[8.]瑜静:修改恒天云报警bug
		\item[9.]江斌:修改恒天云3.8.1bug
		\item[10.]王宇:
		\item[11.]小军:测试恒天云3.8.1bug
		\item[12.]建鹏:恒天云3.8.1和3.8.2,releasse细分
		\item[13.]修琳:
    \end{itemize}
\subsection{周四}
    \begin{itemize}
        \item[1.]威光:分析日志,查看是否因为port vlan tag丢失导致虚拟机获取不到ip
		\item[2.]伟琪:
		\item[3.]明星:萧山电厂演示网盘使用
		\item[4.]李比:完善日志系统功能
		\item[5.]吴芮:开通2个阿里云服务
		\item[6.]文枫:恒天云horizon环境搭建
		\item[7.]艺弥:测恒天云3.8.1的bug
		\item[8.]瑜静:改恒天云3.8.1的bug
		\item[9.]江斌:改恒天云3.8.1的bug
		\item[10.]王宇:
		\item[11.]小军:
		\item[12.]建鹏:
		\item[13.]修琳:完善快照优化
    \end{itemize}
\subsection{周五}
	\begin{itemize}
		\item[1.]威光:通过查找资料,将命名空间经常出错的问题解决掉了 100%
		\item[2.]伟琪:调研info升级方案 0%
		\item[3.]明星:萧山电厂网盘维护 80%
		\item[4.]李比:elk界面展示完成 90%
		\item[5.]吴芮:确定混合云的容灾方案
		\item[6.]文枫:
		\item[7.]艺弥:测3.8.1bug
		\item[8.]瑜静:改3.8.1bug
		\item[9.]江斌:改3.8.1bug
		\item[10.]王宇:ipv6
		\item[11.]小军:
		\item[12.]建鹏:
		\item[13.]修琳:
	\end{itemize}
\subsection{二周目标}
	\begin{itemize}
        \item[1.]威光: 给出具体的网络模式,并解释为什么
		\item[2.]伟琪: 测试和统计info平台信息
		\item[3.]明星: 将ironic在集成neutron的恒天云上跑起来
		\item[4.]李比: 错误日志具体内容展示
		\item[5.]吴芮: 使用阿里云测试不同操作镜像
		\item[6.]文枫: 恒天云菜单栏添加左右导航
		\item[7.]艺弥: 20号之前测完3.8.1
		\item[8.]瑜静: 20号之前改完3.8.1
		\item[9.]江斌: 20号之前改完3.8.1
		\item[10.]王宇:ipv6物理机测试,包管理解决方案
		\item[11.]小军:20号之前测完3.8.1
		\item[12.]建鹏:
		\item[13.]修琳:完成glance代码的上传
    \end{itemize}
\subsection{周二}
    \begin{itemize}
       \item[1.]威光:上传网络界面工作流修改代码
		\item[2.]伟琪:升级测试
		\item[3.]明星:用DD安装集成neutron的恒天云
		\item[4.]李比:完成日志系统的日志数量,错误日志及其详细图
		\item[5.]吴芮:使用阿里云接口进行调用测试
		\item[6.]文枫:学习template
		\item[7.]艺弥:完成3.8.1第二轮测试
		\item[8.]瑜静:3.8.1第二轮bug修改
		\item[9.]江斌:3.8.1第二轮bug修改
		\item[10.]王宇:发现ipv6不同原因
		\item[11.]小军:3.8.1第二轮bug测试
		\item[12.]建鹏:恒天云3.8.1和3.8.2,releasse细分
		\item[13.]修琳:使用pycharm出现问题
    \end{itemize}
\subsection{周三}
    \begin{itemize}
       \item[1.]威光:协助用DD安装开发裸金属的环境,并学习iptables,以及学习三层网络的实现原理
		\item[2.]伟琪:完善测试目录,帮inof解决快照升级失败问题
		\item[3.]明星:去味全统计服务器信息
		\item[4.]李比:后台数据太多,需前端协助在前台显示
		\item[5.]吴芮:在阿里云测试ubuntu镜像
		\item[6.]文枫:学习恒天云代码
		\item[7.]艺弥:完成3.8.1第二轮测试
		\item[8.]瑜静:3.8.1第二轮bug修改完成
		\item[9.]江斌:3.8.1第二轮bug修改完成并优化输入框限制
		\item[10.]王宇:修改包管理方案
		\item[11.]小军:3.8.1第二轮bug测试完成
		\item[12.]建鹏:
		\item[13.]修琳:控制节点重装,上传nova代码
    \end{itemize}
\subsection{周四}
    \begin{itemize}
       \item[1.]威光:neutron后台fail-report-state暂未解决,前台界面网络的基本操作调通
		\item[2.]伟琪:测试了info的一种升级方案,是可行的。
		\item[3.]明星:画味全架构方案图,安装ironic
		\item[4.]李比:错误日志详细展示页面完成
		\item[5.]吴芮:在阿里云测试windows server 2012镜像
		\item[6.]文枫:学习恒天云代码
		\item[7.]艺弥:测ceph的dhcp
		\item[8.]瑜静:帮助解决客户方监控问题
		\item[9.]江斌:协助李比解决前台展示日志系统信息,修改云硬盘Qos限制问题
		\item[10.]王宇:测ipv6物理环境
		\item[11.]小军:测ceph的dhcp
		\item[12.]建鹏:查看热迁移失败问题
		\item[13.]修琳:检查md5值
    \end{itemize}
\subsection{周五}
    \begin{itemize}
       \item[1.]威光:看代码,修改虚拟机无法绑定浮动ip
		\item[2.]伟琪:测试完成,进行环境调研
		\item[3.]明星:继续安装裸金属
		\item[4.]李比:日志界面展示初步完成,又有新的需求
		\item[5.]吴芮:请假
		\item[6.]文枫:学习恒天云代码
		\item[7.]艺弥:测vlan的floating ip和fix ip测试完成60%
		\item[8.]瑜静:测试windows server 2008,调用libvirt拿内存
		\item[9.]江斌:验证Qos,设置上限限制,在本地有效果
		\item[10.]王宇:调吴江致远环境
		\item[11.]小军:测vlan的floating ip和fix ip测试完成60%
		\item[12.]建鹏:写技术评估
		\item[13.]修琳:完成快照优化,优化效果缩短40%
    \end{itemize}



\end{document}