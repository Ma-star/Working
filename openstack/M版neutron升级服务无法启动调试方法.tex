% !TeX spellcheck = en_US
%% 字体:方正静蕾简体
%%		 方正粗宋
\documentclass[a4paper,left=1.5cm,right=1.5cm,11pt]{article}

\usepackage[utf8]{inputenc}
\usepackage{fontspec}
\usepackage{cite}
\usepackage{xeCJK}
\usepackage{indentfirst}
\usepackage{titlesec}
\usepackage{etoolbox}%
\makeatletter
\patchcmd{\ttlh@hang}{\parindent\z@}{\parindent\z@\leavevmode}{}{}%
\patchcmd{\ttlh@hang}{\noindent}{}{}{}%
\makeatother

\usepackage{longtable}
\usepackage{empheq}
\usepackage{graphicx}
\usepackage{float}
\usepackage{rotating}
\usepackage{subfigure}
\usepackage{tabu}
\usepackage{amsmath}
\usepackage{setspace}
\usepackage{amsfonts}
\usepackage{appendix}
\usepackage{listings}
\usepackage{xcolor}
\usepackage{geometry}
\setcounter{secnumdepth}{4}
%\titleformat*{\section}{\LARGE}
%\renewcommand\refname{参考文献}
%\titleformat{\chapter}{\centering\bfseries\huge}{}{0.7em}{}{}
\titleformat{\section}{\LARGE\bf}{\thesection}{1em}{}{}
\titleformat{\subsection}{\Large\bfseries}{\thesubsection}{1em}{}{}
\titleformat{\subsubsection}{\large\bfseries}{\thesubsubsection}{1em}{}{}
\renewcommand{\contentsname}{{ \centerline{目{  } 录}}}
\setCJKfamilyfont{cjkhwxk}{STXINGKA.TTF}
%\setCJKfamilyfont{cjkhwxk}{华文行楷}
%\setCJKfamilyfont{cjkfzcs}{方正粗宋简体}
%\newcommand*{\cjkfzcs}{\CJKfamily{cjkfzcs}}
\newcommand*{\cjkhwxk}{\CJKfamily{cjkhwxk}}
%\newfontfamily\wryh{Microsoft YaHei}
%\newfontfamily\hwzs{华文中宋}
%\newfontfamily\hwst{华文宋体}
%\newfontfamily\hwfs{华文仿宋}
%\newfontfamily\jljt{方正静蕾简体}
%\newfontfamily\hwxk{华文行楷}
\newcommand{\verylarge}{\fontsize{60pt}{\baselineskip}\selectfont}  
\newcommand{\chuhao}{\fontsize{44.9pt}{\baselineskip}\selectfont}  
\newcommand{\xiaochu}{\fontsize{38.5pt}{\baselineskip}\selectfont}  
\newcommand{\yihao}{\fontsize{27.8pt}{\baselineskip}\selectfont}  
\newcommand{\xiaoyi}{\fontsize{25.7pt}{\baselineskip}\selectfont}  
\newcommand{\erhao}{\fontsize{23.5pt}{\baselineskip}\selectfont}  
\newcommand{\xiaoerhao}{\fontsize{19.3pt}{\baselineskip}\selectfont} 
\newcommand{\sihao}{\fontsize{14pt}{\baselineskip}\selectfont}      % 字号设置  
\newcommand{\xiaosihao}{\fontsize{12pt}{\baselineskip}\selectfont}  % 字号设置  
\newcommand{\wuhao}{\fontsize{10.5pt}{\baselineskip}\selectfont}    % 字号设置  
\newcommand{\xiaowuhao}{\fontsize{9pt}{\baselineskip}\selectfont}   % 字号设置  
\newcommand{\liuhao}{\fontsize{7.875pt}{\baselineskip}\selectfont}  % 字号设置  
\newcommand{\qihao}{\fontsize{5.25pt}{\baselineskip}\selectfont}    % 字号设置 

\usepackage{diagbox}
\usepackage{multirow}
\boldmath
\XeTeXlinebreaklocale "zh"
\XeTeXlinebreakskip = 0pt plus 1pt minus 0.1pt
\definecolor{cred}{rgb}{0.8,0.8,0.8}
\definecolor{cgreen}{rgb}{0,0.3,0}
\definecolor{cpurple}{rgb}{0.5,0,0.35}
\definecolor{cdocblue}{rgb}{0,0,0.3}
\definecolor{cdark}{rgb}{0.95,1.0,1.0}
\lstset{
	language=bash,
	numbers=left,
	numberstyle=\tiny\color{black},
	showspaces=false,
	showstringspaces=false,
	basicstyle=\scriptsize,
	keywordstyle=\color{purple},
	commentstyle=\itshape\color{cgreen},
	stringstyle=\color{blue},
	frame=lines,
	% escapeinside=``,
	extendedchars=true, 
	xleftmargin=1em,
	xrightmargin=1em, 
	backgroundcolor=\color{cred},
	aboveskip=1em,
	breaklines=true,
	tabsize=4
} 

%\newfontfamily{\consolas}{Consolas}
%\newfontfamily{\monaco}{Monaco}
%\setmonofont[Mapping={}]{Consolas}	%英文引号之类的正常显示,相当于设置英文字体
%\setsansfont{Consolas} %设置英文字体 Monaco, Consolas,  Fantasque Sans Mono
%\setmainfont{Times New Roman}
%\setCJKmainfont{STZHONGS.TTF}
%\setmonofont{Consolas}
% \newfontfamily{\consolas}{YaHeiConsolas.ttf}
\newfontfamily{\monaco}{MONACO.TTF}
\setCJKmainfont{STZHONGS.TTF}
%\setmainfont{MONACO.TTF}
%\setsansfont{MONACO.TTF}

\newcommand{\fic}[1]{\begin{figure}[H]
		\center
		\includegraphics[width=0.8\textwidth]{#1}
	\end{figure}}
	
\newcommand{\sizedfic}[2]{\begin{figure}[H]
		\center
		\includegraphics[width=#1\textwidth]{#2}
	\end{figure}}

\newcommand{\codefile}[1]{\lstinputlisting{#1}}

\newcommand{\interval}{\vspace{0.5em}}

\newcommand{\tablestart}{
	\interval
	\begin{longtable}{p{2cm}p{10cm}}
	\hline}
\newcommand{\tableend}{
	\hline
	\end{longtable}
	\interval}

% 改变段间隔
\setlength{\parskip}{0.2em}
\linespread{1.1}

\usepackage{lastpage}
\usepackage{fancyhdr}
\pagestyle{fancy}
\lhead{\space \qquad \space}
\chead{调试方法\qquad}
\rhead{\qquad\thepage/\pageref{LastPage}}

\begin{document}

\tableofcontents

\clearpage
\section{调试方法} 
\subsection{openstack的包}
    \begin{itemize}
        \item[1.]/usr/lib/python2.7/dist-packages  安装包所在路经
		\item[2.]由于服务启动不了,在日志文件看不到错误信息,需要使用ps命令
		\item[3.]ps -aux | grep 在正常运行的机器运行该命令,会看到进程运行的命令。复制该命令调试服务
		\item[4.]/usr/bin/python /usr/bin/nova-api --config-file=/etc/nova/nova.conf 执行此命令可以看到错误信息
		\item[5.]vi /usr/lib/python2.7/dist-packages/oslo/config/cfg.py 查看引起错误的页面进行调整
    \end{itemize}
\subsection{openstack各服务启动,并能查看错误信息}
	\begin{itemize}
		\item[1.]查看keystone:python /usr/bin/keystone-all
		\item[2.]查看glance-api:/usr/bin/python /usr/bin/glance-api		
		\item[3.]查看glance-registry:/usr/bin/python /usr/bin/glance-registry 
		\item[4.]查看nova-api:/usr/bin/python /usr/bin/nova-api --config-file=/etc/nova/nova.conf
		\item[5.]查看nova-cert:/usr/bin/python /usr/bin/nova-cert --config-file=/etc/nova/nova.conf
		\item[6.]查看nova-conductor:/usr/bin/python /usr/bin/nova-conductor --config-file=/etc/nova/nova.conf
		\item[7.]查看nova-scheduler:/usr/bin/python /usr/bin/nova-scheduler --config-file=/etc/nova/nova.conf
		\item[8.]查看nova-consoleauth:/usr/bin/python /usr/bin/nova-consoleauth --config-file=/etc/nova/nova.conf
		\item[9.]查看nova-novncproxy:/usr/bin/python /usr/bin/nova-novncproxy --config-file=/etc/nova/nova.conf
	\end{itemize}
\subsection{glance-registry修改的文件列表如下}
	\begin{itemize}
		\item[1.]/usr/lib/python2.7/dist-packages/glance/cmd/__init__.py
		\item[2.]/usr/lib/python2.7/dist-packages/glance/i18n.py
		\item[3.]/usr/lib/python2.7/dist-packages/glance/cmd/registry.py
		\item[4.]/usr/lib/python2.7/dist-packages/glance/common/config.py
		\item[5.]/usr/lib/python2.7/dist-packages/glance/common/wsgi.py
		\item[6.]/usr/lib/python2.7/dist-packages/glance/common/utils.py
		\item[7.]/usr/lib/python2.7/dist-packages/glance/openstack/common/log.py
		\item[8.]/usr/lib/python2.7/dist-packages/glance_store/backend.py
		\item[9.]/usr/lib/python2.7/dist-packages/glance_store/i18n.py
		\item[10.]/usr/lib/python2.7/dist-packages/glance_store/driver.py
		\item[11.]/usr/lib/python2.7/dist-packages/glance/notifier.py
		\item[12.]/usr/lib/python2.7/dist-packages/glance/domain/__init__.py
		\item[13.]
		\item[14.]
		\item[15.]
		\item[16.]
		\item[17.]

	\end{itemize}
\end{document}