% !TeX spellcheck = en_US
%% 字体:方正静蕾简体
%%		 方正粗宋
\documentclass[a4paper,left=1.5cm,right=1.5cm,11pt]{article}
\title{基于恒天云3.8.1源码安装neutron}
\author{周威光整理}
\date{2017-06-28} 
\usepackage[utf8]{inputenc}
\usepackage{fontspec}
\usepackage{cite}
\usepackage{xeCJK}
\usepackage{indentfirst}
\usepackage{titlesec}
\usepackage{etoolbox}%
\makeatletter
\patchcmd{\ttlh@hang}{\parindent\z@}{\parindent\z@\leavevmode}{}{}%
\patchcmd{\ttlh@hang}{\noindent}{}{}{}%
\makeatother

\usepackage{longtable}
\usepackage{empheq}
\usepackage{graphicx}
\usepackage{float}
\usepackage{rotating}
\usepackage{subfigure}
\usepackage{tabu}
\usepackage{amsmath}
\usepackage{setspace}
\usepackage{amsfonts}
\usepackage{appendix}
\usepackage{listings}
\usepackage{xcolor}
\usepackage{geometry}
\setcounter{secnumdepth}{4}
%\titleformat*{\section}{\LARGE}
%\renewcommand\refname{参考文献}
%\titleformat{\chapter}{\centering\bfseries\huge}{}{0.7em}{}{}
\titleformat{\section}{\LARGE\bf}{\thesection}{1em}{}{}
\titleformat{\subsection}{\Large\bfseries}{\thesubsection}{1em}{}{}
\titleformat{\subsubsection}{\large\bfseries}{\thesubsubsection}{1em}{}{}
\renewcommand{\contentsname}{{ \centerline{目{  } 录}}}
\setCJKfamilyfont{cjkhwxk}{STXINGKA.TTF}
%\setCJKfamilyfont{cjkhwxk}{华文行楷}
%\setCJKfamilyfont{cjkfzcs}{方正粗宋简体}
%\newcommand*{\cjkfzcs}{\CJKfamily{cjkfzcs}}
\newcommand*{\cjkhwxk}{\CJKfamily{cjkhwxk}}
%\newfontfamily\wryh{Microsoft YaHei}
%\newfontfamily\hwzs{华文中宋}
%\newfontfamily\hwst{华文宋体}
%\newfontfamily\hwfs{华文仿宋}
%\newfontfamily\jljt{方正静蕾简体}
%\newfontfamily\hwxk{华文行楷}
\newcommand{\verylarge}{\fontsize{60pt}{\baselineskip}\selectfont}  
\newcommand{\chuhao}{\fontsize{44.9pt}{\baselineskip}\selectfont}  
\newcommand{\xiaochu}{\fontsize{38.5pt}{\baselineskip}\selectfont}  
\newcommand{\yihao}{\fontsize{27.8pt}{\baselineskip}\selectfont}  
\newcommand{\xiaoyi}{\fontsize{25.7pt}{\baselineskip}\selectfont}  
\newcommand{\erhao}{\fontsize{23.5pt}{\baselineskip}\selectfont}  
\newcommand{\xiaoerhao}{\fontsize{19.3pt}{\baselineskip}\selectfont} 
\newcommand{\sihao}{\fontsize{14pt}{\baselineskip}\selectfont}      % 字号设置  
\newcommand{\xiaosihao}{\fontsize{12pt}{\baselineskip}\selectfont}  % 字号设置  
\newcommand{\wuhao}{\fontsize{10.5pt}{\baselineskip}\selectfont}    % 字号设置  
\newcommand{\xiaowuhao}{\fontsize{9pt}{\baselineskip}\selectfont}   % 字号设置  
\newcommand{\liuhao}{\fontsize{7.875pt}{\baselineskip}\selectfont}  % 字号设置  
\newcommand{\qihao}{\fontsize{5.25pt}{\baselineskip}\selectfont}    % 字号设置 

\usepackage{diagbox}
\usepackage{multirow}
\boldmath
\XeTeXlinebreaklocale "zh"
\XeTeXlinebreakskip = 0pt plus 1pt minus 0.1pt
\definecolor{cred}{rgb}{0.8,0.8,0.8}
\definecolor{cgreen}{rgb}{0,0.3,0}
\definecolor{cpurple}{rgb}{0.5,0,0.35}
\definecolor{cdocblue}{rgb}{0,0,0.3}
\definecolor{cdark}{rgb}{0.95,1.0,1.0}
\lstset{
	language=bash,
	numbers=left,
	numberstyle=\tiny\color{black},
	showspaces=false,
	showstringspaces=false,
	basicstyle=\scriptsize,
	keywordstyle=\color{purple},
	commentstyle=\itshape\color{cgreen},
	stringstyle=\color{blue},
	frame=lines,
	% escapeinside=``,
	extendedchars=true, 
	xleftmargin=1em,
	xrightmargin=1em, 
	backgroundcolor=\color{cred},
	aboveskip=1em,
	breaklines=true,
	tabsize=4
} 

%\newfontfamily{\consolas}{Consolas}
%\newfontfamily{\monaco}{Monaco}
%\setmonofont[Mapping={}]{Consolas}	%英文引号之类的正常显示,相当于设置英文字体
%\setsansfont{Consolas} %设置英文字体 Monaco, Consolas,  Fantasque Sans Mono
%\setmainfont{Times New Roman}
%\setCJKmainfont{STZHONGS.TTF}
%\setmonofont{Consolas}
% \newfontfamily{\consolas}{YaHeiConsolas.ttf}
\newfontfamily{\monaco}{MONACO.TTF}
\setCJKmainfont{STZHONGS.TTF}
%\setmainfont{MONACO.TTF}
%\setsansfont{MONACO.TTF}

\newcommand{\fic}[1]{\begin{figure}[H]
		\center
		\includegraphics[width=0.8\textwidth]{#1}
	\end{figure}}
	
\newcommand{\sizedfic}[2]{\begin{figure}[H]
		\center
		\includegraphics[width=#1\textwidth]{#2}
	\end{figure}}

\newcommand{\codefile}[1]{\lstinputlisting{#1}}

\newcommand{\interval}{\vspace{0.5em}}

\newcommand{\tablestart}{
	\interval
	\begin{longtable}{p{2cm}p{10cm}}
	\hline}
\newcommand{\tableend}{
	\hline
	\end{longtable}
	\interval}

% 改变段间隔
\setlength{\parskip}{0.2em}
\linespread{1.1}

\usepackage{lastpage}
\usepackage{fancyhdr}
\pagestyle{fancy}
\lhead{\space \qquad \space}
\chead{基于恒天云3.8.1源码安装neutron\qquad}
\rhead{\qquad\thepage/\pageref{LastPage}}

\begin{document}
% 生成标题、作者、日期
\maketitle
\clearpage
% 生成目录
\tableofcontents
\clearpage
\section{环境准备}
\subsection{物理环境}
\begin{itemize}
	\item[(1).]两台3网卡的物理机,一台作为控制节点hty-controller,一台作为计算节点hty-compute1
	\item[(2).]提前在两台机器上用恒天云3.8.1分别安装控制节点和计算节点
	\item[(3).]每台物理机的一张网卡 eth0 访问外网,一张网卡 eth1 作为管理网络的网卡,一张网卡 eth2 作为neutron隧道网络的网卡
	\item[(4).]下面是控制节点和计算节点的网络配置:\\
	控制节点如下:
	\begin{lstlisting}
	# This file describes the network interfaces available on your system
	# and how to activate them. For more information, see interfaces(5).

	# The loopback network interface
	auto lo
	iface lo inet loopback

	# The primary network interface
	auto eth0
	iface eth0 inet static
	address 172.16.19.147
	netmask 255.255.255.0
	gateway 172.16.19.254
	dns-nameservers 172.16.5.1

	auto eth1
	iface eth1 inet static
	address 10.10.10.11
	netmask 255.255.255.0

	auto eth2
	iface eth2 inet static
	address 10.10.11.11
	netmask 255.255.255.0
	\end{lstlisting}
	计算节点如下:
	\begin{lstlisting}
	# This file describes the network interfaces available on your system
	# and how to activate them. For more information, see interfaces(5).

	# The loopback network interface
	auto lo
	iface lo inet loopback

	# The primary network interface
	auto eth0
	iface eth0 inet static
	address 172.16.19.151
	netmask 255.255.255.0
	gateway 172.16.19.254
	dns-nameservers 172.16.5.1

	auto eth1
	iface eth1 inet static
	address 10.10.10.12
	netmask 255.255.255.0

	auto eth2
	iface eth2 inet static
	address 10.10.11.12 
	netmask 255.255.255.0
	\end{lstlisting}
\end{itemize}
\subsection{软件环境}
\begin{itemize}
	\item[(1).]M版openstack的neutron源码包:由于恒天云是基于J版的openstack,安装高版本neutron会出现兼容性问题。需要在虚拟环境通过源码安装neutron
	\item[(2).]openvswitch2.4.0源码包:neutron的DVR特性需要openvswitch2.1版本以上的支持,直接apt-get安装的是低版本的。需要源码安装高版本的。
	\item[(3).]openvswitch和neutron各服务启动脚本:源码安装的服务启动比较麻烦,为了方便部署,准备了管理脚本,需将其拷贝到/etc/init.d,并注册。
\end{itemize}
\section{openvswitch和neutron源码安装}
\subsection{openvswitch2.4.0源码安装}
\begin{itemize}
    \item[1.]下载openvswitch源码 http://openvswitch.org/download/ ,选择的是openvswitch2.4.0
	\item[2.]解压安装包:tar -xzf openvswitch-2.4.0.tar.gz
	\item[3.]构建基于Linux内核的交换机,uname -r用来得到自己linux内核版本号,./configure部分可以用--prefix=参数,可以让OVS完全安装在该目录底下。\\
	命令如下:
	\begin{lstlisting}
	cd openvswitch-2.4.0
	aptitude install dh-autoreconf libssl-dev openssl          #预先安装一些库
	./configure --with-linux=/lib/modules/$(uname -r)/build
	\end{lstlisting}
	\item[4.]编译并安装OVS2.4.0\\
	命令如下:
	\begin{lstlisting}
	make & make install
	\end{lstlisting}
	\item[5.]安装并加载构建的内核模块\\
	命令如下:
	\begin{lstlisting}
	modprobe libcrc32c
	modprobe gre
	modprobe vxlan
	modprobe openvswitch
	insmod datapath/linux/openvswitch.ko
	make modules_install
    \end{lstlisting}
    此时可以通过lsmod |grep openvswitch来查看已载入系统的模块,发现有OpenvSwitch
	\item[6.]使用ovsdb工具初始化配置数据库\\
	命令如下:
	\begin{lstlisting}
	ovsdb-tool create /usr/local/etc/openvswitch/conf.db /usr/local/share/openvswitch/vswitch.ovsschema
    \end{lstlisting}
	\item[7.]将准备好的openvswitch脚本复制到/etc/init.d/,使用如下命令注册服务,并使用注册的服务启动openvswitch
	\begin{lstlisting}
	update-rc.d openvswitch defaults
	mkdir /var/log/openvswitch    #创建存放日志文件夹,否则启动会报错
	service openvswitch start 
	\end{lstlisting}
	\item[8.]手动添加一个虚拟网桥,用于neutron服务使用
	ovs-vsctl add-br br-ex
	\item[9.]向/etc/modules写入如下参数,使得开机自动加载内核模块\\
	命令如下:
	\begin{lstlisting}
	echo "openvswitch " >> /etc/modules
	echo "gre" >> /etc/modules
	echo "vxlan" >> /etc/modules
	echo "libcrc32c" >> /etc/modules
	\end{lstlisting}
\end{itemize}
\subsection{neutron源码安装}
\begin{itemize}
	\item[1.]需要安装pip版本在8.1.2之上,否则可能会报错
	\begin{lstlisting}
	apt-get install python-pip
	pip install --upgrade 'pip>=8.1.2'
	\end{lstlisting}
	\item[2.]安装virtualenv,用于创建虚拟环境
	\begin{lstlisting}
	pip install virtualenv
	\end{lstlisting}
	\item[3.]创建名为venv的虚拟环境用于安装neutron,并切换进venv目录
	\begin{lstlisting}
	virtualenv venv
	cd venv
	\end{lstlisting}
	安装之后使用ls命令,查看当前文件夹的内容,会多出来bin,lib,local等文件夹。以后安装的包会放在
	这些文件夹,将不会和系统的混合在一起。这就是隔离的意义。
	\item[4.]启动虚拟环境
	\begin{lstlisting}
	source bin/activate
	\end{lstlisting}
    \item[5.]切换到neutron源码包,安装neutron依赖包。要先安装python2.7-dev,再安装neutron依赖包,否则可能会报错
	\begin{lstlisting}
	sudo apt-get install python2.7-dev
	pip install -r requirements.txt
	\end{lstlisting}
	\item[6.]安装neutron组件,要先安装git,再安装neutron组件,否则可能会报错
	\begin{lstlisting}
	apt-get install git
	python setup.py install
	\end{lstlisting}
	\item[7.]通过源码的tools文件夹下的generate\_config\_file\_samples.sh脚本生成配置文件,并拷贝到/etc/neutron文件夹下
	ps:配置文件可以从已有的环境直接拷贝过来,可以节省许多时间
	\begin{lstlisting}
	./tools/generate_config_file_samples.sh
	\end{lstlisting}	
	\item[8.]在mysql创建neutron数据库,并使用keystone命令生成neutron的service和endpoint端点\\
	ps:只需要在控制节点执行如下操作
	\begin{lstlisting}
	#创建neutron数据库
	mysql -uroot -phtYun@2014
	CREATE DATABASE neutron;
	GRANT ALL PRIVILEGES ON neutron.* TO 'neutron'@'localhost' \
	IDENTIFIED BY 'NEUTRON_DBPASS';
	GRANT ALL PRIVILEGES ON neutron.* TO 'neutron'@'%' \
	IDENTIFIED BY 'NEUTRON_DBPASS';
    #创建neutron用户
	keystone user-create --name neutron --pass neutron_pass@2014
	keystone user-role-add --user neutron --tenant service --role admin
	#创建neutron服务
	keystone service-create --name neutron --type network \
	--description "OpenStack Networking"
    #创建neutron服务端点endpoint
	keystone endpoint-create \
	--service-id $(keystone service-list | awk '/ network / {print $2}') \
	--publicurl http://hty-controller:9696 \
	--adminurl http://hty-controller:9696 \
	--internalurl http://hty-controller:9696 \
	--region regionOne
	\end{lstlisting}
    \item[9.]修改/etc/nova/nova.conf,主要是注释掉nova-network的配置,加入neutron配置,其余部分不用修改。
	控制节点参考172.16.19.147,计算节点参考172.16.19.151。\\
	nova.conf文件修改如下:
	\begin{lstlisting}
	[DEFAULT]
	...
	
	#下面注释nova-network的配置
	#network_api_class = nova.network.api.API
	#security_group_api = nova
	#firewall_driver =nova.virt.libvirt.firewall.IptablesFirewallDriver
	#firewall_driver = nova.virt.firewall.NoopFirewallDriver

	#network_manager = nova.network.manager.FlatDHCPManager
	#network_manager = nova.network.manager.VlanManager

	#启用neutron配置
	network_api_class = nova.network.neutronv2.api.API
	security_group_api = neutron
	linuxnet_interface_driver = nova.network.linux_net.LinuxOVSInterfaceDriver
	firewall_driver = nova.virt.firewall.NoopFirewallDriver

	use_neutron = True
	...

	[glance] #保持原有配置
	...

	[keystone_authtoken] #保持原有配置
	...
	
	[neutron] #新增
	url = http://hty-controller:9696
	auth_strategy = keystone
	admin_auth_url = http://hty-controller:35357/v2.0
	admin_tenant_name = service
	admin_username = neutron
	admin_password = neutron_pass@2014

	[database] #保持原有配置
	...

	[libvirt]  #保持原有配置
	...
	\end{lstlisting}
	上面配置可能会有遗漏,建议参考172.16.19.147
	\item[10.]复制/etc/neutron文件夹下的各服务配置文件,控制节点复制172.16.19.147,计算节点复制172.16.19.151。这里不再赘述。\\
	ps:也可以通过源码脚本生成各配置文件,参考官方neutron配置,手动配置。会比较麻烦。
	\item[11.]如果在步骤10直接拷贝的,此处可以忽略。如果手动参照官方配置,需要注意控制节点/etc/neutron/rootwrap.d/l3.filters和dhcp.filters这两个文件
	这是由于使用了虚拟环境的缘故
	\begin{lstlisting}
	#将文件中的下面一行
	kill_metadata: KillFilter, root, python, -9
	#改为
	kill_metadata: KillFilter, root, /root/venv/bin/python, -9
	\end{lstlisting}
\end{itemize}
\subsection{额外软件包安装}
neutron源码安装并不能将服务依赖的其他软件包全部下载,还需要手动下载。下载如下包:
\begin{lstlisting}
pip install python-memcached
pip install pymysql
apt-get install arping 
ln -s /root/venv/bin/neutron-ns-metadata-proxy /usr/bin/
apt-get install conntrack
apt-get install dnsmasq-utils
apt-get install ipset
\end{lstlisting}
\subsection{兼容neutron恒天云代码修改部分}
\begin{itemize}
	\item[1.]修改neutron/common/config.py文件
	\begin{lstlisting}
	vi /root/venv/local/lib/python2.7/site-packages/neutron/common/config.py
	#修改config.py文件中的如下函数
	def set_db_defaults():
    # Update the default QueuePool parameters. These can be tweaked by the
    # conf variables - max_pool_size, max_overflow and pool_timeout
    db_options.set_defaults(
        cfg.CONF,
        connection='sqlite://',
        #sqlite_db='', max_pool_size=10, #注释掉这一行
        max_pool_size=10,                #添加这一行
        max_overflow=20, pool_timeout=10)
	\end{lstlisting}
	\item[2.]修改servers.py文件
	\begin{lstlisting}
	vi /usr/lib/python2.7/dist-packages/nova/api/openstack/compute/servers.py
	#搜索关键字floating_ips修改floating_ips的语句如下:
	floating_ips = body['server'].get('fips', None)
	\end{lstlisting}
	\item[3.]修改neutron/agent/common/config.py文件
	\begin{lstlisting}
	vi /root/venv/lib/python2.7/site-packages/neutron/agent/common/config.py
	#搜索关键字verbose修改verbose的条件语句如下:
	def get_log_args(conf, log_file_name, **kwargs):
    ...
    #if conf.verbose:        #注释这一行
    if getattr(conf, "verbose", False): #新增这一行
	...
	\end{lstlisting}
\end{itemize}
\subsection{根据neutron配置文件填充数据库}
\begin{lstlisting}
neutron-db-manage --config-file /etc/neutron/neutron.conf --config-file /etc/neutron/plugins/ml2/ml2_conf.ini upgrade head
\end{lstlisting}
\subsection{配置外网网卡桥接}
租户网络内的vm若要访问外网,需要将外网网卡桥接到br-ex \\
有如下两部分修改
\begin{itemize}
	\item[1.]命令行进行如下操作:
	\begin{lstlisting}
	ip link set eth0 address br-ex的mac
	ovs-vsctl add-port br-ex eth0
	\end{lstlisting}
	\item[2.]修改网卡配置如下
	控制节点如下:
	\begin{lstlisting}
	# This file describes the network interfaces available on your system
	# and how to activate them. For more information, see interfaces(5).

	# The loopback network interface
	auto lo
	iface lo inet loopback

	# The primary network interface
	#auto eth0
	#iface eth0 inet static
	#address 172.16.19.147
	#netmask 255.255.255.0
	#gateway 172.16.19.254
	#dns-nameservers 172.16.5.1

	auto br-ex
	iface br-ex inet static
	address 172.16.19.147
	netmask 255.255.255.0
	gateway 172.16.19.254
	dns-nameservers 8.8.8.8

	auto eth0
	iface eth0 inet manual
	up ifconfig $IFACE 0.0.0.0 up
	up ip link set $IFACE promisc on
	down ip link set $IFACE promisc off
	down ifconfig $IFACE down

	auto eth1
	iface eth1 inet static
	address 10.10.10.11
	netmask 255.255.255.0

	auto eth2
	iface eth2 inet static
	address 10.10.11.11
	netmask 255.255.255.0
	\end{lstlisting}
	计算节点如下:
	\begin{lstlisting}
	# This file describes the network interfaces available on your system
	# and how to activate them. For more information, see interfaces(5).

	# The loopback network interface
	auto lo
	iface lo inet loopback

	# The primary network interface
	#auto eth0
	#iface eth0 inet static
	#address 172.16.19.151
	#netmask 255.255.255.0
	#gateway 172.16.19.254
	#dns-nameservers 172.16.5.1

	auto br-ex
	iface br-ex inet static
	address 172.16.19.151
	netmask 255.255.255.0
	gateway 172.16.19.254
	dns-nameservers 8.8.8.8

	auto eth0
	iface eth0 inet manual
	up ifconfig $IFACE 0.0.0.0 up
	up ip link set $IFACE promisc on
	down ip link set $IFACE promisc off
	down ifconfig $IFACE down

	auto eth1
	iface eth1 inet static
	address 10.10.10.12
	netmask 255.255.255.0

	auto eth2
	iface eth2 inet static
	address 10.10.11.12 
	netmask 255.255.255.0
	\end{lstlisting}
\end{itemize}
更改网络配置,重启网卡,会出现外网无法访问
\subsection{注册neutron各服务管理脚本并启动各服务}
\begin{lstlisting}
#复制脚本到/etc/init.d/
cp neutron-server neutron-openvswitch-agent neutron-l3-agent neutron-dhcp-agent neutron-metadata-agent /etc/init.d/
#注册服务
update-rc.d neutron-server defaults
update-rc.d neutron-openvswitch-agent defaults
update-rc.d neutron-l3-agent defaults
update-rc.d neutron-dhcp-agent defaults
update-rc.d neutron-metadata-agent defaults
#启动服务,控制节点全部启动,计算节点只需要启动
#neutron-openvswitch-agent,neutron-l3-agent,neutron-metadata-agent
mkdir /var/log/neutron #创建neutron日志文件夹,否则启动会报错
service neutron-server start
service neutron-openvswitch-agent start
service neutron-l3-agent start
service neutron-dhcp-agent start
service neutron-metadata-agent start
\end{lstlisting}			
\end{document}