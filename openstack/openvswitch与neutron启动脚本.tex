% !TeX spellcheck = en_US
%% 字体:方正静蕾简体
%%		 方正粗宋
\documentclass[a4paper,left=1.5cm,right=1.5cm,11pt]{article}

\usepackage[utf8]{inputenc}
\usepackage{fontspec}
\usepackage{cite}
\usepackage{xeCJK}
\usepackage{indentfirst}
\usepackage{titlesec}
\usepackage{etoolbox}%
\makeatletter
\patchcmd{\ttlh@hang}{\parindent\z@}{\parindent\z@\leavevmode}{}{}%
\patchcmd{\ttlh@hang}{\noindent}{}{}{}%
\makeatother

\usepackage{longtable}
\usepackage{empheq}
\usepackage{graphicx}
\usepackage{float}
\usepackage{rotating}
\usepackage{subfigure}
\usepackage{tabu}
\usepackage{amsmath}
\usepackage{setspace}
\usepackage{amsfonts}
\usepackage{appendix}
\usepackage{listings}
\usepackage{xcolor}
\usepackage{geometry}
\setcounter{secnumdepth}{4}
%\titleformat*{\section}{\LARGE}
%\renewcommand\refname{参考文献}
%\titleformat{\chapter}{\centering\bfseries\huge}{}{0.7em}{}{}
\titleformat{\section}{\LARGE\bf}{\thesection}{1em}{}{}
\titleformat{\subsection}{\Large\bfseries}{\thesubsection}{1em}{}{}
\titleformat{\subsubsection}{\large\bfseries}{\thesubsubsection}{1em}{}{}
\renewcommand{\contentsname}{{ \centerline{目{  } 录}}}
\setCJKfamilyfont{cjkhwxk}{STXINGKA.TTF}
%\setCJKfamilyfont{cjkhwxk}{华文行楷}
%\setCJKfamilyfont{cjkfzcs}{方正粗宋简体}
%\newcommand*{\cjkfzcs}{\CJKfamily{cjkfzcs}}
\newcommand*{\cjkhwxk}{\CJKfamily{cjkhwxk}}
%\newfontfamily\wryh{Microsoft YaHei}
%\newfontfamily\hwzs{华文中宋}
%\newfontfamily\hwst{华文宋体}
%\newfontfamily\hwfs{华文仿宋}
%\newfontfamily\jljt{方正静蕾简体}
%\newfontfamily\hwxk{华文行楷}
\newcommand{\verylarge}{\fontsize{60pt}{\baselineskip}\selectfont}  
\newcommand{\chuhao}{\fontsize{44.9pt}{\baselineskip}\selectfont}  
\newcommand{\xiaochu}{\fontsize{38.5pt}{\baselineskip}\selectfont}  
\newcommand{\yihao}{\fontsize{27.8pt}{\baselineskip}\selectfont}  
\newcommand{\xiaoyi}{\fontsize{25.7pt}{\baselineskip}\selectfont}  
\newcommand{\erhao}{\fontsize{23.5pt}{\baselineskip}\selectfont}  
\newcommand{\xiaoerhao}{\fontsize{19.3pt}{\baselineskip}\selectfont} 
\newcommand{\sihao}{\fontsize{14pt}{\baselineskip}\selectfont}      % 字号设置  
\newcommand{\xiaosihao}{\fontsize{12pt}{\baselineskip}\selectfont}  % 字号设置  
\newcommand{\wuhao}{\fontsize{10.5pt}{\baselineskip}\selectfont}    % 字号设置  
\newcommand{\xiaowuhao}{\fontsize{9pt}{\baselineskip}\selectfont}   % 字号设置  
\newcommand{\liuhao}{\fontsize{7.875pt}{\baselineskip}\selectfont}  % 字号设置  
\newcommand{\qihao}{\fontsize{5.25pt}{\baselineskip}\selectfont}    % 字号设置 

\usepackage{diagbox}
\usepackage{multirow}
\boldmath
\XeTeXlinebreaklocale "zh"
\XeTeXlinebreakskip = 0pt plus 1pt minus 0.1pt
\definecolor{cred}{rgb}{0.8,0.8,0.8}
\definecolor{cgreen}{rgb}{0,0.3,0}
\definecolor{cpurple}{rgb}{0.5,0,0.35}
\definecolor{cdocblue}{rgb}{0,0,0.3}
\definecolor{cdark}{rgb}{0.95,1.0,1.0}
\lstset{
	language=bash,
	numbers=left,
	numberstyle=\tiny\color{black},
	showspaces=false,
	showstringspaces=false,
	basicstyle=\scriptsize,
	keywordstyle=\color{purple},
	commentstyle=\itshape\color{cgreen},
	stringstyle=\color{blue},
	frame=lines,
	% escapeinside=``,
	extendedchars=true, 
	xleftmargin=1em,
	xrightmargin=1em, 
	backgroundcolor=\color{cred},
	aboveskip=1em,
	breaklines=true,
	tabsize=4
} 

%\newfontfamily{\consolas}{Consolas}
%\newfontfamily{\monaco}{Monaco}
%\setmonofont[Mapping={}]{Consolas}	%英文引号之类的正常显示,相当于设置英文字体
%\setsansfont{Consolas} %设置英文字体 Monaco, Consolas,  Fantasque Sans Mono
%\setmainfont{Times New Roman}
%\setCJKmainfont{STZHONGS.TTF}
%\setmonofont{Consolas}
% \newfontfamily{\consolas}{YaHeiConsolas.ttf}
\newfontfamily{\monaco}{MONACO.TTF}
\setCJKmainfont{STZHONGS.TTF}
%\setmainfont{MONACO.TTF}
%\setsansfont{MONACO.TTF}

\newcommand{\fic}[1]{\begin{figure}[H]
		\center
		\includegraphics[width=0.8\textwidth]{#1}
	\end{figure}}
	
\newcommand{\sizedfic}[2]{\begin{figure}[H]
		\center
		\includegraphics[width=#1\textwidth]{#2}
	\end{figure}}

\newcommand{\codefile}[1]{\lstinputlisting{#1}}

\newcommand{\interval}{\vspace{0.5em}}

\newcommand{\tablestart}{
	\interval
	\begin{longtable}{p{2cm}p{10cm}}
	\hline}
\newcommand{\tableend}{
	\hline
	\end{longtable}
	\interval}

% 改变段间隔
\setlength{\parskip}{0.2em}
\linespread{1.1}

\usepackage{lastpage}
\usepackage{fancyhdr}
\pagestyle{fancy}
\lhead{\space \qquad \space}
\chead{pdb调试python源码\qquad}
\rhead{\qquad\thepage/\pageref{LastPage}}

\begin{document}

\tableofcontents

\clearpage

\subsection{openvswitch启动脚本}
# !/bin/bash

# chkconfig:  - 85 15  

# description: openvswitch  service

#processname:openvswitch


case "$1" in

  start)
        echo "Starting openvswitch..."
        start-stop-daemon -q -S -x /usr/local/sbin/ovsdb-server -- --remote=punix:/usr/local/var/run/openvswitch/db.sock --remote=db:Open_vSwitch,Open_vSwitch,manager_options --pidfile --detach --log-file=/var/log/openvswitch/ovsdb-server.log
        sleep 3
        start-stop-daemon -q -S -x /usr/local/bin/ovs-vsctl  -- --no-wait init
        start-stop-daemon -q -S -x /usr/local/sbin/ovs-vswitchd -- --pidfile --detach --log-file=/var/log/openvswitch/ovs-vswitchd.log
        ;;
  stop)
        echo "Stopping openvswitch..."
        start-stop-daemon -q -K -x /usr/local/sbin/ovsdb-server
        start-stop-daemon -q -K -x /usr/local/bin/ovs-vsctl
        start-stop-daemon -q -K -x /usr/local/sbin/ovs-vswitchd
        ;;

  restart)
        echo "Stopping openvswitch..."
        start-stop-daemon -q -K -x /usr/local/sbin/ovsdb-server
        start-stop-daemon -q -K -x /usr/local/bin/ovs-vsctl
        start-stop-daemon -q -K -x /usr/local/sbin/ovs-vswitchd
        echo "Starting openvswitch..."
        start-stop-daemon -q -S -x /usr/local/sbin/ovsdb-server -- --remote=punix:/usr/local/var/run/openvswitch/db.sock --remote=db:Open_vSwitch,Open_vSwitch,manager_options --pidfile --detach --log-file=/var/log/openvswitch/ovsdb-server.log
        sleep 3
        start-stop-daemon -q -S -x /usr/local/bin/ovs-vsctl  -- --no-wait init
        start-stop-daemon -q -S -x /usr/local/sbin/ovs-vswitchd --  --pidfile --detach --log-file=/var/log/openvswitch/ovs-vswitchd.log
        ;;

  *)
        echo "Usage: $0 {start|stop|restart}"
        exit 1
        ;;
esac

exit 0 
\subsection{neutron-server启动脚本}
#!/bin/bash

# chkconfig:  - 85 15  

# description: neutron-server  service

#processname:neutron-server

if [ ! -d /var/run/neutron/ ]; then
   sudo mkdir /var/run/neutron/
fi

function get_status(){
    start-stop-daemon -q -T -p /var/run/neutron/neutron-server.pid
    echo \$?
}

function do_start(){
    echo "neutron-server start/running"
    start-stop-daemon -q -b -S -x /root/venv/bin/neutron-server -m -p /var/run/neutron/neutron-server.pid -- --config-file=/etc/neutron/neutron.conf --config-file=/etc/neutron/plugins/ml2/ml2_conf.ini --log-file=/var/log/neutron/neutron-server.log
}

function do_stop(){
    echo "neutron-server stop/waiting"
    start-stop-daemon -q -K -p /var/run/neutron/neutron-server.pid
    rm -f /var/run/neutron/neutron-server.pid
}

case "$1" in

  start)
        if [ "$(get_status)" -eq 0 ]; then
            echo "start: Job is already running: neutron-server"
        else
            do_start
        fi
        ;;
  stop)
        if [ "$(get_status)" -ne 0 ]; then
            echo "stop: Unknown instance:"
        else
            do_stop
        fi
        ;;

  status)
        if [ "$(get_status)" -eq 0 ]; then
            echo "neutron-server start/running"
        else
            echo "neutron-server stop/waiting"
        fi
        ;;

  restart)
        if [ "$(get_status)" -ne 0 ]; then
            echo "stop: Unknown instance:"
        else
            do_stop
        fi
        do_start
        ;;

  *)
        echo "Usage: $0 {start|stop|status|restart}"
        exit 1
        ;;
esac

exit 0

\subsection{neutron-openvswitch-agent启动脚本}
#!/bin/bash

# chkconfig:  - 85 15  

# description: neutron-openvswitch-agent  service

#processname:neutron-openvswitch-agent


if [ ! -d /var/run/neutron/ ]; then
   sudo mkdir /var/run/neutron/
fi

function get_status(){
    start-stop-daemon -q -T -p /var/run/neutron/neutron-openvswitch-agent.pid
    echo \$?
}

function do_start(){
    echo "neutron-openvswitch-agent start/running"
    start-stop-daemon -q -b -S -x /root/venv/bin/neutron-openvswitch-agent -m -p /var/run/neutron/neutron-openvswitch-agent.pid -- --config-file=/etc/neutron/neutron.conf --config-file=/etc/neutron/plugins/ml2/openvswitch_agent.ini --log-file=/var/log/neutron/openvswitch-agent.log
}

function do_stop(){
    echo "neutron-openvswitch-agent stop/waiting"
    start-stop-daemon -q -K -p /var/run/neutron/neutron-openvswitch-agent.pid
    rm -f /var/run/neutron/neutron-openvswitch-agent.pid
}

case "$1" in

  start)
        if [ "$(get_status)" -eq 0 ]; then
            echo "start: Job is already running: neutron-openvswitch-agent"
        else
            do_start
        fi
        ;;
  stop)
        if [ "$(get_status)" -ne 0 ]; then
            echo "stop: Unknown instance:"
        else
            do_stop
        fi
        ;;

  status)
        if [ "$(get_status)" -eq 0 ]; then
            echo "neutron-openvswitch-agent start/running"
        else
            echo "neutron-openvswitch-agent stop/waiting"
        fi
        ;;

  restart)
        if [ "$(get_status)" -ne 0 ]; then
            echo "stop: Unknown instance:"
        else
            do_stop
        fi
        do_start
        ;;

  *)
        echo "Usage: $0 {start|stop|status|restart}"
        exit 1
        ;;
esac

exit 0
\subsection{neutron-l3-agent启动脚本}
#!/bin/bash

# chkconfig:  - 85 15  

# description: neutron-l3-agent  service

#processname:neutron-l3-agent

if [ ! -d /var/run/neutron/ ]; then
   sudo mkdir /var/run/neutron/
fi

function get_status(){
    start-stop-daemon -q -T -p /var/run/neutron/neutron-l3-agent.pid
    echo \$?
}

function do_start(){
    echo "neutron-l3-agent start/running"
    start-stop-daemon -q -b -S -x /root/venv/bin/neutron-l3-agent -m -p /var/run/neutron/neutron-l3-agent.pid -- --config-file=/etc/neutron/neutron.conf --config-file=/etc/neutron/l3_agent.ini --log-file=/var/log/neutron/l3-agent.log
}

function do_stop(){
    echo "neutron-l3-agent stop/waiting"
    start-stop-daemon -q -K -p /var/run/neutron/neutron-l3-agent.pid
    rm -f /var/run/neutron/neutron-l3-agent.pid
}

case "$1" in

  start)
        if [ "$(get_status)" -eq 0 ]; then
            echo "start: Job is already running: neutron-l3-agent"
        else
            do_start
        fi
        ;;
  stop)
        if [ "$(get_status)" -ne 0 ]; then
            echo "stop: Unknown instance:"
        else
            do_stop
        fi
        ;;

  status)
        if [ "$(get_status)" -eq 0 ]; then
            echo "neutron-l3-agent start/running"
        else
            echo "neutron-l3-agent stop/waiting"
        fi
        ;;

  restart)
        if [ "$(get_status)" -ne 0 ]; then
            echo "stop: Unknown instance:"
        else
            do_stop
        fi
        do_start
        ;;

  *)
        echo "Usage: $0 {start|stop|status|restart}"
        exit 1
        ;;
esac

exit 0
\subsection{neutron-dhcp-agent启动脚本}
#!/bin/bash

# chkconfig:  - 85 15  

# description: neutron-dhcp-agent  service

#processname:neutron-dhcp-agent


if [ ! -d /var/run/neutron/ ]; then
   sudo mkdir /var/run/neutron/
fi

function get_status(){
    start-stop-daemon -q -T -p /var/run/neutron/neutron-dhcp-agent.pid
    echo \$?
}

function do_start(){
    echo "neutron-dhcp-agent start/running"
    start-stop-daemon -q -b -S -x /root/venv/bin/neutron-dhcp-agent -m -p /var/run/neutron/neutron-dhcp-agent.pid -- --config-file=/etc/neutron/neutron.conf --config-file=/etc/neutron/dhcp_agent.ini --log-file=/var/log/neutron/dhcp-agent.log
}

function do_stop(){
    echo "neutron-dhcp-agent stop/waiting"
    start-stop-daemon -q -K -p /var/run/neutron/neutron-dhcp-agent.pid
    rm -f /var/run/neutron/neutron-dhcp-agent.pid
}

case "$1" in

  start)
        if [ "$(get_status)" -eq 0 ]; then
            echo "start: Job is already running: neutron-dhcp-agent"
        else
            do_start
        fi
        ;;
  stop)
        if [ "$(get_status)" -ne 0 ]; then
            echo "stop: Unknown instance:"
        else
            do_stop
        fi
        ;;

  status)
        if [ "$(get_status)" -eq 0 ]; then
            echo "neutron-dhcp-agent start/running"
        else
            echo "neutron-dhcp-agent stop/waiting"
        fi
        ;;

  restart)
        if [ "$(get_status)" -ne 0 ]; then
            echo "stop: Unknown instance:"
        else
            do_stop
        fi
        do_start
        ;;

  *)
        echo "Usage: $0 {start|stop|status|restart}"
        exit 1
        ;;
esac

exit 0
\subsection{neutron-metadata-agent启动脚本}
#!/bin/bash

# chkconfig:  - 85 15  

# description: neutron-metadata-agent  service

#processname:neutron-metadata-agent


if [ ! -d /var/run/neutron/ ]; then
   sudo mkdir /var/run/neutron/
fi

function get_status(){
    start-stop-daemon -q -T -p /var/run/neutron/neutron-metadata-agent.pid
    echo \$?
}

function do_start(){
    echo "neutron-metadata-agent start/running"
    start-stop-daemon -q -b -S -x /root/venv/bin/neutron-metadata-agent -m -p /var/run/neutron/neutron-metadata-agent.pid  -- --config-file=/etc/neutron/neutron.conf --config-file=/etc/neutron/metadata_agent.ini --log-file=/var/log/neutron/neutron-metadata-agent.log
}

function do_stop(){
    echo "neutron-metadata-agent stop/waiting"
    start-stop-daemon -q -K -p /var/run/neutron/neutron-metadata-agent.pid
    rm -f /var/run/neutron/neutron-metadata-agent.pid
}
case "$1" in

  start)
        if [ "$(get_status)" -eq 0 ]; then
            echo "start: Job is already running: neutron-metadata-agent"
        else
            do_start
        fi
        ;;
  stop)
        if [ "$(get_status)" -ne 0 ]; then
            echo "stop: Unknown instance:"
        else
            do_stop
        fi
        ;;

  status)
        if [ "$(get_status)" -eq 0 ]; then
            echo "neutron-metadata-agent start/running"
        else
            echo "neutron-metadata-agent stop/waiting"
        fi
        ;;

  restart)
        if [ "$(get_status)" -ne 0 ]; then
            echo "stop: Unknown instance:"
        else
            do_stop
        fi
        do_start
        ;;

  *)
        echo "Usage: $0 {start|stop|status|restart}"
        exit 1
        ;;
esac

exit 0
\end{document}