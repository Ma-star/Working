% !TeX spellcheck = en_US
%% 字体:方正静蕾简体
%%		 方正粗宋
\documentclass[a4paper,left=1.5cm,right=1.5cm,11pt]{article}

\usepackage[utf8]{inputenc}
\usepackage{fontspec}
\usepackage{cite}
\usepackage{xeCJK}
\usepackage{indentfirst}
\usepackage{titlesec}
\usepackage{etoolbox}%
\makeatletter
\patchcmd{\ttlh@hang}{\parindent\z@}{\parindent\z@\leavevmode}{}{}%
\patchcmd{\ttlh@hang}{\noindent}{}{}{}%
\makeatother

\usepackage{longtable}
\usepackage{empheq}
\usepackage{graphicx}
\usepackage{float}
\usepackage{rotating}
\usepackage{subfigure}
\usepackage{tabu}
\usepackage{amsmath}
\usepackage{setspace}
\usepackage{amsfonts}
\usepackage{appendix}
\usepackage{listings}
\usepackage{xcolor}
\usepackage{geometry}
\setcounter{secnumdepth}{4}
%\titleformat*{\section}{\LARGE}
%\renewcommand\refname{参考文献}
%\titleformat{\chapter}{\centering\bfseries\huge}{}{0.7em}{}{}
\titleformat{\section}{\LARGE\bf}{\thesection}{1em}{}{}
\titleformat{\subsection}{\Large\bfseries}{\thesubsection}{1em}{}{}
\titleformat{\subsubsection}{\large\bfseries}{\thesubsubsection}{1em}{}{}
\renewcommand{\contentsname}{{ \centerline{目{  } 录}}}
\setCJKfamilyfont{cjkhwxk}{STXINGKA.TTF}
%\setCJKfamilyfont{cjkhwxk}{华文行楷}
%\setCJKfamilyfont{cjkfzcs}{方正粗宋简体}
%\newcommand*{\cjkfzcs}{\CJKfamily{cjkfzcs}}
\newcommand*{\cjkhwxk}{\CJKfamily{cjkhwxk}}
%\newfontfamily\wryh{Microsoft YaHei}
%\newfontfamily\hwzs{华文中宋}
%\newfontfamily\hwst{华文宋体}
%\newfontfamily\hwfs{华文仿宋}
%\newfontfamily\jljt{方正静蕾简体}
%\newfontfamily\hwxk{华文行楷}
\newcommand{\verylarge}{\fontsize{60pt}{\baselineskip}\selectfont}  
\newcommand{\chuhao}{\fontsize{44.9pt}{\baselineskip}\selectfont}  
\newcommand{\xiaochu}{\fontsize{38.5pt}{\baselineskip}\selectfont}  
\newcommand{\yihao}{\fontsize{27.8pt}{\baselineskip}\selectfont}  
\newcommand{\xiaoyi}{\fontsize{25.7pt}{\baselineskip}\selectfont}  
\newcommand{\erhao}{\fontsize{23.5pt}{\baselineskip}\selectfont}  
\newcommand{\xiaoerhao}{\fontsize{19.3pt}{\baselineskip}\selectfont} 
\newcommand{\sihao}{\fontsize{14pt}{\baselineskip}\selectfont}      % 字号设置  
\newcommand{\xiaosihao}{\fontsize{12pt}{\baselineskip}\selectfont}  % 字号设置  
\newcommand{\wuhao}{\fontsize{10.5pt}{\baselineskip}\selectfont}    % 字号设置  
\newcommand{\xiaowuhao}{\fontsize{9pt}{\baselineskip}\selectfont}   % 字号设置  
\newcommand{\liuhao}{\fontsize{7.875pt}{\baselineskip}\selectfont}  % 字号设置  
\newcommand{\qihao}{\fontsize{5.25pt}{\baselineskip}\selectfont}    % 字号设置 

\usepackage{diagbox}
\usepackage{multirow}
\boldmath
\XeTeXlinebreaklocale "zh"
\XeTeXlinebreakskip = 0pt plus 1pt minus 0.1pt
\definecolor{cred}{rgb}{0.8,0.8,0.8}
\definecolor{cgreen}{rgb}{0,0.3,0}
\definecolor{cpurple}{rgb}{0.5,0,0.35}
\definecolor{cdocblue}{rgb}{0,0,0.3}
\definecolor{cdark}{rgb}{0.95,1.0,1.0}
\lstset{
	language=bash,
	numbers=left,
	numberstyle=\tiny\color{black},
	showspaces=false,
	showstringspaces=false,
	basicstyle=\scriptsize,
	keywordstyle=\color{purple},
	commentstyle=\itshape\color{cgreen},
	stringstyle=\color{blue},
	frame=lines,
	% escapeinside=``,
	extendedchars=true, 
	xleftmargin=1em,
	xrightmargin=1em, 
	backgroundcolor=\color{cred},
	aboveskip=1em,
	breaklines=true,
	tabsize=4
} 

%\newfontfamily{\consolas}{Consolas}
%\newfontfamily{\monaco}{Monaco}
%\setmonofont[Mapping={}]{Consolas}	%英文引号之类的正常显示,相当于设置英文字体
%\setsansfont{Consolas} %设置英文字体 Monaco, Consolas,  Fantasque Sans Mono
%\setmainfont{Times New Roman}
%\setCJKmainfont{STZHONGS.TTF}
%\setmonofont{Consolas}
% \newfontfamily{\consolas}{YaHeiConsolas.ttf}
\newfontfamily{\monaco}{MONACO.TTF}
\setCJKmainfont{STZHONGS.TTF}
%\setmainfont{MONACO.TTF}
%\setsansfont{MONACO.TTF}

\newcommand{\fic}[1]{\begin{figure}[H]
		\center
		\includegraphics[width=0.8\textwidth]{#1}
	\end{figure}}
	
\newcommand{\sizedfic}[2]{\begin{figure}[H]
		\center
		\includegraphics[width=#1\textwidth]{#2}
	\end{figure}}

\newcommand{\codefile}[1]{\lstinputlisting{#1}}

\newcommand{\interval}{\vspace{0.5em}}

\newcommand{\tablestart}{
	\interval
	\begin{longtable}{p{2cm}p{10cm}}
	\hline}
\newcommand{\tableend}{
	\hline
	\end{longtable}
	\interval}

% 改变段间隔
\setlength{\parskip}{0.2em}
\linespread{1.1}

\usepackage{lastpage}
\usepackage{fancyhdr}
\pagestyle{fancy}
\lhead{\space \qquad \space}
\chead{手动源码安装neutron\qquad}
\rhead{\qquad\thepage/\pageref{LastPage}}

\begin{document}

\tableofcontents

\clearpage

\subsection{neutron源码各文件夹作用}
    \begin{itemize}
        \item[1.]bin/:可执行的二进制脚本文件
		\item[2.]etc/:配置文件
		\item[2.]build/:没有理解到位
		\item[3.]neutron/:源码文件
		\item[4.]tools/:工具文件夹,例如install_venv.sh,安装virtualenv,建立独立的pyhton开发环境(安装neutron需要的第三方类库)
		\item[5.]run_tests.sh:安装virtualenv,并进行单元测试
		\item[6.]setup.py:利用setuptools工具,安装neutron。
		\item[7.]setup.cfg:pbr工具,解析过滤该文件,并将解析结果作为setup.py中setup函数的默认参数。
		\item[8.]requirements.txt:neutron的第三方依赖包
		\item[9.]test-requirements.txt:neutron的测试依赖包
		\item[10.]在install_venv.sh就是,利用pip安装requirements.txt,test-requirements.txt,virtualenv
    \end{itemize}
\subsection{手动源码安装步骤}
    \begin{itemize}
		\item[1.]需要安装pip版本在8.1.2之上,否则会报错
			apt-get install python-pip
			pip install --upgrade 'pip>=8.1.2'
		\item[2.]安装virtualenv
			begin{lstlisting}
				pip install virtualenv
			end{lstlisting}
		\item[3.]创建名为venv的虚拟环境,并切换	进venv目录
			begin{lstlisting}
				virtualenv venv
				cd venv
			end{lstlisting}
			安装之后使用ls命令,查看当前文件夹的内容,会多出来bin,lib,local等文件夹。以后安装的包会放在
			这些文件夹,将不会和系统的混合在一起。这就是隔离的意义。
		\item[4.]启动虚拟环境
			begin{lstlisting}
				source bin/activate
			end{lstlisting}
        \item[5.]切换到neutron源码包,安装neutron依赖包
			begin{lstlisting}
				sudo apt-get install python2.7-dev
				pip install -r requirements.txt
			end{lstlisting}
		\item[6.]安装neutron组件
			begin{lstlisting}
				apt-get install git
				python setup.py install
			end{lstlisting}
		\item[7.]通过源码的tools文件夹下的generate\_config\_file\_samples.sh脚本生成配置文件,并拷贝到/etc/neutron文件夹下
			begin{lstlisting}
				./tools/generate_config_file_samples.sh
			end{lstlisting}	
		\item[8.]在mysql创建neutron数据库,并使用keystone命令生成neutron的service和endpoint端点
			begin{lstlisting}
				mysql -uroot -phtYun@2014
				CREATE DATABASE neutron;
				GRANT ALL PRIVILEGES ON neutron.* TO 'neutron'@'localhost' \
				IDENTIFIED BY 'NEUTRON_DBPASS';
				GRANT ALL PRIVILEGES ON neutron.* TO 'neutron'@'%' \
				IDENTIFIED BY 'NEUTRON_DBPASS';

				keystone user-create --name neutron --pass neutron_pass@2014
				keystone user-role-add --user neutron --tenant service --role admin
				keystone service-create --name neutron --type network \
				--description "OpenStack Networking"

				keystone endpoint-create \
				--service-id \$(keystone service-list | awk '/ network / {print $2}') \
				--publicurl http://controller:9696 \
				--adminurl http://controller:9696 \
				--internalurl http://controller:9696 \
				--region regionOne

			end{lstlisting}	
		\item[9.]配置/etc/nova/nova.conf和/etc/neutron文件夹下的各服务配置文件,控制节点参考172.16.19.147,计算节点参考172.16.19.151。这里不再赘述
		\item[10.]一些额外的修改和安装才能运行neutron服务

			#安装memcached,pymysql
			pip install python-memcached
			pip install pymysql
			apt-get install arping 
			ln -s /root/venv/bin/neutron-ns-metadata-proxy /usr/bin/
			apt-get install conntrack
			apt-get install dnsmasq-utils
			apt-get install ipset

			#直接运行neutron-server,TypeError: set\_defaults() got an unexpected keyword argument 'sqlite_db'?
			解决办法:vi /root/venv/local/lib/python2.7/site-packages/neutron/common/config.py,在218行将该参数删除,即可。

			#vi /usr/lib/python2.7/dist-packages/nova/api/openstack/compute/servers.py
			搜索关键字floating_ips修改floating_ips的语句如下:
			floating_ips = body['server'].get('fips', None)

			#vi /root/venv/lib/python2.7/site-packages/neutron/agent/common/config.py  
			搜索关键字verbose修改verbose的条件语句如下:
			if getattr(conf, "verbose", False):
				cmd_args.append('--verbose')

			#控制节点vi /etc/neutron/rootwrap.d/l3.filters 和 dhcp.filters
			kill_metadata: KillFilter, root, python, -9
			改为
            kill_metadata: KillFilter, root, /root/venv/bin/python, -9

			#同步neutron数据库
			neutron-db-manage --config-file /etc/neutron/neutron.conf --config-file /etc/neutron/plugins/ml2/ml2\_conf.ini upgrade head

			#源码安装openvswitch-switch2.4.0
			参考另一篇文档
			#手动添加br-ex
			ovs-vsctl add-br br-ex
		
    \end{itemize}


\end{document}