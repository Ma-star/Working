% !TeX spellcheck = en_US
%% 字体:方正静蕾简体
%%		 方正粗宋
\documentclass[a4paper,left=1.5cm,right=1.5cm,11pt]{article}

\usepackage[utf8]{inputenc}
\usepackage{fontspec}
\usepackage{cite}
\usepackage{xeCJK}
\usepackage{indentfirst}
\usepackage{titlesec}
\usepackage{etoolbox}%
\makeatletter
\patchcmd{\ttlh@hang}{\parindent\z@}{\parindent\z@\leavevmode}{}{}%
\patchcmd{\ttlh@hang}{\noindent}{}{}{}%
\makeatother

\usepackage{longtable}
\usepackage{empheq}
\usepackage{graphicx}
\usepackage{float}
\usepackage{rotating}
\usepackage{subfigure}
\usepackage{tabu}
\usepackage{amsmath}
\usepackage{setspace}
\usepackage{amsfonts}
\usepackage{appendix}
\usepackage{listings}
\usepackage{xcolor}
\usepackage{geometry}
\setcounter{secnumdepth}{4}
%\titleformat*{\section}{\LARGE}
%\renewcommand\refname{参考文献}
%\titleformat{\chapter}{\centering\bfseries\huge}{}{0.7em}{}{}
\titleformat{\section}{\LARGE\bf}{\thesection}{1em}{}{}
\titleformat{\subsection}{\Large\bfseries}{\thesubsection}{1em}{}{}
\titleformat{\subsubsection}{\large\bfseries}{\thesubsubsection}{1em}{}{}
\renewcommand{\contentsname}{{ \centerline{目{  } 录}}}
\setCJKfamilyfont{cjkhwxk}{STXINGKA.TTF}
%\setCJKfamilyfont{cjkhwxk}{华文行楷}
%\setCJKfamilyfont{cjkfzcs}{方正粗宋简体}
%\newcommand*{\cjkfzcs}{\CJKfamily{cjkfzcs}}
\newcommand*{\cjkhwxk}{\CJKfamily{cjkhwxk}}
%\newfontfamily\wryh{Microsoft YaHei}
%\newfontfamily\hwzs{华文中宋}
%\newfontfamily\hwst{华文宋体}
%\newfontfamily\hwfs{华文仿宋}
%\newfontfamily\jljt{方正静蕾简体}
%\newfontfamily\hwxk{华文行楷}
\newcommand{\verylarge}{\fontsize{60pt}{\baselineskip}\selectfont}  
\newcommand{\chuhao}{\fontsize{44.9pt}{\baselineskip}\selectfont}  
\newcommand{\xiaochu}{\fontsize{38.5pt}{\baselineskip}\selectfont}  
\newcommand{\yihao}{\fontsize{27.8pt}{\baselineskip}\selectfont}  
\newcommand{\xiaoyi}{\fontsize{25.7pt}{\baselineskip}\selectfont}  
\newcommand{\erhao}{\fontsize{23.5pt}{\baselineskip}\selectfont}  
\newcommand{\xiaoerhao}{\fontsize{19.3pt}{\baselineskip}\selectfont} 
\newcommand{\sihao}{\fontsize{14pt}{\baselineskip}\selectfont}      % 字号设置  
\newcommand{\xiaosihao}{\fontsize{12pt}{\baselineskip}\selectfont}  % 字号设置  
\newcommand{\wuhao}{\fontsize{10.5pt}{\baselineskip}\selectfont}    % 字号设置  
\newcommand{\xiaowuhao}{\fontsize{9pt}{\baselineskip}\selectfont}   % 字号设置  
\newcommand{\liuhao}{\fontsize{7.875pt}{\baselineskip}\selectfont}  % 字号设置  
\newcommand{\qihao}{\fontsize{5.25pt}{\baselineskip}\selectfont}    % 字号设置 

\usepackage{diagbox}
\usepackage{multirow}
\boldmath
\XeTeXlinebreaklocale "zh"
\XeTeXlinebreakskip = 0pt plus 1pt minus 0.1pt
\definecolor{cred}{rgb}{0.8,0.8,0.8}
\definecolor{cgreen}{rgb}{0,0.3,0}
\definecolor{cpurple}{rgb}{0.5,0,0.35}
\definecolor{cdocblue}{rgb}{0,0,0.3}
\definecolor{cdark}{rgb}{0.95,1.0,1.0}
\lstset{
	language=bash,
	numbers=left,
	numberstyle=\tiny\color{black},
	showspaces=false,
	showstringspaces=false,
	basicstyle=\scriptsize,
	keywordstyle=\color{purple},
	commentstyle=\itshape\color{cgreen},
	stringstyle=\color{blue},
	frame=lines,
	% escapeinside=``,
	extendedchars=true, 
	xleftmargin=1em,
	xrightmargin=1em, 
	backgroundcolor=\color{cred},
	aboveskip=1em,
	breaklines=true,
	tabsize=4
} 

%\newfontfamily{\consolas}{Consolas}
%\newfontfamily{\monaco}{Monaco}
%\setmonofont[Mapping={}]{Consolas}	%英文引号之类的正常显示,相当于设置英文字体
%\setsansfont{Consolas} %设置英文字体 Monaco, Consolas,  Fantasque Sans Mono
%\setmainfont{Times New Roman}
%\setCJKmainfont{STZHONGS.TTF}
%\setmonofont{Consolas}
% \newfontfamily{\consolas}{YaHeiConsolas.ttf}
\newfontfamily{\monaco}{MONACO.TTF}
\setCJKmainfont{STZHONGS.TTF}
%\setmainfont{MONACO.TTF}
%\setsansfont{MONACO.TTF}

\newcommand{\fic}[1]{\begin{figure}[H]
		\center
		\includegraphics[width=0.8\textwidth]{#1}
	\end{figure}}
	
\newcommand{\sizedfic}[2]{\begin{figure}[H]
		\center
		\includegraphics[width=#1\textwidth]{#2}
	\end{figure}}

\newcommand{\codefile}[1]{\lstinputlisting{#1}}

\newcommand{\interval}{\vspace{0.5em}}

\newcommand{\tablestart}{
	\interval
	\begin{longtable}{p{2cm}p{10cm}}
	\hline}
\newcommand{\tableend}{
	\hline
	\end{longtable}
	\interval}

% 改变段间隔
\setlength{\parskip}{0.2em}
\linespread{1.1}

\usepackage{lastpage}
\usepackage{fancyhdr}
\pagestyle{fancy}
\lhead{shell变量 \qquad}
\rhead{\qquad\thepage/\pageref{LastPage}}

\begin{document}

\tableofcontents

\clearpage

\section{shell变量}
\subsection{变量相关命令}
\begin{itemize}
    \item[1.]name=weiguangzhou:设置变量
	\item[2.]echo \${name}:输出变量
	\item[3.]export name:将变量设为环境变量,在子程序中也可以使用
	\item[4.]\$(uname -r)和`uname -r`:输出命令执行的结果
	\item[5.]env:查看所有的环境变量
	\item[6.]unset:取消变量的设置
	\item[6.]alias ll='ls -alF':设置别名
	\item[7.]unalias ll:取消ll别名的设置
	\item[8.]set:查看所有环境变量和自定义变量以及其他的一些信息
	\item[9.]export:查看所有的环境变量
	\item[10.]locale:查看可支持的语言
	\item[11.]read -p "请输入你的名字:" -t 5 yourname:从键盘读取信息,赋值给yourname,-p是提示信息,-t是输入等待时间
	\item[12.]declare:不加任何参数,查看所有的变量,等同与set
	\item[13.]declare -aixr:将变量设置为数组,整型,环境变量,只读
	\item[14.]declare +x:取消之前设置的环境变量动作
	\item[15.]declare -p sum:列出变量的类型
	\item[16.]ulimit HSacfdltu:限制用户使用的系统资源,H严格的限制,s警告不能超过该配额,a显示所有配额,c核心文件大小,f普通文件大小,d最大segments大小,
	l锁定内存大小,t可使用的最大cpu时间,u单一用户可使用的最大程序数量。
\end{itemize}
\subsection{变量内容的删除与取代}
\begin{itemize}
	\item[1.]\${变量名#最开始的匹配*最后的匹配}:删除变量,从左到右最短匹配。*表示任意多个字符,#表示从左到右最短匹配,##表示从左到右最长匹配
	\item[2.]\${变量名\%最开始的匹配*最后的匹配}:删除变量,从右到左最短匹配。*表示任意多个字符,\%表示从右到左最短匹配,\%\%表示从右到左最长匹配
	\item[3.]注意:#最开始的匹配一定要是最开头,\%最后的匹配一定是最末尾
	\item[4.]\${path//sbin/SBIN}:将sbin替换成SBIN,单/表示替换一个,双/表示替换全部。
	\item[5.]上面所有操作不改变原变量的值,只产生一个新的字符串
\end{itemize}
\subsection{规则整理}
\begin{itemize}
	\item[1.]\${变量#关键词} 若变量内容从头开始的数据符合『关键词』,则将符合的最短数据删除
	\item[2.]\${变量##关键词}	若变量内容从头开始的数据符合『关键词』,则将符合的最长数据删除
	\item[3.]\${变量\%关键词} 若变量内容从尾向前的数据符合『关键词』,则将符合的最短数据删除
	\item[4.]\${变量\%\%关键词}	若变量内容从尾向前的数据符合『关键词』,则将符合的最长数据删除
	\item[5.]\${变量/旧字符串/新字符串} 若变量内容符合『旧字符串』则『第一个旧字符串会被新字符串取代』
	\item[6.]\${变量//旧字符串/新字符串} 若变量内容符合『旧字符串』则『全部的旧字符串会被新字符串取代』
\end{itemize}
\subsection{根据变量是否存在,对其进行赋值}
\begin{itemize}
	\item[1.]new_var=\${old_var:-content}:如果旧变量不存在以及为空串时则不将content内容赋值给新变量
	\item[2.]new_var=\${old_var+|=|?content}:其他组合方式暂不考虑
\end{itemize}
\subsection{命名别名与历史命令}
\begin{itemize}
	\item[1.]alias:查看系统所有别名
	\item[2.]alias hh='ll -alF':设置一个新的别名
	\item[3.]unalias hh:取消一个别名
	\item[4.]histroy n:列出前n条历史命令
\end{itemize}
\subsection{命名别名与历史命令}
\begin{itemize}
	\item[1.]source或者.:手动将配置文件的内容读取到当前shell环境中,就不需要重新注销登录了。
\end{itemize}
\end{document}