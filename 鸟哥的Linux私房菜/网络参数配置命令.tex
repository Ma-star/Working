% !TeX spellcheck = en_US
%% 字体:方正静蕾简体
%%		 方正粗宋
\documentclass[a4paper,left=1.5cm,right=1.5cm,11pt]{article}
\title{网络参数配置命令}
\author{周威光整理\footnote{简介:恒天云FTE}}
\date{2017-07-16} 
% 引用宏包
\usepackage[utf8]{inputenc}
\usepackage{fontspec}
\usepackage{cite}
\usepackage{xeCJK}
\usepackage{indentfirst}
\usepackage{titlesec}
\usepackage{etoolbox}%
\makeatletter
\patchcmd{\ttlh@hang}{\parindent\z@}{\parindent\z@\leavevmode}{}{}%
\patchcmd{\ttlh@hang}{\noindent}{}{}{}%
\makeatother

\usepackage{longtable}
\usepackage{empheq}
\usepackage{graphicx}
\usepackage{float}
\usepackage{rotating}
\usepackage{subfigure}
\usepackage{tabu}
\usepackage{amsmath}
\usepackage{setspace}
\usepackage{amsfonts}
\usepackage{appendix}
\usepackage{listings}
\usepackage{xcolor}
\usepackage{geometry}
\setcounter{secnumdepth}{4}
%\titleformat*{\section}{\LARGE}
%\renewcommand\refname{参考文献}
%\titleformat{\chapter}{\centering\bfseries\huge}{}{0.7em}{}{}
\titleformat{\section}{\LARGE\bf}{\thesection}{1em}{}{}
\titleformat{\subsection}{\Large\bfseries}{\thesubsection}{1em}{}{}
\titleformat{\subsubsection}{\large\bfseries}{\thesubsubsection}{1em}{}{}
\renewcommand{\contentsname}{{ \centerline{目{  } 录}}}
\setCJKfamilyfont{cjkhwxk}{STXINGKA.TTF}
%\setCJKfamilyfont{cjkhwxk}{华文行楷}
%\setCJKfamilyfont{cjkfzcs}{方正粗宋简体}
%\newcommand*{\cjkfzcs}{\CJKfamily{cjkfzcs}}
\newcommand*{\cjkhwxk}{\CJKfamily{cjkhwxk}}
%\newfontfamily\wryh{Microsoft YaHei}
%\newfontfamily\hwzs{华文中宋}
%\newfontfamily\hwst{华文宋体}
%\newfontfamily\hwfs{华文仿宋}
%\newfontfamily\jljt{方正静蕾简体}
%\newfontfamily\hwxk{华文行楷}
\newcommand{\verylarge}{\fontsize{60pt}{\baselineskip}\selectfont}  
\newcommand{\chuhao}{\fontsize{44.9pt}{\baselineskip}\selectfont}  
\newcommand{\xiaochu}{\fontsize{38.5pt}{\baselineskip}\selectfont}  
\newcommand{\yihao}{\fontsize{27.8pt}{\baselineskip}\selectfont}  
\newcommand{\xiaoyi}{\fontsize{25.7pt}{\baselineskip}\selectfont}  
\newcommand{\erhao}{\fontsize{23.5pt}{\baselineskip}\selectfont}  
\newcommand{\xiaoerhao}{\fontsize{19.3pt}{\baselineskip}\selectfont} 
\newcommand{\sihao}{\fontsize{14pt}{\baselineskip}\selectfont}      % 字号设置  
\newcommand{\xiaosihao}{\fontsize{12pt}{\baselineskip}\selectfont}  % 字号设置  
\newcommand{\wuhao}{\fontsize{10.5pt}{\baselineskip}\selectfont}    % 字号设置  
\newcommand{\xiaowuhao}{\fontsize{9pt}{\baselineskip}\selectfont}   % 字号设置  
\newcommand{\liuhao}{\fontsize{7.875pt}{\baselineskip}\selectfont}  % 字号设置  
\newcommand{\qihao}{\fontsize{5.25pt}{\baselineskip}\selectfont}    % 字号设置 

\usepackage{diagbox}
\usepackage{multirow}
\boldmath
\XeTeXlinebreaklocale "zh"
\XeTeXlinebreakskip = 0pt plus 1pt minus 0.1pt
\definecolor{cred}{rgb}{0.8,0.8,0.8}
\definecolor{cgreen}{rgb}{0,0.3,0}
\definecolor{cpurple}{rgb}{0.5,0,0.35}
\definecolor{cdocblue}{rgb}{0,0,0.3}
\definecolor{cdark}{rgb}{0.95,1.0,1.0}
\lstset{
	language=bash,
	numbers=left,
	numberstyle=\tiny\color{black},
	showspaces=false,
	showstringspaces=false,
	basicstyle=\scriptsize,
	keywordstyle=\color{purple},
	commentstyle=\itshape\color{cgreen},
	stringstyle=\color{blue},
	frame=lines,
	% escapeinside=``,
	extendedchars=true, 
	xleftmargin=1em,
	xrightmargin=1em, 
	backgroundcolor=\color{cred},
	aboveskip=1em,
	breaklines=true,
	tabsize=4
} 

%\newfontfamily{\consolas}{Consolas}
%\newfontfamily{\monaco}{Monaco}
%\setmonofont[Mapping={}]{Consolas}	%英文引号之类的正常显示,相当于设置英文字体
%\setsansfont{Consolas} %设置英文字体 Monaco, Consolas,  Fantasque Sans Mono
%\setmainfont{Times New Roman}
%\setCJKmainfont{STZHONGS.TTF}
%\setmonofont{Consolas}
% \newfontfamily{\consolas}{YaHeiConsolas.ttf}
\newfontfamily{\monaco}{MONACO.TTF}
\setCJKmainfont{STZHONGS.TTF}
%\setmainfont{MONACO.TTF}
%\setsansfont{MONACO.TTF}
% 自定义添加图片命令
\newcommand{\fic}[1]{\begin{figure}[H]
		\center
		\includegraphics[width=0.8\textwidth]{#1}
	\end{figure}}
	
\newcommand{\sizedfic}[2]{\begin{figure}[H]
		\center
		\includegraphics[width=#1\textwidth]{#2}
	\end{figure}}
% 
\newcommand{\codefile}[1]{\lstinputlisting{#1}}

\newcommand{\interval}{\vspace{0.5em}}

\newcommand{\tablestart}{
	\interval
	\begin{longtable}{p{2cm}p{10cm}}
	\hline}
\newcommand{\tableend}{
	\hline
	\end{longtable}
	\interval}

% 改变段间隔
\setlength{\parskip}{0.2em}
\linespread{1.1}

\usepackage{lastpage}
% 设置页眉页脚
\usepackage{fancyhdr}
\pagestyle{fancy}
\lhead{\space \qquad \space}
\chead{网络参数配置命令\qquad}
\rhead{\qquad\thepage/\pageref{LastPage}}

% 参考文献
\def\hang{\hangindent\parindent}
\def\textindent#1{\indent\llap{#1\enspace}\ignorespaces}
\def\re{\par\hang\textindent}
%超链接
\usepackage[CJKbookmarks=true,
            bookmarksnumbered=true,
            bookmarksopen=true,
            colorlinks, %注释掉此项则交叉引用为彩色边框(将colorlinks和pdfborder同时注释掉)
            pdfborder=001,   %注释掉此项则交叉引用为彩色边框
            linkcolor=green,
            anchorcolor=green,
            citecolor=green
            ]{hyperref}  

\begin{document}
\maketitle
\clearpage
\tableofcontents
\clearpage
\section{ifconfig命令}
ifconfig interface {options}\\
参数:\\
interface:网络卡接口代号,包括 eth0, eth1, ppp0 等等\\
options  :可以接的参数,包括如下:\\
    up, down :启动 (up) 或关闭 (down) 该网络接口(不涉及任何参数)\\
    mtu      :可以配置不同的 MTU 数值,例如 mtu 1500 (单位为 byte)\\
    netmask  :就是子屏蔽网络;\\
    broadcast:就是广播地址啊!
\begin{lstlisting}
	ifconfig:默认显示所有已经启动网卡的信息
	ifconfig eth0:只显示eth0的信息
	ifconfig eth0 192.168.100.100:配置eth0的ip,其他参数自动计算 
	ifconfig eth0 192.168.100.100 netmask 255.255.255.128 mtu 8000:配置eth0,同时修改mtu的值
	ifconfig eth0 mtu 8000:单独修改mtu的值
	ifconfig eth0:0 192.168.50.50:配置eth0的子网卡eth0:0
	ifconfig eth0 down|up:关闭或启动eth0网卡 
	/etc/init.d/networking restart:重启网络,上面配置的临时数据全部丢失
\end{lstlisting}
注意:ifconfig只能实时的配置网卡的三层参数,就是ip,子网掩码,无法配置二层参数(mac等)
\section{ifdown和ifup命令}
\begin{lstlisting}
#ifup或ifdown只能启动配置文件配置的接口,对于ifconfig等实时创建的接口不起作用
ifdown eth0:关闭eth0
ifup eth0:启动eth0
\end{lstlisting}

\section{route路由表命令}
\begin{lstlisting}
route [-nee]:显示路由表信息
观察的参数:
   -n  :不要使用通讯协议或主机名,直接使用 IP 或 port number;
   -ee :使用更详细的信息来显示
route add|del [-net|-host] [网域或主机] netmask [mask] [gw|dev]
#添加路由表条目
route add -net 192.168.100.0 netmask 255.255.255.0 dev eth0
#删除路由表条目
route del -net 169.254.0.0 netmask 255.255.0.0 dev eth0
#添加默认路由
route add default gw 192.168.10.30 等同于 route add -net 0.0.0.0 netmask 0.0.0.0 gw 192.168.10.30
default 等同于 -net 0.0.0.0 netmask 0.0.0.0
#注意添加路由表和删除路由表的参数要全部加上
\end{lstlisting}
\section{ip命令}
\subsection{基本格式}
\begin{lstlisting}
ip [option] [动作] [命令]
参数:
option :配置的参数,主要有:
    -s :显示出该装置的统计数据(statistics),例如总接受封包数等;
动作:亦即是可以针对哪些网络参数进行动作,包括有:
    link  :关于装置 (device) 的相关配置,包括 MTU, MAC 地址等等
    addr/address :关于额外的 IP 协议,例如多 IP 的达成等等;
    route :与路由有关的相关配置
\end{lstlisting}
\subsection{ip link命令}
主要用于更新二层的参数
\begin{lstlisting}
ip [-s] link show
ip link set [device] [动作与参数]

show:仅显示出这个装置的相关内容,如果加上 -s 会显示更多统计数据;
set :可以开始配置项目, device 指的是 eth0, eth1 等等界面代号;
动作与参数:包括有底下的这些动作:
   up|down  :启动 (up) 或关闭 (down) 某个接口,其他参数使用默认的以太网络;
   address  :如果这个装置可以更改 MAC 的话,用这个参数修改!
   name     :给予这个装置一个特殊的名字;
   mtu      :就是最大传输单元啊!

ip link show 等同于 ip link list:显示所有接口的硬件信息
ip link show eth0:显示某一接口的信息
ip link set eth0 up|down:启动或关闭eth0接口
ip link set eth0 name xxx:更改eth0的网卡名
ip link set eth0 address xx:xx:xx:xx:xx:xx:更改eth0的mac
ip link set eth0 mtu xx:更改eht0的mtu

添加删除虚拟网卡
ip link add link eth0 name eth0.100 type vlan id 100
ip link del eth0.100 type vlan
\end{lstlisting}
\subsection{ip addr/address命令}
主要用于更新三层的参数
\begin{lstlisting}
ip address show   <==就是查阅 IP 参数啊!
ip address [add|del] [IP参数] [dev 装置名] [相关参数]

ip address show:显示所有接口的三层信息
ip address add 192.168.50.50/24 dev eth0
ip address del 192.168.50.50/24 dev eth0
\end{lstlisting}
\subsection{ip route命令}
\begin{lstlisting}
ip route show:显示路由表
ip route [add|del] [IP或网域] [via gateway] [dev 装置]
ip route add 192.168.5.0/24 dev eth0
ip route add 192.168.10.0/24 via 192.168.5.100 dev eth0
ip route add default via 192.168.1.2 dev eth0
ip route del 192.168.10.0/24
\end{lstlisting}
\clearpage
\begin{center}%参考文献的书写
参考文献
\end{center}
\re{[1]} \href{http://cn.linux.vbird.org/linux_server/0140networkcommand/0140networkcommand-centos4.php}{Linux 常用网络命令介绍} 
\end{document}