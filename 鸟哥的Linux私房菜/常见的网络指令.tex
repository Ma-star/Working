% !TeX spellcheck = en_US
%% 字体:方正静蕾简体
%%		 方正粗宋
\documentclass[a4paper,left=1.5cm,right=1.5cm,11pt]{article}

\usepackage[utf8]{inputenc}
\usepackage{fontspec}
\usepackage{cite}
\usepackage{xeCJK}
\usepackage{indentfirst}
\usepackage{titlesec}
\usepackage{etoolbox}%
\makeatletter
\patchcmd{\ttlh@hang}{\parindent\z@}{\parindent\z@\leavevmode}{}{}%
\patchcmd{\ttlh@hang}{\noindent}{}{}{}%
\makeatother

\usepackage{longtable}
\usepackage{empheq}
\usepackage{graphicx}
\usepackage{float}
\usepackage{rotating}
\usepackage{subfigure}
\usepackage{tabu}
\usepackage{amsmath}
\usepackage{setspace}
\usepackage{amsfonts}
\usepackage{appendix}
\usepackage{listings}
\usepackage{xcolor}
\usepackage{geometry}
\setcounter{secnumdepth}{4}
%\titleformat*{\section}{\LARGE}
%\renewcommand\refname{参考文献}
%\titleformat{\chapter}{\centering\bfseries\huge}{}{0.7em}{}{}
\titleformat{\section}{\LARGE\bf}{\thesection}{1em}{}{}
\titleformat{\subsection}{\Large\bfseries}{\thesubsection}{1em}{}{}
\titleformat{\subsubsection}{\large\bfseries}{\thesubsubsection}{1em}{}{}
\renewcommand{\contentsname}{{ \centerline{目{  } 录}}}
\setCJKfamilyfont{cjkhwxk}{STXINGKA.TTF}
%\setCJKfamilyfont{cjkhwxk}{华文行楷}
%\setCJKfamilyfont{cjkfzcs}{方正粗宋简体}
%\newcommand*{\cjkfzcs}{\CJKfamily{cjkfzcs}}
\newcommand*{\cjkhwxk}{\CJKfamily{cjkhwxk}}
%\newfontfamily\wryh{Microsoft YaHei}
%\newfontfamily\hwzs{华文中宋}
%\newfontfamily\hwst{华文宋体}
%\newfontfamily\hwfs{华文仿宋}
%\newfontfamily\jljt{方正静蕾简体}
%\newfontfamily\hwxk{华文行楷}
\newcommand{\verylarge}{\fontsize{60pt}{\baselineskip}\selectfont}  
\newcommand{\chuhao}{\fontsize{44.9pt}{\baselineskip}\selectfont}  
\newcommand{\xiaochu}{\fontsize{38.5pt}{\baselineskip}\selectfont}  
\newcommand{\yihao}{\fontsize{27.8pt}{\baselineskip}\selectfont}  
\newcommand{\xiaoyi}{\fontsize{25.7pt}{\baselineskip}\selectfont}  
\newcommand{\erhao}{\fontsize{23.5pt}{\baselineskip}\selectfont}  
\newcommand{\xiaoerhao}{\fontsize{19.3pt}{\baselineskip}\selectfont} 
\newcommand{\sihao}{\fontsize{14pt}{\baselineskip}\selectfont}      % 字号设置  
\newcommand{\xiaosihao}{\fontsize{12pt}{\baselineskip}\selectfont}  % 字号设置  
\newcommand{\wuhao}{\fontsize{10.5pt}{\baselineskip}\selectfont}    % 字号设置  
\newcommand{\xiaowuhao}{\fontsize{9pt}{\baselineskip}\selectfont}   % 字号设置  
\newcommand{\liuhao}{\fontsize{7.875pt}{\baselineskip}\selectfont}  % 字号设置  
\newcommand{\qihao}{\fontsize{5.25pt}{\baselineskip}\selectfont}    % 字号设置 

\usepackage{diagbox}
\usepackage{multirow}
\boldmath
\XeTeXlinebreaklocale "zh"
\XeTeXlinebreakskip = 0pt plus 1pt minus 0.1pt
\definecolor{cred}{rgb}{0.8,0.8,0.8}
\definecolor{cgreen}{rgb}{0,0.3,0}
\definecolor{cpurple}{rgb}{0.5,0,0.35}
\definecolor{cdocblue}{rgb}{0,0,0.3}
\definecolor{cdark}{rgb}{0.95,1.0,1.0}
\lstset{
	language=bash,
	numbers=left,
	numberstyle=\tiny\color{black},
	showspaces=false,
	showstringspaces=false,
	basicstyle=\scriptsize,
	keywordstyle=\color{purple},
	commentstyle=\itshape\color{cgreen},
	stringstyle=\color{blue},
	frame=lines,
	% escapeinside=``,
	extendedchars=true, 
	xleftmargin=1em,
	xrightmargin=1em, 
	backgroundcolor=\color{cred},
	aboveskip=1em,
	breaklines=true,
	tabsize=4
} 

%\newfontfamily{\consolas}{Consolas}
%\newfontfamily{\monaco}{Monaco}
%\setmonofont[Mapping={}]{Consolas}	%英文引号之类的正常显示,相当于设置英文字体
%\setsansfont{Consolas} %设置英文字体 Monaco, Consolas,  Fantasque Sans Mono
%\setmainfont{Times New Roman}
%\setCJKmainfont{STZHONGS.TTF}
%\setmonofont{Consolas}
% \newfontfamily{\consolas}{YaHeiConsolas.ttf}
\newfontfamily{\monaco}{MONACO.TTF}
\setCJKmainfont{STZHONGS.TTF}
%\setmainfont{MONACO.TTF}
%\setsansfont{MONACO.TTF}

\newcommand{\fic}[1]{\begin{figure}[H]
		\center
		\includegraphics[width=0.8\textwidth]{#1}
	\end{figure}}
	
\newcommand{\sizedfic}[2]{\begin{figure}[H]
		\center
		\includegraphics[width=#1\textwidth]{#2}
	\end{figure}}

\newcommand{\codefile}[1]{\lstinputlisting{#1}}

\newcommand{\interval}{\vspace{0.5em}}

\newcommand{\tablestart}{
	\interval
	\begin{longtable}{p{2cm}p{10cm}}
	\hline}
\newcommand{\tableend}{
	\hline
	\end{longtable}
	\interval}

% 改变段间隔
\setlength{\parskip}{0.2em}
\linespread{1.1}

\usepackage{lastpage}
\usepackage{fancyhdr}
\pagestyle{fancy}
\lhead{常见的网络指令 \qquad}
\rhead{\qquad\thepage/\pageref{LastPage}}

\begin{document}

\tableofcontents

\clearpage

\section{网络参数配置命令}
\subsection{ifconfig,ifup,ifdown}
ifconfig:查看所有已经启动的接口,无论是否有ip
ifconfig interface:查看对应接口的详细信息
\begin{itemize}信息描述:
    \item[1.]eth0:就是网络卡的代号,也有 lo 这个 loopback ;
    \item[2.]HWaddr:就是网络卡的硬件地址,俗称的 MAC 是也;
    \item[3.]inet addr:IPv4 的 IP 地址,后续的 Bcast, Mask 分别代表的是 Broadcast 与 netmask 喔!
    \item[4.]inet6 addr:是 IPv6 的版本的 IP ,我们没有使用,所以略过;
    \item[5.]MTU:就是 MTU 啊!
    \item[6.]RX:那一行代表的是网络由启动到目前为止的封包接收情况, packets 代表封包数、errors 代表封包发生错误的数量、 dropped 代表封包由于有问题而遭丢弃的数量等等
    \item[7.]TX:与 RX 相反,为网络由启动到目前为止的传送情况;
    \item[8.]collisions:代表封包碰撞的情况,如果发生太多次, 表示您的网络状况不太好;
    \item[9.]txqueuelen:代表用来传输数据的缓冲区的储存长度;
    \item[10.]RX bytes, TX bytes:总传送、接收的字节总量
    \item[11.]Interrupt, Memory:网络卡硬件的数据, IRQ 岔断与内存地址;
\end{itemize}
ifup,ifdown:启动和关闭接口
\subsection{路由修改route}
\begin{itemize}信息描述:
    \item[1.]route -n:信息显示的时候,以ip显示,而不是主机名
    \item[2.]route add [-net|-host]     [gw GW] [netmask NM] [dev]
    \item[3.]route del [-net|-host] target [gw GW] [netmask NM] [dev]
    \item[4.]例如:route del -net 169.254.0.0 netmask 255.255.0.0 dev eth0
    \item[5.]注意:删除的时候要按照添加的参数全部写入才行!!!! 
    \item[6.]当出现『SIOCADDRT: Network is unreachable』 这个错误时,肯定是由于 gw 后面接的 IP 无法直接与您的网域沟通 (Gateway 并不在你的网域内)
\end{itemize}
\subsection{ip命令}
[root@linux ~]# ip [option] [动作] [命令]
参数:
option :配置的参数,主要有:
    -s :显示出该装置的统计数据(statistics),例如总接受封包数等;
动作:亦即是可以针对哪些网络参数进行动作,包括有:
    link  :关于装置 (device) 的相关配置,包括 MTU, MAC 地址等等
    addr/address :关于额外的 IP 协议,例如多 IP 的达成等等;
    route :与路由有关的相关配置
\begin{itemize}
    \item[1.]ip [-s] link show  <== 单纯的查阅该装置相关的信息
            ip link set [device] [动作与参数]
            参数:
            show:仅显示出这个装置的相关内容,如果加上 -s 会显示更多统计数据;
            set :可以开始配置项目, device 指的是 eth0, eth1 等等界面代号;
            动作与参数:包括有底下的这些动作:
            up|down  :启动 (up) 或关闭 (down) 某个接口,其他参数使用默认的以太网络;
            address  :如果这个装置可以更改 MAC 的话,用这个参数修改!
            name     :给予这个装置一个特殊的名字;
            mtu      :就是最大传输单元啊!
    \item[2.]ip address show   <==就是查阅 IP 参数啊!
            ip address [add|del] [IP参数] [dev 装置名] [相关参数]
            参数:
            show    :单纯的显示出接口的 IP 信息啊;
            add|del :进行相关参数的添加 (add) 或删除 (del) 配置,主要有:
                IP 参数:主要就是网域的配置,例如 192.168.100.100/24 之类的配置喔;
                dev:这个 IP 参数所要配置的接口,例如 eth0, eth1 等等;
                相关参数:主要有底下这些:
                    broadcast:配置广播地址,如果配置值是 + 表示『让系统自动计算』
                    label:亦即是这个装置的别名,例如 eth0:0 就是了!
                    scope:这个界面的领域,通常是这几个大类:
                            global:允许来自所有来源的联机;
                            site:仅支持 IPv6 ,仅允许本主机的联机;
                            link:仅允许本装置自我联机;
                            host:仅允许本主机内部的联机;
                            所以当然是使用 global 啰!默认也是 global 啦!
    \item[3.]ip route show  <==单纯的显示出路由的配置而已
            ip route [add|del] [IP或网域] [via gateway] [dev 装置]
            参数:
            show :单纯的显示出路由表,也可以使用 list ;
            add|del :添加 (add) 或删除 (del) 路由的意思。
                IP或网域:可使用 192.168.50.0/24 之类的网域或者是单纯的 IP ;
                via     :从那个 gateway 出去,不一定需要;
                dev     :由那个装置连出去,这就需要了!
                mtu     :可以额外的配置 MTU 的数值喔!
\subsection{dhclient命令}
发送广播消息,向dnsmasq请求ip地址
dhclient -d -4 -q eth0:在前台运行程序,请求ipv4地址,安静模式遇到错误不打印,eth0是请求的接口

\end{itemize}


-----------------------------------------------------------tcpdump-----------------------------------------------------------
\section{抓包命令}
\subsection{tcpdump:抓取经过某个端口的包,并过滤,显示相应的包}
[root@linux ~]# tcpdump [-nn] [-i 接口] [-w 储存档名] [-c 次数] [-Ae]
                        [-qX] [-r 文件] [所欲撷取的数据内容]
参数:
-nn:直接以 IP 及 port number 显示,而非主机名与服务名称
-i :后面接要『监听』的网络接口,例如 eth0, lo, ppp0 等等的界面;
-w :如果你要将监听所得的封包数据储存下来,用这个参数就对了!后面接档名
-c :监听的封包数,如果没有这个参数, tcpdump 会持续不断的监听,
     直到使用者输入 [ctrl]-c 为止。
-A :封包的内容以 ASCII 显示,通常用来捉取 WWW 的网页封包数据。
-e :使用数据连接层 (OSI 第二层) 的 MAC 封包数据来显示;
-q :仅列出较为简短的封包信息,每一行的内容比较精简
-X :可以列出十六进制 (hex) 以及 ASCII 的封包内容,对于监听封包内容很有用
-r :从后面接的文件将封包数据读出来。那个『文件』是已经存在的文件,
     并且这个『文件』是由 -w 所制作出来的。
所欲撷取的数据内容:我们可以专门针对某些通讯协议或者是 IP 来源进行封包撷取,
     那就可以简化输出的结果,并取得最有用的信息。常见的表示方法有:
     'host foo', 'host 127.0.0.1' :针对单部主机来进行封包撷取
     'net 192.168' :针对某个网域来进行封包的撷取;
     'src host 127.0.0.1' 'dst net 192.168':同时加上来源(src)或目标(dst)限制
     'tcp port 21':还可以针对通讯协议侦测,如 tcp, udp, arp, ether 等
     还可以利用 and 与 or 来进行封包数据的整合显示呢!
-----------------------------------------------------------telnet-----------------------------------------------------------
\section{远程联机命令}
\subsection{telnet:远程登录到主机,明文传输数据}
要想登录到远程主机,必须在远程主机安装telnet服务端。ubuntu默认安装telnet客户端,服务端需要手动安装
telnet:对应客户端;telnetd:对应服务端;具体安装详见另一篇文章
telnet ip port:查看远程主机的port端口是否开启
telnet ip:远程登录到主机

-----------------------------------------------------------ping traceroute netstat nmap host nslookup-----------------------------------------------------------
\section{网络侦错与观察命令}
\subsection{ping:本地主机向待检测主机发送icmp数据包要求其响应,用于检测是否存在于网络环境中}
ping [-bcstnM] IP
参数:
-b :后面接的是 broadcast 的 IP,用在你『需要对整个网域的主机进行 ping 』时;
-c :后面接的是运行 ping 的次数,例如 -c 5 ;
-n :不进行 IP 与主机名的反查,直接使用 IP ;
-s :发送出去的 ICMP 封包大小,默认为 56(bytes),再加 8 bytes 的 ICMP 表头数据
-t :TTL 的数值,默认是 255,每经过一个节点就会少一;
-M [do|dont] :主要在侦测网络的 MTU 数值大小,两个常见的项目是:
   do  :代表传送一个 DF (Don't Fragment) 旗标,让封包不能重新拆包与打包;
   dont:代表不要传送 DF 旗标,表示封包可以在其他主机上拆包与打包

\subsection{traceroute:检测到目的主机所经过的节点数}
对每个节点会检测3次,回复时间大于5秒的,会输出×号 
traceroute [-nwig] IP
参数:
-n :可以不必进行主机的名称解析,单纯用 IP ,速度较快!
-w :若对方主机在几秒钟内没有回声就宣告不治...默认是 5 秒
-i :用在比较复杂的环境,如果你的网络接口很多很复杂时,才会用到这个参数;
     举例来说,你有两条 ADSL 可以连接到外部,那你的主机会有两个 ppp,
     你可以使用 -i 来选择是 ppp0 还是 ppp1 啦!
-g :与 -i 的参数相仿,只是 -g 后面接的是 gateway 的 IP 就是了。

\subsection{netstat:查看本机的服务,端口是否开启}
    \subsubsection{查看服务对应端口映射(相当于一个字典)}
    \begin{lstlisting}
        vi /etc/services
    \end{lstlisting}
    \subsubsection{查看哪些服务在运行,对应的端口是什么?}
    \begin{itemize}
    \item[1.]netstat查看的是本机的
        netstat :命令不带参数显示本机所有建立链接的服务
        下面是相关参数介绍:
        -a:显示监听的和非监听的服务,即本机开启的服务,所有已经建立的连接
        -l:显示监听的服务,即本机开启的服务
        -t:显示tcp的
        -u:显示udp相关的
        -n:以ip地址和端口号显示
        -p:显示进程号
    \item[2.]nmap可以查看指定ip的服务启动情况
        nmap可以查看远程主机的启动情况
        nmap ip:查看对应ip的主机的所有已经启动的服务
        nmap ip port:查看对应ip的对应port是否启动
    \end{itemize}
\subsection{host:列出主机名对应的ip地址}
    host [-a] hostname [server]
    参数:
    -a :列出该主机详细的各项主机名配置数据
    [server] :可以使用非为 /etc/resolv.conf 的 DNS 主机来查询。
    host www.baidu.com 8.8.4.4  :使用8.8.4.4这个dns服务器,对www.baidu.com进行解析
\subsection{nslookup:不仅可以查主机名到ip,还可以查ip到主机名}
     nslookup [-query=[type]] [hostname|IP]
     参数:
     -query=type:查询的类型,除了传统的 IP 与主机名对应外,DNS 还有很多信息,
                所以我们可以查询很多不同的信息,包括 mx, cname 等等,
                例如: -query=mx 的查询方法!
\end{document}