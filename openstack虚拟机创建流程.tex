% !TeX spellcheck = en_US
%% 字体:方正静蕾简体
%%		 方正粗宋
\documentclass[a4paper,left=1.5cm,right=1.5cm,11pt]{article}

\usepackage[utf8]{inputenc}
\usepackage{fontspec}
\usepackage{cite}
\usepackage{xeCJK}
\usepackage{indentfirst}
\usepackage{titlesec}
\usepackage{etoolbox}%
\makeatletter
\patchcmd{\ttlh@hang}{\parindent\z@}{\parindent\z@\leavevmode}{}{}%
\patchcmd{\ttlh@hang}{\noindent}{}{}{}%
\makeatother

\usepackage{longtable}
\usepackage{empheq}
\usepackage{graphicx}
\usepackage{float}
\usepackage{rotating}
\usepackage{subfigure}
\usepackage{tabu}
\usepackage{amsmath}
\usepackage{setspace}
\usepackage{amsfonts}
\usepackage{appendix}
\usepackage{listings}
\usepackage{xcolor}
\usepackage{geometry}
\setcounter{secnumdepth}{4}
%\titleformat*{\section}{\LARGE}
%\renewcommand\refname{参考文献}
%\titleformat{\chapter}{\centering\bfseries\huge}{}{0.7em}{}{}
\titleformat{\section}{\LARGE\bf}{\thesection}{1em}{}{}
\titleformat{\subsection}{\Large\bfseries}{\thesubsection}{1em}{}{}
\titleformat{\subsubsection}{\large\bfseries}{\thesubsubsection}{1em}{}{}
\renewcommand{\contentsname}{{ \centerline{目{  } 录}}}
\setCJKfamilyfont{cjkhwxk}{STXINGKA.TTF}
%\setCJKfamilyfont{cjkhwxk}{华文行楷}
%\setCJKfamilyfont{cjkfzcs}{方正粗宋简体}
%\newcommand*{\cjkfzcs}{\CJKfamily{cjkfzcs}}
\newcommand*{\cjkhwxk}{\CJKfamily{cjkhwxk}}
%\newfontfamily\wryh{Microsoft YaHei}
%\newfontfamily\hwzs{华文中宋}
%\newfontfamily\hwst{华文宋体}
%\newfontfamily\hwfs{华文仿宋}
%\newfontfamily\jljt{方正静蕾简体}
%\newfontfamily\hwxk{华文行楷}
\newcommand{\verylarge}{\fontsize{60pt}{\baselineskip}\selectfont}  
\newcommand{\chuhao}{\fontsize{44.9pt}{\baselineskip}\selectfont}  
\newcommand{\xiaochu}{\fontsize{38.5pt}{\baselineskip}\selectfont}  
\newcommand{\yihao}{\fontsize{27.8pt}{\baselineskip}\selectfont}  
\newcommand{\xiaoyi}{\fontsize{25.7pt}{\baselineskip}\selectfont}  
\newcommand{\erhao}{\fontsize{23.5pt}{\baselineskip}\selectfont}  
\newcommand{\xiaoerhao}{\fontsize{19.3pt}{\baselineskip}\selectfont} 
\newcommand{\sihao}{\fontsize{14pt}{\baselineskip}\selectfont}      % 字号设置  
\newcommand{\xiaosihao}{\fontsize{12pt}{\baselineskip}\selectfont}  % 字号设置  
\newcommand{\wuhao}{\fontsize{10.5pt}{\baselineskip}\selectfont}    % 字号设置  
\newcommand{\xiaowuhao}{\fontsize{9pt}{\baselineskip}\selectfont}   % 字号设置  
\newcommand{\liuhao}{\fontsize{7.875pt}{\baselineskip}\selectfont}  % 字号设置  
\newcommand{\qihao}{\fontsize{5.25pt}{\baselineskip}\selectfont}    % 字号设置 

\usepackage{diagbox}
\usepackage{multirow}
\boldmath
\XeTeXlinebreaklocale "zh"
\XeTeXlinebreakskip = 0pt plus 1pt minus 0.1pt
\definecolor{cred}{rgb}{0.8,0.8,0.8}
\definecolor{cgreen}{rgb}{0,0.3,0}
\definecolor{cpurple}{rgb}{0.5,0,0.35}
\definecolor{cdocblue}{rgb}{0,0,0.3}
\definecolor{cdark}{rgb}{0.95,1.0,1.0}
\lstset{
	language=bash,
	numbers=left,
	numberstyle=\tiny\color{black},
	showspaces=false,
	showstringspaces=false,
	basicstyle=\scriptsize,
	keywordstyle=\color{purple},
	commentstyle=\itshape\color{cgreen},
	stringstyle=\color{blue},
	frame=lines,
	% escapeinside=``,
	extendedchars=true, 
	xleftmargin=1em,
	xrightmargin=1em, 
	backgroundcolor=\color{cred},
	aboveskip=1em,
	breaklines=true,
	tabsize=4
} 

%\newfontfamily{\consolas}{Consolas}
%\newfontfamily{\monaco}{Monaco}
%\setmonofont[Mapping={}]{Consolas}	%英文引号之类的正常显示,相当于设置英文字体
%\setsansfont{Consolas} %设置英文字体 Monaco, Consolas,  Fantasque Sans Mono
%\setmainfont{Times New Roman}
%\setCJKmainfont{STZHONGS.TTF}
%\setmonofont{Consolas}
% \newfontfamily{\consolas}{YaHeiConsolas.ttf}
\newfontfamily{\monaco}{MONACO.TTF}
\setCJKmainfont{STZHONGS.TTF}
%\setmainfont{MONACO.TTF}
%\setsansfont{MONACO.TTF}

\newcommand{\fic}[1]{\begin{figure}[H]
		\center
		\includegraphics[width=0.8\textwidth]{#1}
	\end{figure}}
	
\newcommand{\sizedfic}[2]{\begin{figure}[H]
		\center
		\includegraphics[width=#1\textwidth]{#2}
	\end{figure}}

\newcommand{\codefile}[1]{\lstinputlisting{#1}}

\newcommand{\interval}{\vspace{0.5em}}

\newcommand{\tablestart}{
	\interval
	\begin{longtable}{p{2cm}p{10cm}}
	\hline}
\newcommand{\tableend}{
	\hline
	\end{longtable}
	\interval}

% 改变段间隔
\setlength{\parskip}{0.2em}
\linespread{1.1}

\usepackage{lastpage}
\usepackage{fancyhdr}
\pagestyle{fancy}
\lhead{\space \qquad \space}
\chead{neutron相关知识\qquad}
\rhead{\qquad\thepage/\pageref{LastPage}}

\begin{document}

\tableofcontents

\clearpage

\section{}
根据官方提供的流程图,我个人将OpenStack云主机的创建步骤分为四个阶段:
KeyStone验证阶段
Nova服务组件交换
OpenStack其它服务交换
执行创建
\subsection{第一阶段:KeyStone验证}
1.用户使用Dashboard Horizon或者命令行CLI,通过REST API给Identity 服务Keystone发送用户凭据(credentials)并验证(authenticates)。
Keystone使用用户凭据进行验证,然后返回一个auth-token。然后后续操作就可以使用这个auth-token通过REST调用请求OpenStack其他的组件。
2.Horizon或者CLI将launch instance 或者nova-boot转换形成为一个REST API的请求发送给nova-api。
3.nova-api接到这个请求后,首先向keystone发送一个请求来确认auth-token是否有效和是否有访问权限。Keystone确认auth-token后,发送一个包含角色和权限的更新后的认证头。
4.nova-api和Nova数据库交互,将用户的创建虚拟机的请求在nova 数据库里记录下来。

\subsection{第二阶段:Nova服务组件交互}
5.nova-api以rpc.call的方式发送一个请求给nova-schedule,让nova-scheduler去选择一个计算节点来创建虚拟机。注意是通过消息队列发送给nova-scheduler。
6.nova-schedule调度服务会侦听Scheduler队列,从队列中获取数据。
7.nova-scheduler和Nova数据库交互,通过调度算法,也就是filtering 和weighing最终选择一台运行nova-compute的计算节点,然后nova-schedule将虚拟机信息使用rpc.cast的模式发送至nova-compute.计算节点队列。让nova-compute在选择好的计算节点中去创建实例。
8.Nova-Compute从队列获取请求。
9.nova-compute发送一个rpc.call 请求给nova-conductor,去获取实例的信息,比如host ID和选择的Flavor(CPU、内存和磁盘)。
10.nova-conductor从队列中获取请求。
11.nova-conductor与nova的数据库进行交互。nova-conductor返回实例的信息。nova-compute从队列中获取实例的信息。

\subsection{第三阶段:OpenStack其它服务交互}
在第二阶段nova-compute为了获取到创建实例所需要的资源,比如镜像、网络、存储。
会使用在第一阶段用户验证后获取到的auth-tokon分别和镜像服务Glance的glance-api,网络服务Neutron的neutron-server已经块存储服务Cinder的cinder-api进行交互。
而且每次对方收到请求后都需要到keystone上去验证auth-token是否有效。
12.nova-compute使用验证后获取的auth-token发起一个REST调用给glance-api获取镜像。然后nova-compute使用使用镜像ID。
从镜像服务中得到Image URI。从(image storage)镜像存储中加载镜像。
13.glance-api去Keystone上验证auth-token是否有效。如果有效,nova-compute就可以获取镜像的元数据metadata。
14.nova-compute使用验证后获取的auth-token执行一个REST调用给neutron-server,让neutron-server给分配和配置网络,为实例分配IP地址。
15.neutron-Server去Keystone验证auth-token是否有效。如果有效,nova-compute就可以获取到网络的相关信息。
16.nova-compute使用验证后获取的auth-token执行一个REST调用给cinder-api,给实例附加卷存储,也就是云硬盘。
17.cinder-api去Keystone验证auth-token是否有效,如果有效,那么nova-compute就可以获取到块存储的相关信息。

\subsection{第四阶段:执行创建}
在第三阶段,nova-compute已经通过Glance、Neutron和Cinder分别获取到了镜像、网络和存储相关的信息。那么在第四阶段nova-compute就开始创建虚拟机了。
18.nova-compute为hypervisor的驱动生成数据,并且通过libvirt或者其他API让hypervisor执行请求来创建虚拟机。
这样虚拟机创建的交互流程基本结束,剩下的步骤就是hypervisor最终创建虚拟机的流程。
然后nova-api去轮训nova database,查看虚拟机的状态是否变成正确创建虚拟机的状态(Active,none,sunning),若状态正确,则虚拟机创建正常成功。
	
\end{document}